\subsection{Projekta komunikācijas formas}
%3.1.6
Ik katrs projekts var būt veiksmīgs vai neizdoties tik vien dēļ tā ka (ne)tika nodrošināta
pienācīga komunikācija starp projekta dalībniekiem. It sevišķi uz šo var novērot projektos, kuri tiek
realizēti vairākās lokācijās, dažreiz pat diezgan attālās valstīs.
\par
Autors izveidoja dažādu sanāksmju aprakstu \ref{app:Projekta_komunikacijas_formas} pielikumā. Tajā tiek apkopoti
dati par komunikācijas veidu, saturu, dalībniekiem, cik bieži tā notiks un kurā vietā.
\par
Pati pirmā kopīgā sanāksme ir projekta starta, jeb kick-off sanāksme, tajā tiek izrunāti projekta mērķi, komanda
tiek savstarpēji iepazīstināta, kā arī iepazīstināta ar projekta saturu, mērķiem, vīziju utt.
Nākošā sanāksme notiek katru dienu, tās garums maksimāli ir 15 minūtes, taču kopumā katram dalībniekam ir iedotas vien
pāris minūtes, lai ātri un precīzi atibildēt uz šiem jautājumiem: 1) kas tika paveikts vakar, 2) kādi plāni šodienai,
3) vai ir kāds bloķējošs notikums. Šī sanāksme var notikt gan klātienē, gan izmantojot dažādus digitālos risinājumus.
Komandas operatīvās sanāksmes notiek reizi nedēļā. Tās ir npeieciešanas, lai laicīgi varētu noteikt novirzes no
projekta plāna, tajā tiek iztirzātas dažādas problēmas, kuras eksistē komandas iekšienē. Abās iepriekšējās piedalās
projekta komanda, taču projekta uzdevuma devējs var arī ierasties uz šīm, ja viņam ir laiks un interese.
Nākošā sanāksme notiek starp projekta vadītāju un proejtak uzdevuma devēju. Tajā projekta vadītājs dod nelielu
atskaiti par paveikto, tiek izrunāti projektam aktuālie jautājumi. Kad projekts nonāk finiša taisnē komanda atrāda 
sevis paveikto projekta komandas prezentācijā. Tajā piedalās visi interesenti, protams projekta, vadītājam projekta
komandai un projekta uzdevuma devējām ir jābūt klāt obligāti. Kad projekts ir veiksmīgi pabeigts tiek apkopota pēdējā
informācija, projekta uzdevuma devējs paraksta pieņemšanas - nodošanas aktu un noslēdz projektu.
\par
Bez pienācīgas komunikācijas projektam var klāties samērā bēdīgi. Augstāk tika aprkastīas nepieciešamās sanāksmes, kuras,
autora prāt, izlīdzēs projekta vadības gaitā.
\par
Nākošā nodaļa tiek apskatīts projekta plānošanas apraksts. Tajā tiek analīzēts projekta struktūrplāns, gaita un termiņi,
resursus un izmaksas.
