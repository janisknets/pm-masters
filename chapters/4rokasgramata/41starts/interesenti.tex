\subsection{Projekta interesentu analize}
%3.1.2
Projekta interesenti parasti tie iedalīti trīs lielākās kategorijās -
Projekta komanda, projekta iesaistītas personas un projekta ārējās personas.
\par
Dotajā projekta projekta komanda sastāv no 3 cilvēkiem - Projekta vadītāja, jeb PV,
projekta vadītāja tehniskā asistenta, jeb PVT un projekta vadītāja mācību materiālu
asistents, jeb PVM. Šie trīs cilvēki vadīs un īstenos doto projektu. No ieinteresētām
personām ir projekta uzdevuma devējis - PUD, tad ir Start(IT) fonda valde, kura
attiecīgi finansē doto projektu. \acrshort{izm} ir pēdējais iekšējais interesents, jo 
mācību priekšmeta saturs tiks veidots ar \acrshort{visc} palīdzību. \acrshort{visc} arī 
ir ieinteresēts iegūt jaunus un kvalitatīvus mācību materiālus. Ārējie interesenti ir
skolotāji, kā gala produkta lietotāji, skolnieki, kuri saņems šo produktu, citi Latvijas IT uzņēmumi,
jo tie potenciāli varētu atbalstīt šo iniciatīvu, kā arī pēc vairākiem gadiem, viņi varēs iegūt darbiniekus,
kuri kādreiz tika apmācīti skolās pateicoties dotajai mācību programmai. Citi Latvijas 
uzņēmumi arī būtu ieinteresēti dotā projekta produktā, jo viņi gan spētu izmantot izveidotos
materiālus arī saviem nolūkiem, kā arī pēc vairākiem gadiem arī spēs iegūt labāk kvalificētus
darbiniekus. Dotās programmas veiksmīgā izpildē ir arī ieintersēts \acrshort{vid}, tā kā projekts ir
tiek veikts kā labdarības projekts, tad attiecīgi tiek aprēķināts iegūtais labums, kas tiek vērtēts
pēc iegūtiem nodokļiem, kuri palielinājās gan dēļ lielākiem tēriņiem uz IT tehniku un pakalpojumiem,
gan dēļ lielāka iedzīvotāju ienākuma nodokļa ieņēmumu skaita.
\par
%TODO ielikt attēlu
\par
Autors veica ieinteresēto pušu analīzi, ko var apskatīt \ref{app:Projekta_interesentu_analize} pielikumā.
Katra no itnersentu grupām tika novērtētu viņu attieksme pret projektu, ietekmes pakāpe, cerības un bailes,
izvēlētā stratēģija attiecībā pret šiem cilvēkiem un pasākumi, kuri būtu veicami attiecībā pret šo cilvēku
grupām.
\par 
Kopumā projekts tiek uztverts ļoti pozitīvi un kā ļoti vajadzīgs Latvijas sabiedrībai, līdz ar to kopumā
attieksme pret projektu ir vai nu neitrāla, vai pozitīva. Lielākā daļa uztver šo proejktu ar lielām cerībām,
taču pašiem skolniekiem šis ir lielas izmaiņas, līdz ar to viņu attieksme tiek novērtēta vairāk uz baiļu pusi.
Citi Latvijas IT uzņēmumi uztver šo projektu neduadz bailīgi dēļ tā ka Start(IT) fonds lielākoties tiek
atbalstīts no Accenture Latvija puses un bažas ir saistītas ar slēpto reklāmu un rekrutēšanu jau tā 
ļoti "sausā" darba tirgū.
\par
Lielākā ietekm uz projektu protams ir Start(IT) valdei un projekta uzdevuma devējam. Projekta vadītājs arī
neatpaliek dotajos rādītājos. Pēc tam seko projekta komanda, kura sastāv no diviem papildus cilvēkiem.
Vidēja ietekme ir skolotājiem, jo mācību materiāli tomēr tiek veidoti viņu vajadzībām, taču šie skoltāji
nevarēs pilnībā mainīt gala prdouktu. Ar zemākām ietekmes pakāpēm ir novērtēts VID, citi Latvijas uzņēmumi,
un citi Latvijas iedzīvotāji. Šīs cilvēku grupas kopumā ir ieinteresētas projekta rezultātā, taču nekādas
tiešās ietekmes uz šo projektu viņiem nav.
\par
Izvērtējot ietekmes pakāpi un attieksmi pret doto projektu tika izvēlētas dažādas stratēģijas katrai no cilvēku
grupām. Pārsvarā tā ir diskursīvā stratēģija, un kā pasākums tiek pārsvarā izvēlētas skaidrojošās darbības.
Informēt šīs cilvēku grupas par labumiem, kuras šis projekts piedāvā. Projektā iesaistītām personām, kurām
ir lielāka ietekme uz projektu tiek izvēlēta participatīva stratēģija ar mērķi pēc iespējas vairāk iesaistīt
šīs personas sarunās un lēmumu pieņemšanā. Tādā veidā viņi būs informēti par projekta gaitu un spēs pieņemt
labākus lēmumus. 
\par
Kopumā tika identificētas vairākas cilvēku grupas, kurām ir saskarsne ar šo projektu. Analizējot šo cilvēku
grupu attieksmi, ietekmi tika izvēlēta sadarbības stratēģija un tika izvēlēti konkrēti pasākumi.
\par
Nākošā apakšnodaļa veic risku kvalitatīvo analīzi. Pirms tam identificētie riski tiek padziļināti analizēti
un aprakstīti, vēlāk tiek veikta risku ranžēšana.
