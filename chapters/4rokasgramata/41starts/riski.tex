\subsection{Projekta risku kvalitatīvā analīze}
%3.1.3
Ir ļoti svarīgi laicīgi identificēt riskus un sagatavot stratēģijas to novēršanai, vai, ja tas nav
iespējams, tad, vismaz, sagatavot rīcības to iestāšanās gadījumā. Risku laicīga atpazīšana un 
sekmīga apstrāde var novērst lielus zaudējumus projektam kopumā.
\par
Projektu vadībā risku pārvaldībā ir vairāki soļi - risku identificēšana, kura jau tika veikta alternatīvu
salīdzināšanas laikā; risku kvantitatīva un kvalitātīvā analīze, vēlāk riska iestāšanās gadījumā atbides 
reakciju izveidošana un visbeidzot risku novēršanas stratēģijas novēršana
%TODO add citation to PMBOK, 5th edition, 309 page
%PMBOK 5th Ed 11.3.1 - Perfrom qualitative risk analysis
%Aprakstam kādi ir ieejas dati,
\par
Lai veiktu risku anlīzi autors izveidoja 9 x 9 matricu, kur viena no asīm norāda uz riska iestāšanos
iespējamību, otrā ass norāda riska ietekmi uz projektu. Katram no identificētiem riskiem
tiek iedots identifikators - R<SKAITLIS> - vēlāk šie skaitļi tiek izvietoti atbilstoši izveidotā matricā.
Riska līmenis tiek noteikts sareizinot abus skalu attiecīgos rādītājus.
\par
\begin{figure}[h!]
  \centering
  \includegraphics[width=\linewidth]{./images/riski.png}
  \label{image:riski}
  \caption{Risku ranžēšanas grafiks}
\end{figure}
\par
Kopā tika identificēti 12 riski: (R1) Projekta izmaksas pārsniedz plānotās; 
(R2) Fonda biedru izstāšanās no Start(it), kas samazinātu maksātspēju par projektu;
(R3) Mākoņpakalpojumu sadardzināšanās;
(R4) Tīmekļa vietnes nesavietojamība ar jaunajiem mācību materiāliem;
(R5) Nekvalitatīva eksistējošā tehniskā dokumentācija;
(R6) Mācību platformas pārslodze;
(R7) Ļaundabīgi uzbrukumi mācību platformai;
(R8) Valsts izglītības sistēmas maiņa - atcelta Skola 2030 programmaībām;
(R9) Valsts Izglītības Satura centra atteikšanās no sadrbības;
(R10) Neatbilstība GDPR pras;
(R11) Mācību materiālu izstrādātāju kvalifikācijas trūkums;
(R12) Programmētāju kvalifikācijas trūkums.
No kuriem 2 ir saimnieciskie, 5 ir tehniskie, 2 tiesiskie un 2 personāla riski.
\par
Autors veica šo risku analīzi un izveidoja tabulu, kur var apskatīties \ref{app:Projekta_risku_analize} pielikumā.
Tā sastāv no riska apraksta, cēloņa, sekām, iestāšanās varbūtības vērtējuma, radīto seku ietkmes vērtējuma, kopējā 
riska līmeņa vērtējuma un preventetīviem pasākumiem.
\par
Kopumā ir tikai viens risks - R11 ir iekļuvis augstā risku kategorija un tikai viens - R1 ir iekļuvis vidēji augstā
kategorijā. Ļoti liels risku grupējums ir atrodams ar vidējām sekām, bet zemu iestāšanās varbūtību, kopā tie ir 7 riski.
Viens zems risks - R3 un 2 ar augstām sekām, taču zemu varbūtību - riski R10 un R12. Autora viedotā tabulā var redzēt
piedāvātos veidus kā novērst šos riskus un kādas būtu sekas. 
\par
Risku noteikšanu un pārvērtēšanu vajadzētu veikt katras fāzes kontroles stadijā. Tas ļautu laicīgi identificēt vai
kāds no riskiem ir iestājies, vai arī vai kāds no tiem vairs nav aktuāls. Protams var tikt atklāti arī jauni riski,
kurus vajadzētu pievienot dotajam ranžējumam.
\par
Nākošā nodaļā tiek apskatīta organizatoriskā struktūra dotajam projektam. Šī dokumenta daļa palīdzēs ātri un vienkārši
saprast kurš cilvēks atbild par kuru daļu no projekta un ar ko un kā vajadzētus sazināties.


