\subsection{Projekta institūciju lomu apraksts}
%3.1.5
Lai veiksmīgi īstenotu jebkuru projektu ir ļoti svarīgi apzināties visu iesaistīto pušu
atbildības un pienākumus. Šādam nolūkam tiek izveidots lomu apraksts, kurš uzskaita 
pienākumus, atbildības un tiesības. Tiek izveidots arī apraksts par to kura no lomām apstiprina
viena vai otra darba paveikšanu. Kopumā šī rokasgrāmatas daļa palīdz izvairities no domstarpībām
un nevajadzīgiem strīdiem par to kurš par ko ir atbildīgs un ko var veikt.
\par
Tā kā iepriekšējā apakšnodaļā tika izvēlēta stiprās matricas organizatoriskā struktūra, tad
no tā var secināt, ka projekta vadītājam būs pietiekoši liela brīvība un iespējas realizēt doto
projektu. Taču vajadzētu aprakstīt tās gan projekta vadītājam, gan pārējiem komandas biedriem.
\par
Projekta komandas lomas tika aprakstītas \ref{app:Projekta_instituciju_apraksts} pielikumā. Izmantojot
to autors izveidoja tabulu ar atbildībām dažādās fāzēs
\begin{figure}[h]
  \centering
  \includegraphics[width=\linewidth]{./images/atbildibas.png}
  % \label{image:atbildiba}
  \caption{Atbildibas matrica}
\end{figure}
No tabulas var secināt, ka sākumā projekta uzdevuma devējs iesaistās izpildē, taču tas notiek tiekai
projekta start fāzē, pēc tam viņš tikai veic sadarbību un kontroli. Lielākā daļa kontroles arī pāriet 
pie projetka vadītāja. Nelielu kontroli ir jāveic arī projekta vadītāja asistentiem, tas notiek
realizācijas un ieviešanas fāzēs, kad abiem asistentiem ir savas komandas ar izpildītājiem, kuriem 
jānodod sākotnējās izpētes un analīzes rezultātus, lai tie veiksmīgi varētu izpildīt savus darbu.
\par
Tā kā uz projekta vadītāju un viņa asistentiem tiek uzliktas pietiekoši lielas prasības, tad attiecīgi
ir jāmeklē un jāpārbauda zināšanu līmenis katram no šiem trim dalībniekiem. No projekta vadītāja tiek
sagaidīta pabeigta izglītība projektu vadīšanā, vai līdzvērtīgu projektu vadīšanas pieredze vismaz 3
gadu garumā. Jābūt zināšanām gan par IT jomu, gan kā notiek apmācības skolās, gan kā vadīt projektus.
Jābūt spējīgam ātri orientēties notikumos, kuri attīstās ar lielu ātrumu un ne gluži kā bija sākotnēji
paredzēts. No asistentiem tiek sagaidīta vairāku gadu pieredze attiecīgā sfērā, būtu jābūt arī gataviem
arī pašiem strādāt un veikt darbus, nevis tikai veikt analītiskos darbus.
\par
Apkopojot augstāk rakstīto - tika sagatavots dažādu lomu apraksts un paskaidrotas to atbildības un veicamie
darbi. Kā arī tika izvirzītas aptuvenās prasības potenciāliem kandidātiem uz šīm pozīcijām.
\par
Tagad, kad ir aprakstītas lomas un viņu savastarpējās attiecības paliek vien apskatīt savastarpējās
komunikācijas formas. Tas arī tiek darīts nākošā apakšnodaļā.
