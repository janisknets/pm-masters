\phantomsection
\subsection{Projekta uzdevums un starta organizācija}
%3.1.1
Projekta starta fāze ir ļoti svarīgs projekta brīdis. Šīs fāzes ietvaros vēl var labot 
projekta plānojumu ar minimālu ietekmi uz budžetu un veicamo darbu saturu. Ja izmaiņas būs
jāievieš vēlāk, tad tas izmaksās ar vien dārģak.
\par
"Šajā fāzē projekta vadītājs analizē projekta uzdevumu, detalizēti analizē projektā
sastopamos riskus, nosaka risku pakāpi un izstrādā risku novēršanas plānu. Starta fāzē analizē
projekta interesentus un nosaka to ietekmes samazināšanas pasākumus, izveidot projekta
organizatorisko struktūru, vienojas ar projekta uzdevuma devēju par projekta fāžu modeli, kā arī
nosaka projekta organizatorisko infrastruktūr."\cite{zane_12}.
%TODO adjust this text
\par
Projekta uzdevums ir dokuments, ko paraksta uzdevuma devējs un projekta uzdevuma
ņēmējs – projekta vadītājs. Ar šo dokumentu atbildība par projektu tiek nodota projekta vadītāja
rokās, kā arī projekta uzdevuma devējs apliecina, ka apņemas nodrošināt nepieciešamos resursus %\cite{zane_11}
\par
Autors izveidoja projekta uzdevumu, kuru var apskatīt \ref{app:Projekta_uzdevums} pielikumā. Tajā ir
norādīti vairāki punkti, tādi kā iegustamo rezultātu saraksts, sākuma un beigu datumi, kāds ir nepieciešamais 
projekta budžets. Tiek arī īsuma paskaidrots projekta iemesls, iedots īss apraksts par projektu kopumā un tā
pamatojums. 
\par
Projekta izstrādes laikā tiek iegūti vairāki rezultāti - uzlabota eksistējoša tīmekļa vietne, gan vizuāli,
gan tehniski; tiek sagatavoti mācību materiāli skolām, ar visu nepieciešamo papildus dokumentāciju 
skolotājiem; tiek filmēti vairāki mācību video materiāli ar kopējo garumu 1440 min. Projekta laikā ir arī
ieplānotas pilotapmācības skolotājiem, kuru laikā šaurs skolotāju loks tiktu apmācīts kā darboties ar
jaunizveidotajiem materiāliem un tīmekļa vietni. Šīs pilotapmācības ir svarīgs brīdis, kad varēs pārliecināties
par paveikto darbu efektivitāti pirmo reizi.
\par
Projekta starts tiek ieplānots uz 17 Jūniju 2019. gadā. Kopumā projekts ilgs nedaudz virs 9 mēnešiem.
Tas tiks pabeigts 2020 gada 17. martā. Taču šie datumi nav galīgi un var būt mainīti projekta gaitā
atkarībā no uzdevumu izpildes ātruma. Kopējāis projekta budžets ir 733 092,98 €, šis ir provizoriskais 
budžets kurš veidojās no alternatīvu salīdzināšanas darbiem 2. nodaļā. 
\par
Projekts sastāv no 8. fāzēm - starts, plānošana, izpēte, pamatkoncepcija, detaļkoncepcija, 
realizācija, ieviešana, noslēgums. 
\par
Projekta startu ir jāapstiprina Start(IT) fonda valdes pilnvarotajam projekta uzdevuma devējam,
kā tas notiks un projekta uzdevums tiks parakstīts, projekts varēs sākties.
\par
Nākošā apakšnodaļā tiek apskatīti projekta interesenti. Pirms sākt izstrādāt produktu ir nepieciešams
noteikt iesaistītos cilvēkus projektā, viņu motivāciju un ieinteresētību attiecībā pret projektu.
Šo interesentu analīze palīdzēs projekta izstrādes laikā veikt labāku komunikāciju ar iesaistītām pusēm.
Šāda analīze arī ļauj ātrāk noteikt negatīvi noskaņotās ieinteresētās puses un veikt preventetīvu darbību.