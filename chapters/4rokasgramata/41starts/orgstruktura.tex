\subsection{Projekta organizatoriskās struktūras izveide}
%3.1.4
Lai varētu nosakidrot kādā veidā notiks komunikācija projekta un kā tiks 
piesaistīti jauni projekta dalībnieki vajadzētu noteikt organizatorisko struktūru.
\par
PMBOK piektais izevums runā par 5 dažādām organizācijas formām, kā arī kā dažādas struktūras var būt 
izmantotas dažādos uzņēmuma līmeņos.
Šīs struktūras ir - \textit{Funkcionālā} - vairāk ir domāta mazem projektiem, cilvēki nav pilnībā
piesaistīti projektam, tiem nav lielas pārvaldības pāri budžetu un citiem faktoriem; - \textit{Matricas} - 
šī organizatoriskā struktūra iedalās trīs apakšadaļās - vājās, balansētās un stiprās matricas org. struktūra.
Jo specīgāka matricas struktūra tiek izmantota, jo vairāk lēmejspējas ir projekta vadībai un jo mazāk projekta
komanda ir padota kādai citai struktūrai. Šādu organizatorisko struktūru ir vērts izmantot ja cilvēki tiks
piesaistīti projektam uz ilgstošāku laiku. Pēdējā organizatoriskā struktūra ir \textit{projektētā} organizatoriskā
struktūra. Tajā cilvēki ir pilnībā piesaistīti pie projekta un tiem vairs nav neviena cita priekšnieka. Visa 
projekta komanda eksistē tikai šīs struktūras ietvaros. Šo variantu iesaka izmantot ja viss darba spēks pilnu
laika slodzi strādās pie šī projekta un viņiem ir nepieciešama manverēšanas brīvība. \cite{PMBOK}
%TODO PMBOK 2.1.3 21 - 26
\par
Tā kā šis projekts sastāvēs no mazas komandas, taču tai ir nepieciešams būt pietiekoši brīvai savā rīcībā, tad
vairāk būt piemērotas stiprās matricas vai projektētās organizatoriskas struktūras variants. Ņemot vērā, ka daļa
no izpildītājiem visticamāk nāks no eksistējošiem Start(IT) fonda dalībnieku uzņēmumiem, tad autors izvēlējās
stiprās matricas organizatorisko struktūru.
\par
\begin{figure}[h!]
  \centering
  \includegraphics[width=\linewidth]{./images/orgstruktura.png}
  \label{image:riski}
  \caption{Stiprās matricas organizatoriskā struktūra}
\end{figure}
\par
Start(IT) valde ieceļ projekta uzdevuma devēju dotajam projekta. Savukārt projekta uzdevuma devējs strādā
kopā ar viņa izvēlētu projekta komandu, primāri sadarbojoties ar projekta vadītāju. Projekta vadītājam ir
divi asistenti - tehniskais un mācību materiālu. Katrs no šiem asistentiem joprojām paliek sava uzņēmuma
organizatoriskā struktūrā un piedalās savu uzņēmumu dzīvē. Taču viņi arī strādā uz pilnu slodzi dotā
projekta ietvaros. Uz daļu no projekta vēl tiktu piesaistīti papildus darbaspēki, lai veiktu nepieciešamos
darbus ar tīmekļa vietni un izstrādāt nepieciešamos gala produktus. Tajā laikā abi asistenti pilnīs tiešo
priekšnieku pienākumus un vadīs šīs abas komandas.
\par
Lai noskaidrotu savastarpējās saistības starp projekta komandas biedriem, kā arī viņu atbildības, uzdevumus
un tiesības, nākošā apakšnodaļā tiek apskatīts projektu institūciju lomu apraksts.  

