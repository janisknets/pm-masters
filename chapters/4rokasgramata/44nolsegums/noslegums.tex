\section{Projekta noslēgums}
%3.4
Izpildot visus projekta paredzētos uzdevumus projekts nonāk dabīgā noslēguma fāzē,
tajā tiek apkopoti visi iegūtie rezultāti, tiek sagatavota nepieciešamā dokumentācija,
atbildība par izveidoto produktu tiek nodota projekta uzdevuma devējam.
%cite zane 11, 78-79
\par
Informācījas saglabāšana nākotnei ir viens no būtiskiem soļiem, tas viecinās nākotnes
projektu ātrumu un kvalitāti, līdz ar to pret šo solit attiecās atbildīgi.
\par
Noslēguma fāzē autors ir paredzējis 7 norises - pieņemšanas un nodošanas akta parakstīšana;
faktisko izmaksu apkopošana; projekta pieredzes apkopošana; gala atskaites sastādīšana;
projekta dokumentācijas sakārtošana un nodošana; projekta noslēguma sanāksmes organizēšana;
projekta noslēguma sanāksmes novadīšana.
\par
Ļoti svarīgi noslēguma sanāksmes laikā dokumentēt kādi bija veiksmes stāsti un kādas problēmas
tika atrisinātas, vai bija palikušas neadresētas. Būtu jāpārrunā iegūto rezultātu nākotnes
attīsības iespējas. Arī ir vērts pakavēties pie sekām, kuras radās dēļ dotā projekta.
Kā arī citi temati, kuri būtu svarīgi kādam no projekta komandas vai projekta uzdevuma devējam.
\par
Par šo norisi atbildīgais ir projekta vadītājs, taču viņam joprojām palīdz viņa asistenti, it sevišķi
sakarā ar dokumentāciju. Projekta noslēguma dokumentācija kalpos kā projekta vēstures dokuments, kā 
arī kā atsauces dokuments nākotnē.
\par
Līdz ar šīs nodaļas beigām tiek sasniegts rokasgrāmatas beigas. Nākošā nodaļa tiek veikti secinājumu un 
priekšlikumi no autora puses.
