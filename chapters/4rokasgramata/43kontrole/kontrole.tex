\section{Projekta kontrole}
%3.3
Projekta kontroles tiek veidota izmantojot iepriekš izvirzītos projekta mērķus, 
izveidotiem plāniem, kā arī uz to brīdi noslēgtām vienošanām un līgumiem. Kontroles
laikā tiek salīdzināts reālais progress projektā pret iepriekš plānoto. Pēc 
tiešās kontroles pabeigšanas, var pieņemt lēmumums par dotā projekta plāna labojumiem.
Kā arī tiek viedotas atskaites par dotā brīža finanšu tēriņiem salīdzinot ar plānotiem.
%cite zane 13 193lpp. 
\par
Autors izmanto Keznera piedāvāto kontrolēšanas procesu trīs soļos - Izmērīšana, kuras laikā tiek
veiktas pārbaudes par paveikto skatoties uz dažādie rādītājiem; Izvērtēšana, kuras laikā tiek
izpētiti cēloņi konkrētām nesakritībām; Pēdējais solis ir koriģēšana, pasākumu īstenošana, 
kura palīdz novērst radušās nesakritības.
%\cite zane 13 193 lpp
\par
Kopumā tika izveidoti 8 robežstabi, viens katras fāzes beigās. Robežštabu datumi:
\begin{itemize}
  \item Starta fāzes kontrole -	6/28/2019
  \item Plānošanas fāzes kontrole -	7/22/2019
  \item Izpētes fāzes kontrole -	8/13/2019
  \item Pamatkoncepcijas fāzes kontrole -	9/2/2019
  \item Detaļkoncepcijas fāzes kontrole -	10/3/2019
  \item Realizācijas fāzes kontrole -	1/13/2020
  \item Ieviešanas fāzes kontrole -	2/12/2020
  \item Noslēguma fāzes kontrole -	3/6/2020
\end{itemize}
\par
Projekta kontrolingu ietvaros sasniegto rezultātu apskats aplūkojams darba 34. pielikumā.
Katra kontrolinga ietvaros norisinās sēde, kurā projekta uzdevuma devējs tiek iepazīstināts ar
paveiktajiem darbiem, tiek informēts par sasniegtajiem rezultātiem, iespējamām problēmām vai
pozitīvajām tendencēm, kas ietekmējušas projektu. Projekta kontrolings balstās uz ziņojumu, kuru
ir sagatavojis projekta vadītājs.
Projekta kontrole piedalās visa projekta komanda kopā ar projekta uzdevuma devēju. Tiek pārbaudīti
sasniedzamie rezultāti un resursu tēriņš, vēlāk pēc šiem datiem tiks sastādītas atskaites.
Projekta fāžu kontroles robežštabu datumus un aprakstu var redzēt \ref{app:Projekta_kontrole} pielikumā.
\par
Atskaites tiek veidotas vadoties no konkrētā robežstaba izpildes prasībām. Paveiktais darbs tiek novērtēts
ar to cik procentus no tā komanda izpildīja, šis vertējums arī tiek ierakstīts projekta atskaites formā.
Autors iesaka izmantot fiksētu procentu skaitu kuru drīkst ierakstīt formā, tādā veidā samazinot strīdus;
Iteicams izmantot tikai 25/50/75 un 100\% vērtības. Formas paraugu var atrast 
\ref{app:Projekta_atskaitas_forma} pielikumā.
\par
Nākošā skaidro darbības kuras jāveic projekta noslēguma fāzē.