\subsection{Projekta resursu un izmaksu analīze}
%3.2.3
Resursu plānošana ir viena no būtiskajām sastāvdaļām proejkta resursu vadīšanā. Atpazīstot un
nozīmējot resursus ar nepieciešamo pieejamību var ietaupīt projekta finanšu resursus. Par resursiem
tiek uzskatīti cilvēki, finanses, aprīkojums, materiāli, iekārtas, zināšanas, dokumenti u.c.
Faktiski jebkas, kas varētu būt izmantots projekta ietvaros, lai iegūtu vajazīgot rezultātu.
Projekta vadītāja viens no galveniem uzdevumiem ir šo resursu pareiza pārvaldīšana un nozīmēšana,
viņam arī vajadzētu sekot ka visi nepieciešami resursi ir pieejami, kad tie ir nepieciešami
%cite zane_10, 58 lpp
\par
Dotā projekta papildus pie jau eksistējošās projekta komandas, kura sastāv no projekta
vadītāja un viņa diviem asistentiem, ir piesaistīti vēl vairāki resrusi. Tika lemts piesaistīt
trīs mācību materiālu veidošanas specialistus, vienu filmēšanas grupu, kā arī 3 programmētājus
tīmekļa vietnes izstrādei. Tiek arī paredzēts grāmatveža darbs, lai nodrošinātu proejkta korektu 
grāmatvedību. Mācību materiālu izstrādes specialistu uzraudzīs projekta vadītāja mācību materiālu
asistents, kamēr programmētājus pieskatīs projekta vadītāja tehniskais asistents. Pats projekta
vadītājs nodrošina abu šo apakškomandu saskaņotu darbību un beigu rezultāts varētu veiksmīgi tikt
savienots. Projektam nav paredzēti nekādi materiālie resursi.
\par
\begin{figure}[h!]
  \centering
  \includegraphics[width=0.5\linewidth]{./images/alok_pv.png}
  \label{image:alok_pv}
  \caption{Projekta vaditāja noslodze}
\end{figure}
Projektā slodzes ir sadalītas diezgan vienlīdzīgi starp projekta vadītāju un viņa asistentiem.
Tajā laikā, kad tiek izveidoti mācību materiāli, projekta vadītājam sanāk pat palikt bez darba,
līdz ar to viņš varētu būt pisaistīts kādam citam īsam projektam vai paņemt atvaļinājumu.
Zemāk var redzēt vairākus grafikus ar iesaistīto resursu noslogojumu
\par
\begin{figure}[h!]
  \centering
  \includegraphics[width=0.5\linewidth]{./images/alok_pvt.png}
  \label{image:alok_pvt}
  \caption{Projekta vaditāja tehniskā asistenta noslodze}
\end{figure}
Attēlā ar zilo krāsu ir iekrāsota normālā slodze, kuru veic konkrētais darbinieks. Ar sarkano
krāsu būtu iezīmēti tie stabiņi, kuros ir novērojama pārslodze. 
\begin{figure}[h!]
  \centering
  \includegraphics[width=0.5\linewidth]{./images/alok_pvm.png}
  \label{image:alok_pvm}
  \caption{Projekta vadītāja mācību materiālu assistenta noslodze}
\end{figure}
Projekta vadītāja asistenti ir vienlīdzīgi noslogojuma ziņa un abi ir noslogoti līdz maksimumam.
Viņiem ir ieplānota pilnda darbība no projekta sākuma līdz beigām.
\par
\begin{figure}[h!]
  \centering
  \includegraphics[width=0.5\linewidth]{./images/alok_prog_be.png}
  \label{image:alok_prog_be}
  \caption{Pirmā programmētāja noslodze}
\end{figure}
\begin{figure}[h!]
  \centering
  \includegraphics[width=0.5\linewidth]{./images/alok_prog_fe.png}
  \label{image:alok_prog_fe}
  \caption{Otrā programmētāja noslodze}
\end{figure}
Tālāk seko 3 programmētāji, divi no kuriem ir tiešie programmētāji un viņu noslodzes ir vienādas.
Tiek plānots modernizēt eksistējoš tīmekļa vietni divos mēnešos.
\par
\begin{figure}[h!]
  \centering
  \includegraphics[width=0.5\linewidth]{./images/alok_prog_test.png}
  \label{image:alok_prog_test}
  \caption{Testētāja noslodze}
\end{figure}
Tā kā testētājam sākumā nebūtu ko testēt, tad viņam piesaistīšanās projektam notiek diezgan vēlu.
un uz diezgan īsu brīdi. 
\par
Lai veikt projekta izmaksas plānojumu, tiek mērītas katras darba paketes, norises un visa projekta
kopējās izmaksas. Tad sasniegtās izmaksas tiek salīdzinātas ar projekta sākumā prognozētām
izmaksām. Naudas finanšu sadalīšana pa fāzēm nodrošina to, ka vajadzīgos naudas resursus var
sagatavot laicīgi pirms tās fāzes kad tie ir nepieciešami. Dotajam projektam it sevišķi svarīgi
ir sagatavot resursus uz realizācijas fāzi, jo tās laikā ir jāveic maksājumi par video filmēšanu,
kas sastāda lielāko daļu no visām izmaksām.
\par
Projekta ietavros autors veica izmaksu plānošanu, tā ietver sevī aprēķinus, lai iegūtu visus
nepieciešamos finanšu resursus projekta īstenošanai, kā arī paredzot 10\% -  47 656.90 papildus no paredzētā 
budžeta neparedzētiem gadījumiem. 1537.00 € tiek ieplānoti projekta biroja uzturēšanai, tai skaitā
arī grāmatvedībai. Mācību materiālu izveidošanas prognozētās izmaksās sastāda 13 760 €, video materiālu
filmēšana tiek prognozēta ar kopējām izmaksām ap 432 000.00 €. Pārējie tērīņi sastāda algas 
izmaksāšanu projekta komandai. Datus šiem cipariem var redzēt \ref{app:Projekta_resursu_izmaksu_tabula}
pielikumā. Kopējās prognozētās izmaksas sastāda 476,569.01 €, savukārt 2. nodoļā prognozētie izdevumi
sastādīja 573 549.89 €. Tik liela starpība var būt izskaidrota ar ietaupītiem pieciem mēnešiem
darba, kas veidotu samērā lielas algas izmaksas visiem darbiniekiem. Kopumā veicot detalizēto
plānošanu tika ietaupīti 96 980.88 €. Vēlreiz norādot uz to cik daudz ietaupījumu un pozitīvu faktoru 
var iegūt veicot detalizētu plānošanu.
\par
Autors arī veica salīdzinājumu starp agriem un vēliem laikem, kurus var apskatīties
\ref{app:Projekta_izmaksu_salidzinajums_agrais} un \ref{app:Projekta_izmaksu_salidzinajums_velais}
pielikumos attiecīgi. Zemāk ir redzams salīdzinājums starp agriem un vēliem laikiem grafiskā veidā
\par
\begin{figure}[h!]
  \centering
  \includegraphics[width=0.5\linewidth]{./images/banans.png}
  % \label{image:banans} 3.10
  \caption{Salīdzinājums starp agriem un vēliem laikiem.}
\end{figure}
\par
Kā redzams 3.10 attēla, lielāka daļa izmaksu notiek tajā pašā laikā, taču skatoties
pēc vēliem laikiem (ornžā līnija), tad video materiālu izmaksas tiek pavirzītas tālāk par vienu mēnesi.
Šāds pavērsiens dažos gadījumos var būt diezgan būtisks, atkarībā no uzņēmuma likviditātes un 
pieejamiem finanšu resursiem.
\par
Nākošā apakšnodālā tiks apskatīta projekta fāžu kontrole. To ir būtiski veikt pēc katras fāzes,
lai varētu sekot projekta izpildes progresam un laicīgi reaģēt uz jebkādiem kavējumiem
