\phantomsection
\subsection{Projekta struktūrplāna analīze}
%3.2.1
Viens no galvenajeim rīkiem projekta vadītāja jostā ir "skaldi un valdi" princips. Tajā tiek runāts par to,
ka darbus vajadzētu dalīt uz pēc iespējas mazākām vienībām, jo tādā veidā daudz vienkāršāk saprast cik daudz
darba vēl ir palicis ko veikt, kas ir izdarīts, kādā stāvokli ir izdarītais.
\par
Dotajā gadījumā projekts tiek sadalīts pa fāzēm, tālāk pa darbu paketēm un individuālām norisēm. Tādā veidā
ir vienkāršāk izsekot projekta progresam. PMBOK standartā tiek pieminēts WBS, jeb \textit{work brakedown structure},
kur tiek runāts par kopējo projekta darbu apjoma sadali uz paketēm un tālāk uz konkrēti veicamiem darbiem.
%TODO PMBOK 126lpp
\par
Projekta struktūrplāns tiek veidots vai nu grafiski, vai arī tabulāri. Pirmajā gadījumā, tas tiek attēlots kā hierarhiska koka 
struktūra, otrajā gadījumā visi elemti tiek izvietoti tabulā. Dotā projekta struktūrplāns tika izstrādāts sākumā projektu
sadalot 8 fāzēs, tālāk sadalot katru fāzi uz darba paketēm, visbeidzot tās sadalot uz indivuālām norisēm. Tātad tika izmantos
no augšas uz leju princips.
%TODO cite zane_11
\par
Izveidoto struktūrplānu var aplūkot \ref{app:Projekta_strukturplans} pielikumā. Tas sastāv no vairākām fāzēm:
projekta vadīšana, izpēte, pamatkoncepcija, detaļkoncepcija, realizācija un ieviešana. Realizācijas fāze ir izdalīta smalkāk,
jo tajā tiks veikts lielākais darbu apjoms, līdz ar to vajadzētu uzreiz izdalīt darbus nedaudz smalkāk. Struktūrplāna tumši
zilā krāsā ir iekrāsotas fāzes, gaiši zilā darba paketes, baltā ir iekrāsotas apakšdarba paketes, jo attiecīgajām norisēm
ir lielāks darbu apjoms. Kopumā tika izstrādātas 54 darba paketes kuras var apskatīt \ref{app:Projekta_darba_paketes} pielikumā kopā
ar to norisēm.
\par
Katrai darba paketei ir noteikta viena atbildīgā persona, tiek pieminētas dotās darba paketes norises, sagaidāmie rezultāti
ir izveidota vieta arī projekta vadītāja un darba paketes vadītāja parakstiem, kad tās būs pabeigtas. Ir arī atzīmets
nepieciešamais darba dienu skaits. 
\par
Projekta struktūrplāns kopā ar darba pakešu aprakstu ļauj veikt detalizētu projekta gaitas plānošanu, jo ir skaidri saprotams
darba apjoms un cilvēku skaits, kurš ir nepieciešams, kā arī tiek parādīta izpildes aptuvenā secība.
\par
Nākamā nodaļā tas arī tiek īstenots - tiek analizēti projekta gaitas termiņi, tiek arī atrasts kritiskais ceļs projektam.
