\subsection{Projekta gaitas un termiņu analīze}
%3.2.2
Atsauce uz pmbok vai citu par to kāpēc projekta giatas un termiņu analīze ir nepieciešama
\par
Apkopojums par katras fāzes saturu, cik daudz norises ir katrā no fāzēm, cik kopā. 
Atzīmēt svarīgos robežstabus, kontroles punktus, dokumentu apkopošanu utt,
\par
Paragrāfs par to ka tika izveidots darba plāns
\par
Paskaidrojums par MS Project un pielietoto metodi, paskadirojums par to kā un kas tiek attēlots
pieikumā. 
\par
Kritiskā ceļa definīcija un apraksts no tā ko var redzēt pielikumā.
\par
Kā plāns tika veidots attiecībā pret darba resursiem
\par
Apraskts par termiņiem, laiku, vēlreiz tiek pieminēti svarīgie robežstabi
\par
Kopējā projekta garuma ilgums un salīdzināšana ar oriģinālo plānojumu otrajā nodaļā
\par
ievads nākošā apakšnodaļā - resursu un izmakšu analīze

