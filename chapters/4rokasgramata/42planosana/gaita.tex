\subsection{Projekta gaitas un termiņu analīze}
%3.2.2
Projekta gaitas plāns parāda kādā veidā darba paketes un to norises ir savstarpēji saistītas.
Tas arī parāda norišu garumu laikā, palīdz veikt norišu vērtējumu, kā arī parāda resursu piesaisti
pie konkrētās norises. %cite zane_10, lpp61
\par
Autors izveidoja MS project lietotnē projekta plānojumu, viens no tā skatiem ir apskatāms
\ref{app:Projekta_terminu_plans} pielikumā. Tajā var redzēt pakešu skaitu un to agros un vēlos laikus,
kā arī secību un dienu garumu. Projekta vadīšanas fāzē ir paredzēta 31 norise; Izpētes fāzē to ir 
19; 21 norise tika izveidota Pamatkoncepcijas fāzē un 19 detaļkoncepcijas fāzē; Visvairāk norišu ir 
realizācijas fāzē - 27; Ieviešanas fāže norišu skaits ir 17. Gandrīz pirms kataras fāzes tiek ieplānoti
projekta pārplānošanas darbi, savukārt pēc katras fāzes tiek veikta kontrole par paveikto.
\par
Izmantojot iegūto darba norišu plānu tika izveidots termiņu plāns, pēc kur var noteikt kurā brīdī
būtu jāsāk kura norise, lai projektu varētu sekmīgi izpildīt. Ši plānošana jāveic ļoti rūpīgi, jo
ja tā ir paveikta veiksmīgi, tad var ietapuīt uz darbinieku algošanu tikai nepieciešamajiem darbiem,
un tie varēs pilnvērtīgi strādāt pilnu slodzi. Izmantojot šo informāciju tika izveidots projekta
gaitas plāns, kurš ir apskatāms \ref{app:Projekta_gaitas_plans} pielikumā. Tajā tiek izmantots 
tīkla plāna modelis, kur katra norise ir taisnstūra bloks, sarkanā krāsā iekrāsotās norises atrodas
uz kritiskā ceļa. Bultas norāda savstarpējo sekošanu. Bultas ir virzīta no vecākas norises uz jaunāko.
\par
Kritiskais ceļš ir to darba norišu virkne, kurām agrie un vēlie laiki sakrīt un to secība nevar būt
mainīta. Ja jebkura no šim norisēm aizkavēsies, tiks aizkavēts viss projekts par tādu pašu laika posmu.
Džeims Luiss to skaidro šādi: "projеktа tīklа gаrākаis posms projеktu gаitаs
tīklа plānā, un tаs nosаkа iеspējаmo аgrāko brīdi, kаd dаrbs vаr tikt pаbеigts. Visаs norisеs
kritiskаjā cеļā ir jāvеic sаskаņā аr grаfiku, prеtējā gаdījumā nobeiguma brīdis sāks attālināties –
vienu dienu par katru norises mezgla nokavēto dienu" %\cite zane_4.
\par
Tā kā Projekta komanda sastāv no 3 cilvēkiem, tad daļa no darbiem var būt izpildīta ātrāk un citi darbi
var būt veikti paralēli. Tādā veidā tiek samazināts kopējais dienu skaits salīdzinoši ar otrajā nodaļā
paredzēto. Realizācijas fāzē tiek algoti arī papildus aŗējie pakalpojumu sniedzēji, kuri piegādā 
nepieciešamos produktus, kas atkal ļauj paplašināt paralēļo darbu skaitu.
Tādā veidā tiek iegūts pārskatāms plāns norišu izpildei ar visu nepieciešamo informāciju.
\par
Dotā projekta sākuma datums tika izraudzīts 17 Jūnijs 2019. gadā, ar MS Project programmas palīdzību
tika aprēķināts kopējais projekta garums, kurš sastāda 190 dienas un tiek apredzēts, ka projekts
beigsies 6. Martā, 2019 gadā. Šie dati ir pamatoti ar \ref{app:Projekta_terminu_plans} pielikumu.
Kā arī kopā ar to tika izveidots projekta līnijdaigrammas termiņu plāns \ref{app:Projekta_linijdiagrammas}.
pielikumā. Visas norises ir ieplānotas ar darba dienām 8 stundu garumā.
\par
Sākotnēja analīzē tika paredzēts ka šis projekts kopā aizņems 309 dienas, attiecīgi veicot precīzu
plaņošanu atbilstoši resursu skaitam un pieejamībai, kā arī optimizējot slodzes sanāca ietaupīt
119 dienas. Šis fakts norāda uz plānošanas fāzes svarīgumu un potenciāliem ieguvumiem.
\par
Nākošā nodaļā tiek apskatīta resursu un izmakšu analīze. Šis process ļauj īstenot finanšu resursu
ietaupījumu un veic detalizetāku naudas plānošanu.

