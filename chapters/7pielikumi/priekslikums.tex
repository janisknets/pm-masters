\section{Projekta priekšlikums}
\begin{table}[h!]
  \begin{tabular}{|p{0.9\textwidth}|}
    \hline
      \textbf{Projekta nosaukums}:\\
      Start(it) tīmekļa vietnes pielāgošana Skola 2030 projektam \\
    \hline
      \textbf{Konkrētās problēmas īss apraksts}:\\
      Latvijas iedzīvotājiem kopumā trūkst mācību materiālu latviešu valodā par dažādām
      datorprasmēm. Maksas kursi pārsvarā pasniedz tikai programmēšanu, bet ne papildus
      datorzinātnes un citas prasmes, kuras ir saistītas ar darbu ar datoru. Šo zināšanu
      trūkums tiešā veidā ietekmē Latvijas IKP un kopējo darba ražīgumu. \\
    \hline
    \textbf{Projekta vispārējais mērķis}: \\
      Viecināt padziļināto datorprasmju attīstību Latvijā \\
    \hline
      \textbf{Projekta konkrētais mērķis}: \\
      Paplašināt padziļināto datorprasmju kursu spektra piedāvājumu Latvijas tirgū \\
    \hline
    \textbf{Projekta izvēlētā alternatīva (veicamie uzdevumi)}:\\
      Start(it) tīmekļa vietnes uzlabošana, ar mērķi tās saturu pielāgot Skola2030 projektam,
      tā lai Latvijas skolās varētu sākt pasniegt datorikas stundas.
      \begin{itemize}
        \item Uzlabot eksistējošās tīmekļa vietnes tehnisko risinājumu
        \item Uzlabot eksistējošās tīmekļa vietnes vizuālo izskatu
        \item Izveidot katrai no 12 klasēm 8 apmācošus video 15 minūšu garumā
        \item Izveidot datorikas mācību materiālus 12 klasēm
        \item Izveidot skolotāju pavadošos mācību materiālus
        \item Novadīt pilotapmācības skolotājiem
        \item Nodrošināt tīmekļa vietnes uzturēšanu gan tehniski, gan satura ziņā
      \end{itemize}\\
    \hline
      \textbf{Iegūstamā produkta īss apraksts}:\\
      Tiek izveidota jauna mācību platforma, kuru izmantos Latvijas skolas no 1. līdz 12. klasei
      pasniedzot jaunu mācību priekšmetu - datoriku. Tiek sagatavoti mācību materiāli visām klasēm kopā
      ar video materiāliem, kā arī skolotājus pavadošā informācija. Projekts tiek izstrādāts uz jau eksistējošās
      tīmekļa vietnes bāzes. Projekta uzturēšanai tiek algoti divi darbinieki.\\
    \hline
      \textbf{Projekta pamatojums}: \\
    \hline
      \textbf{Projekta ilgums (mēnešos)}: \\
        Sākums - 2019. gada 14. jūnijā \\
        Beigas - 2020 gada 2 februārī. \\
        Kopējāis garums - 9,04 mēneši \\
    \hline
      \textbf{Projekta budžets}: 733 092.98 €\\
    \hline
  \end{tabular}
\end{table}
\label{app:Projekta_priekslikums}


\clearpage