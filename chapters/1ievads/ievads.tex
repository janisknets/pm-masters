\chapter*{Ievads} % * asterisk does not number the chapter
\addcontentsline{toc}{chapter}{Ievads} %adds unnumbered section to table of contents
Dotā maģistra darba aktualitāti nosaka ta, ka Latvijā jau vairākus gadus ir izjūtams darbinieku 
trūkums IT nozarē. Otrs ietkmējošais faktors ir tas, ka digitālās tehnoloģijas ar vien vairāk
ienāk mūsu ikdienas dzīvē, tās ir jāizmanto ar vien plašākā darba vietu skaitā; Taču Latvijā
digitālo prasmju līmenis ir salīdzinoši zems. 
\par
Dotajā darbā autors cenšas risināt šīs problēmas ar Start(it) fonda pieejamiem resursiem. Fonda mērķis
ir popularizēt datoriku Latvijas skolās. Šī fonda eksistējošā vietne tiks uzlabota tā, lai varētu 
nodrošināt apmācības plašākam cilvēku lokam.
\par
Projekta \textbf{pētījuma objekts} ir digitālās prasmes Latvijā.
\textbf{Maģistra darba mērķis} ir padziļināto digitālo prasmju apmācību paplašināšanas iespējas Latvijā.

\par
\textbf{Mērķa sasniegšanai veicamie uzdevumi} ietver:
\begin{enumerate}
    \item nozares raksturojuma veikšanu, noskaidrojot aktuālās tendences;
    \item noskaidrot nozares ekspertu viedokli par padziļinātām datorprasmēm, to pieejamību un nozīmīgumu;
    \item veikt problēmas noteikšanu un analīzi;
    \item izstrādāt projekta priekšlikumu;
    \item izstrādāt projekta īstenošanas rokasgrāmatu;
    \item izstrādāt secinājumus un priekšlikumus.
\end{enumerate}
\par
Maģistra darbu veido trīs daļas, kā arī secinājumi, priekšlikumi un pielikumi.
Darba \textbf{pirmajā nodaļā} tiek aplūkots Eiropas Savienības \acrshort{desi} indekss, kurš norāda
cik labi ir ieviestas digitālās inovācijas Eiropas valstīs un tās tiek savstarpēji salīdzinātas. Turpmāk
tiek apskatīti Latvijā eksistējošo kursu piedāvājumi un Skola 2030 apraksts. Tiek veikta digitālo prasmju
nozīmīguma un kursu pieejamības analīze ekspertu skatījumā. Šīs analīzes rezultātā tiek izvirzīta 
pamatproblēma, veikta tās analīze un izvēlēta konkrēti risināmā problēma. 
\textbf{Otrajā nodaļā} tiek definēti projekta mērķi, turpinājumā tiek izvirzītas trīs projekta alternatīvas,
Šīs alternatīvas tiek izvērtētas pēc dažādiem kritērijiem. Otrā nodaļas beigās tiek izvēlēta labākā alternatīva, 
kura tiek noformēta projekta priekšlikuma veidā. 
\textbf{Trešajā nodaļā} tiek izveidota projekta rokasgrāmata iepriekšējā nodaļā izvēlētajai alternatīvai,
kurā ietverts projekta starts, plānošana, pamatkoncepcija, detaļkoncepcija, realizācija, ieviešana un noslēgums. 
Darba noslēgumā tiek piedāvāti autora secinājumi un priekšlikumi.
\par
\textbf{Pētniecības metodes}, kuras autors izmantojis ir datu analīze,
prāta kartes, intervijas, riksa analīzes, tīkla plānošanu
\par
\textbf{Pētījuma periods} no 2005 līdz 2019
\par
Maģistra darba izstrādes procesa tika izmantoti tādi \textbf{literatūras un datu avoti} kā statistikas dati
gan no Eiropas Savienības, gan arī Latvijas, zinātniiskie darbi, noslēguma darbi, publiski pieejamā informācija
no tīmekļa vietnēm, projektu vadības teorijas literatūra un citi materiāli.
