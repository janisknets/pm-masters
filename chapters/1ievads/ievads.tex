\chapter*{Ievads} % * asterisk does not number the chapter
\addcontentsline{toc}{chapter}{Ievads} %adds unnumbered section to table of contents
Maģistra darba aktualitāte
\par
Projekta mērķis
\par
Projekta \textbf{pētījuma objekts} ir
\par
\textbf{Maģistra darba mērķis} ir 
\par
\textbf{Mērķa sasniegšanai veicamie uzdevumi} ietver:
\begin{enumerate}
    \item nozares raksturojuma veikšanu, noskaidrojot aktuālās tendences;
    \item noskaidrot nozares ekspertu viedokli par ;
    \item veikt problēmas noteikšanu un analīzi;
    \item izstrādāt projekta priekšlikumu;
    \item izstrādāt projekta īstenošanas rokasgrāmatu;
    \item izstrādāt secinājumus un priekšlikumus.
\end{enumerate}
\par
Maģistra darbu veido trīs daļas, kā arī secinājumi, priekšlikumi un pielikumi.
Darba \textbf{pirmajā nodaļā} tiek aplūkots Eiropas Savienībā un Latvijā, lai gūtu priekšstatu par 
aktuālajām tirgus tendencēm, kā arī gūtu priekšstatu par. Papildus tam tiek ekspertu skatījumā. 
Balstoties uz veikto izpēti, tiek izvirzīta pamatproblēma, veikta tās analīze un izvēlēta konkrēti
risināmā problēma. 
\textbf{Otrajā nodaļā} tiek definēti projekta mērķi,
izvirzītas trīs alternatīvas, tās izvērtētas pēc dažādiem kritērijiem un izvēlēta labākā alternatīva, 
kura noformēta projekta priekšlikuma veidā. 
Savukārt, \textbf{trešajā nodaļā} tiek izveidota projekta
rokasgrāmata izvēlētajai alternatīvai, kurā ietverts projekta starts, plānošana, pamatkoncepcija,
detaļkoncepcija, realizācija, ieviešana un noslēgums. 
Darba noslēgumā tiek piedāvāti autore secinājumi un priekšlikumi. Pieliekam šeit vēl teksta un 
varbūt vēl nedaudz lai sanāktu papildus rinda, tomēr laikam nepietiek, varbūt tagad? BTW darbs tiek
organizēts pēc \gls{agile}
\par
\textbf{Pētniecības metodes}, kuras autors izmantojis ir datu analīze,
prāta kartes, intervijas, riksa analīzes, tīkla plānošana, datorprogrammas. LĪdz ar to viss 
būs veiksmīgi sarakstīts un te būs lielisks teksts, paldies ka šo izlasāt
\par
\textbf{Pētījuma periods} no 2005 līdz 2019
\par
Maģistra darba izstrādē tiek izmantoti tādi \textbf{literatūras un datu avoti} kā LR likumdošanas aktu
prasības, statistikas dati, Latvijas un ārvalstu zinātnieku darbi un noslēguma darbu publikācijas,
aktuālās publikācijas medijos, publiski pieejamā informācija iestāžu un organizāciju mājaslapās,
projektu vadības teorijas literatūra, kā arī dažādi nepublicētie materiāli.