\subsection{Skola 2030 projekta analīze}
%1.2.2
\gls{visc} ir izveidojis Skola 2030 programmu kuras mērķis ir izstrādāt, izmēģināt un pēctecīgi ieviest Latvijā
izglītību, kas ļautu skolniekiem veiksmīgāk iesaistīties mūsdienu dzīvē un gūtu tai nepieciešamās zināšanas,
prasmes un attieksmes.
\par
Projekts ilgts kopā piecus gadus - no 2016. gada 17. oktobra līdz 2021.gada 16. oktobrim. Šobrīd notiek
sākotnējā aprobācijā 100 pilotskolās, kā arī tiek veidoti jauni izglītības materiāli.
\par
Projekts arī ietver izglītības mācībū spēku, proti, pedagogu un skoltāju, kvalifikācijas uzlabošanu. Šim nolūkam
top mācību līdzekļu vietne, tā palīdzētu nodrošināt pieeju nepieciešamai dokumentācijai un digitāliem mācību
materiāliem ikvienā skolā. 
\par
Dotais projekts tiek īstenots sadarbībā ar lielākām Latvijas universitātēm, citām izglītības iestādēm.
\par
Jaunā izglītības sistēma paredz mācību priekšmetus apkopot dažādās jomās:
\begin{itemize}
    \item Valodu; 
    \item sociālā un pilsoniskā;
    \item kultūras izpratnes un pašizpausmes mākslā;
    \item dabaszinātņu; 
    \item matemātikas;
    \item tehnoloģiju;
    \item veselības un fiziskās aktivitātes
\end{itemize}
Kā arī tās tiek saistītas ar caurviju prasmēm:
\begin{itemize}
    \item Kritiskā domāšana un problēmrisināšana;
    \item jaunrade un uzņēmējspēja; 
    \item pašvadīta mācīšanās; sadarbība;
    \item pilsoniskā līdzdalība; digitālās prasmes.
\end{itemize}
Visbeidzot ir arī vērtību un tikumu saraksts: atbildība, centība, drosme, godīgums, gudrība,
laipnība, līdzcietība, mērenība, savaldība, solidaritāte, taisnīgums, tolerance
\par
Aprobācijas laikā tiek veidoti saistošie dokumenti un tiek veidotas vadlīnijas jauno mācību materiālu veidošanai.
Visās Latvijas skolās izmantotais standarts tika apstiprināts 2018. gada 27. novembrī un pēc tā veidotās mācību
programmas izmantošana sāksies 2019./2020. mācību gadā. Sākotnēji šo programmu ieviešot pakāpeniski 3 gadu laikā.
\par
Projekts tiek finansēts no valsts budžeta un Eiropas Savienības fonda; 2 094 133 eiro un 11 866 751 eiro attiecīgi;
kopējais sākotnējais projekta budžets bija 13 960 884 eiro. 2018. gada 13. novembrī finansējums tika palielināts
līdz 18 458 382 eiro, kas ietver 15 689 624 eiro veido ESF finansējumu un  2 768 758 euro valsts budžeta finansējumu.     
