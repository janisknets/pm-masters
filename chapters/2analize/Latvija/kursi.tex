\phantomsection
\subsection{Latvijā pieejamo digitālo prasmju kursu analīze}
%1.2.1
Turpmāk apskatāmo izglītības materiālu autoru sarakstu var atrast \ref{app:pieejamo_kursu_aspkats} pielikumā.
Šie kursi tika salīdzināti pēc piedāvāto kursu skaita, kursu garuma, cenas, un vai tie ir klātienes vai nē.
Kursi tika izvēlēti pēc pirmajiem atrastajiem tīmeklī, tādā veidā izmantojot meklēšanas algoritmus savā labā.
\par
\textbf{Code Academy} ir Lietuvas uzņēmuma filiāle Latvijā. Kopumā piedāvā 9 dažādu kursu veidus. Piedāvāto kursu klāsts
sedz plaši pielietotās programmēšanas valodas, tādas kā Java un PHP, kā arī HTML, CSS. Taču ir arī pieejami
tīmekļa vietņu dizaina kursi. Kā arī SEO, jeb tīmekļu vietņu pielāgošana tīmekļu meklēšanas algoritmiem.
Cenas ir sākot no 489 EUR un sasniedz pat 3500 EUR. 3500 EUR ir jāmaksā par kursu kurš tiek saukts par
"Studijas 101" (piez. programmēšanas pasaulē bieži vien par pirmo / iesācēju līmeņa apmācības sauc par 1.0.1).
Šie jau vairs nav kursi, bet tiek pozicionēti kā studiju aizstāšana ar daudz īsāku laiku (1.5 gadi pret 3-4 gadiem)
\par
\textbf{Riga Coding School} ir vēl viens Lietuvas meitas uzņēmums Latvijā. Piedāvājumu klāsts ir līdzīgs konkurentiem.
Taču tas tiek piedāvāts grupētā variantā, kopumā ir 12 dažādas programmas. Tiek arī piedāvāti kursi darbinieku 
kvalifikācijas celšanai. Cenas variē no 590 EUR līdz 1050 EUR. Apmācību ilgums ir no 4 līdz 7 nedēļām. Riga coding
school arī piedāvā lielo datu apstrādes kursus, kas ir diezgan reti sastopams citiem konkurentiem.
\par
\textbf{Smart Ninja} piedāvājumā ir 8 kursi. Kursu garums ir atkarīgs no konkrētā kursa, tas ir pat no 3 stundu(!)
garuma līdz 36 stundu garumam. Cenas variē arī līdzvērtīgi - no 99 EUR līdz pat 499 EUR. Jāpiezīmē ka šīs cenas
ir "Agrā putniņa", un ir otrā cena, kas, parasti, ir divas reizes lielāka.  Interesanti atzīmēt, ka
Smart Ninja piedāvā kursus blokķēdes tehnoloģijā, kura par pamatu tiek izmantota kripto valūtām, kā arī
kiberdrošības apmācības un sarakstes robotu veidošanas kursus.
\par
\textbf{TSI} papildus pie izglītības programmām piedāvā dažus no saviem universitātes priekšmetiem kā atsevišķus
apmacības kursus. Diemžēl lielākā daļa no kursiem ir pieejama tikai krievu vai angļu valodās, līdz ar to
tabulā tika atzīmēti tikai divi kursi. Šo kursu garums ir 42 stundas un maksa ir 300 EUR. Kursi ir par C++ 
programmēšanu. Kopējais garums ir 2 mēneši.
\par
\textbf{RTU} ir otra augstskola kura piedāvā kursus. Šeit kursu garums ir vai nu 28 stundas, vai 32 stundas.
Cenas svārstās no tuvu 200 EUR līdz 423.50 EUR. RTU ir vienīgie kuri piedāvā AutoCAD programmas vidi, kas ir
viena no visizplatītākajām rasēšanas programmatūrām pasaulē. 
\par
\textbf{Codelex} ir viens no interesantākiem konkuretniem tirgū, jo viņi neprasa apmaksu par saviem izglītības
kursiem, tajā vietā tiek noslēgts līgums, kura viens no galvenajiem nosacījumiem ir, ja absolvents iegūst darbu
IT vidē, un, pēc nodokļu nomaksas, saņem vairāk nekā 1000 EUR, tad divu gadu laikā viņš maksās 20\% no savas algas
Codelex par sniegto izglītību. Kopumā tiek piedāvāti 3 kursi, ar lielu apjomu mājasdarbu, klātienē būs jātiekas 20
reizes, taču ir pietiekoši apjomīgs mājasdarbu skaits. Kā arī ir salīdzinoši strikti noteikumi par apmeklējumu un
mājasdarbu pildīšanu.
\par
\textbf{Latvijas tālmācības profesionālais centrs} vizuāli izskatās nedaudz novecojis uz pārējo piedāvājumu fona.
Kopumā tiek piedāvāti 2 dažādi kursu veidi, taču to garums ir 960h, maksa ir arī līdzvērtīga - 1620 EUR, apmacības
ilgs 15 mēnešus, apmaksa ir sadalīta pa mēnešiem. Tiek apmacītas nepieciešamās prasmes programmēšnas darbam. Apmācības
nav klātienē, kas varētu būt par vienu no labākiem plusiem šim piedāvājumam. Jāizceļ arī sadarbību ar \acrshort{nva}.
\par
\textbf{LLU Neklātienes programmēšanas skola} piedāvā visizdevīgāko variantu - par apmacībām vispār nav jāmaksā.
Apmācības ilgs no Novembra līdz jūnijam izmantojot e-studiju vidi. Ieguldītais laiks ir atkarīgs no interesenta
iespējām, līdzīgi būs arī rezultāts. Kopā tiek piedāvāti 6 dažādi kursi. Tā pat kā iepriekšējam piedāvājumam, 
šijā gadījumā nav klātienes kursu.
\par
Ir vēl vairāki citi piedāvājumi, daži no tiem ir domāti tikai skolniekiem, citi piedāvā līdzīgus kursus. 
Kopumā cenas ir līdzvērtīgas un piedāvājums krasi arī nemainās - Java, python, HTML+CSS+Javascript, PHP.
Diezgan reti var atrast kursus, kuri ļautu apgūt papildus nepieciešamās zināšanas citām profesijām izņemot
programmētājus.
