\subsection{Start(it) fonda analīze}
Start(it) fonda pirmsākumi ir meklējami 2014 gadā, kad Accenture Latvija redzot, ka Latvijas izglītības
sistēma nepiedāvā programmēšanu skolās, programmā eksistē tikai datorika, nolēma izveidot fondu, kuram
vajadzētu sekmēt programmēšanas apmācību Latvijas skolās.
\paragraph{}
Protams valsts izglītības satura centrs neļaus tik vienkāršu iejaukšanos izglītības programmā, līdz ar
to kursi tika ieviesti 760 pilotskolās. Šī programma arī vairāk tika izmantota kā papildus pulciņi 
interesentiem, nevis kā obligātās apmācības.
\paragraph{}
Piecu gadu rezultātā tika izveidoti kursi latviešu valodā. Ar šo kursu palīdzību jebkurš skolnieks 
var apgūt programmēšanas pamatus jebkurā Latvijas mājā, ja vien viņam ir piekļuve pie interneta. Šos kursus palīdzēja
veidot gan paši skolotāji, gan universitāšu pasniedzēji, gan nozares profesionāļi.
\paragraph{}
Latvija ar 2020 gadu sāks pārēju uz jaunu izglītības sistēmu - Skola2030; Start(it) fonds vēlētos pievienoties
kā galvenais satura veidotājis programmēšanas saturam. Tas būtu izdevīgi gan skolām, gan skolniekiem,
gan arī beigās fonda dalībniekiem, jo pēc 3-5 gadiem tie spēs iegūt apmācītus un specīgus darbiniekus.
Šīs iemaņas arī stiprinās Latvijas pozīcijas kopējā Eiropas darba tirgū. Tajā pašā laikā tas veido risku
par darba spēka aizplūšanu uz citām Eiropas valstīm, kur atalgojums ir salīdzinoši lielāks.
\paragraph{}
Šis projekts sastapās ar vairākām problēmām - fondam pievienojās ne tik daudz gribētāju, dotajā brīdī
saturs ir novecojis un neatbilst Skola2030 un VISC prasībām. Tomēr šis projekts saglabā lielu potenciālu.
Viens no lielākiem plusiem šiem kursiem ir tāds, ka tos var izmantot ne tikai bērni, bet jebkurš Latvijas
iedzīvotājs. Programmēšanas iemaņas būs nepieciešamas ar vien vairāk mūsu ikdienas darbā, līdz ar to fonda
attīstība varētu ietevert ne tikai skolas, bet arī mūžizglītībā iesasistītos iedzīvotājus. Šis plaši palielinātu
uzņēmuma atpazīstamību un ļautu piesaistīt jaunos darbiniekus.
%TODO some fixing required
\par
Nākošā apakšnodaļā tiks apskatīts viedoklis par digitālo prasmju pieejamību un nozīmīgumu Latvijā. Tiek veikta
ekspertu intervēšana un vēlāk viņu atbilžu analīze.
