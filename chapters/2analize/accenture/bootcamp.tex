\phantomsection
\subsection{Bootcamp rekrutēšanas programmas analīze}
Bootcamp angļu valodas nozīmi var skaidrot kā militāras apmācības nometne, kurā apmāca jauniesauktos
karavīrus. Accenture Latvija jau kopš 2005 gada veido savus Bootcamp nometnes, kur vienas līdz četru 
nedēļu laikā tiek apmācīts jebkurš cilvēks, lai viņš varētu kļūt par uzņēmuma darbinieku.
\par
Kursi sākumā nebija plaši, apmācīja izmantot jaunākās tehnoloģijas, jo ne universitātēs, ne citur
nevarēja apgūt uzņēmumam vajadzīgās prasmes attiecīgā līmenī. Šī tradīcija turpinās arī šodien,
tehnoloģijas ir mainījušās, bet kursi ir kļuvuši par ļoti veiksmīgu projektu.
\par
Pēdējā gada laikā tika apmācīti vairāk nekā 800 cilvēki, no kuriem vairāk nekā 750 palika uz tālākām
apmācībām kā praktikanti. Lielākā daļa vēlāk arī tika pieņemti kā uzņēmuma darbinieki. Šie kursi ir
viens no galvenajiem veidiem kā uzņēmums ir spējis augt tik ātri, tai skaitā iegūstot VID apbalvojumu.
\par
Salīdzinot šos kursus ar citām alternatīvām Latvijā, kā jau minēts iepriekšējā nodaļā,
ir diezgan grūti atrast konkurentu, tā iemesla dēļ, ka šie kursi ir bez maksas un konkurējošie projekti
Latvijā piedāvā līdzvērtīgu materiālu par maksu un ne vienmēr ar iespēju pēc tam turpināt strādāt 
darba tirgū.
\par
Kursi pārsvarā tiek piedāvāti studentiem, kuriem vēl nav nekādas darba pieredzes. Šī ir iespēja
ar kuras palīdzību var audzēt savas prasmes strādājot industrijā. Tā kā kursi ir ļoti populāri un piedāvā
iespēju uzsākt karjeru ITK jomā, ar vien biežāk var redzēt cilvēkus no citām nozarēm, kuri vēlas iegūt
darbu jaunajā nozarē. Šie cilvēki apzinās, ka nozare kurā viņi darbojās pirms tam nav ne tik labi atalgota,
ne dod pietiekoši daudz iespējas izaugsmei un savai nākotnes labklājībai.
\par
