\section{Digitālās ekonomikas un savienojamības indekss}
Lai aktualizētu problēmu un izpētīt kopēju ietekmi uz darba tirgu no digitālām prasmēm 
autors nolēma izpētīt pieejamo statistiku \gls{es} līmenī, viens no labākiem
rādītājiem ir \gls{desi}. Šis indekss apkopo sevī 5 rādītāju kopas:
\begin{enumerate}
    \item Savienojamība
    \item Cilvēkkapitāls
    \item Interneta lietošana
    \item Digitālo tehnoloģiju integrācija
    \item Digitālie publiskie pakalpojumi
\end{enumerate}

\paragraph{}
Latvija dotajā indeksā ieņem 19. vietu no 28 \acrshort{es} valstīm. Salīdzinot
ar tuvākiem kaimiņiem Latvija ir pēdējā - Lietuva ir 12. vietā un Igaunija 9 vietā. Pēdējo
gadu laikā Latvijas rādītāji ir nemainīgi, kas norāda uz to ka Latvijas attīstība DESI
ir līdzvērtīga citām \acrshort{es} valstīm. Latvija ir sasniegusi lielus panākumus 
Digitālo publisko pakalpojumu rādītājā, Lielākās problēmas jau vairākus gadus sagādā
Cilvēkkapitāls un Digitālo tehnoloģiju intergrācijas rādītāji.
\paragraph{}
Kopumā Latvija pieder pie to valstu grupas, kuru rezultāti ir vidēji.
\paragraph{}
2013. gadā valdība pieņēma Informācijas sabiedrības attīstības pamatnostādnes 2014.–
2020. gadam – pašreizējo nacionālo digitalizācijas stratēģiju. Pamatnostādnes balstītas uz
septiņiem pīlāriem: IKT izglītība un prasmes, plaši pieejama piekļuve internetam, moderna
un efektīva valsts pārvalde, sabiedrībai pieejami e-pakalpojumi un digitālais saturs,
pārrobežu sadarbība digitālā satura vienotajā tirgū, pētniecība un inovācija IKT jomā,
uzticēšanās un drošība\cite{desi_Latvija, soc_dev}.
