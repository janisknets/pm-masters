\phantomsection
\section{Digitālās ekonomikas un savienojamības indekss}
Ar vien biežāk var lasīt rakstus par kādu drīz izzudošu aroda veidu, kāds sola, ka roboti un 
mākslīgais intelekts pārņems darba vietas un cilvēkiem vairs nebūs ko darīt. Autors uzskata,
ka šādam procesam ir taisnība, taču tas nav kaut kas jauns, tas ir noticis gadsimtiem ilgi.
Vienīgā starpība mūsdienās ir tāda, ka šis process notiek daudz ātrāk un ietekmē cilvēkus viņa
mūža laikā tādā veidā, ka viņam ir jāapgūst jaunas iemaņas, ja viņš vēlas konkurēt darba tirgū
par labi atalgotu darbu.
\par
Lai aktualizētu problēmu un izpētīt kopēju ietekmi uz darba tirgu no digitālām prasmēm 
autors nolēma izpētīt pieejamo statistiku \gls{es} līmenī, viens no labākiem
rādītājiem ir \gls{desi}. Šis indekss apkopo sevī 5 rādītāju kopas: Savienojamība, Cilvēkkapitāls
Interneta lietošana, Digitālo tehnoloģiju integrācija, Digitālie publiskie pakalpojumi
\par
Latvija dotajā indeksā ieņem 19. vietu no 28 \acrshort{es} valstīm. Salīdzinot
ar tuvākiem kaimiņiem Latvija ir pēdējā - Lietuva ir 12. vietā un Igaunija 9 vietā. Pēdējo
gadu laikā Latvijas rādītāji ir nemainīgi, kas norāda uz to ka Latvijas attīstība DESI
ir līdzvērtīga citām \acrshort{es} valstīm. Latvija ir sasniegusi lielus panākumus 
Digitālo publisko pakalpojumu rādītājā, Lielākās problēmas jau vairākus gadus sagādā
Cilvēkkapitāls un Digitālo tehnoloģiju intergrācijas rādītāji.
\par
2013. gadā valdība pieņēma Informācijas sabiedrības attīstības pamatnostādnes 2014.–
2020. gadam – pašreizējo nacionālo digitalizācijas stratēģiju. Pamatnostādnes balstītas uz
septiņiem pīlāriem: IKT izglītība un prasmes, plaši pieejama piekļuve internetam, moderna
un efektīva valsts pārvalde, sabiedrībai pieejami e-pakalpojumi un digitālais saturs,
pārrobežu sadarbība digitālā satura vienotajā tirgū, pētniecība un inovācija IKT jomā,
uzticēšanās un drošība \cite{desi_Latvija} \cite{soc_dev}.
