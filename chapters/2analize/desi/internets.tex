\desigraph{desi_internets}{3. Internets}
  {(2014,7.77212)(2015,7.83149)(2016,8.21007)(2017,8.17428)(2018,8.21653)}
  {(2014,7.5689)(2015,8.25655)(2016,8.10022)(2017,8.34098)(2018,8.52669)}
  {(2014,8.05949)(2015,8.53042)(2016,8.81596)(2017,8.99781)(2018,9.24099)}
  {(2014,6.00682)(2015,6.4728)(2016,6.7953)(2017,7.12884)(2018,7.57412)}

Rādītāji - Interneta pakalpojumu izmantošana
3a Saturs (33\%), 3b Komunikācija (33\%), 3c Pārskaitījumi (33\%)

Interneta lietotāju īpatsvars Latvijas iedzīvotāju vidū joprojām pārsniedz ES vidējo rādītāju.
Jo īpaši augstāks par vidējo ir internetbankas lietotāju īpatsvars (75 \%, kas Latviju ierindo 8.
vietā ES), taču iecienīti ir arī citi interneta pakalpojumi – ziņu lasīšana (84 \%), mūzikas
klausīšanās, video skatīšanās vai spēļu spēlēšana (77 \%) un sociālo tīklu izmantošana
(74 \%). No otras puses, iepirkšanās tiešsaistē ir salīdzinoši mazāk populāra. Patiešām,
pagājušajā gadā tikai nedaudz vairāk par pusi (55 \%) interneta lietotāju norādīja, ka 2017.
gadā ir iepirkušies tiešsaistē (ES tie ir 68 \%).
