\desigraph{desi_cilvekkapitals}{2. Cilvēkkapitāls}
  {(2014,9.8611)(2015,10.2967)(2016,10.7013)(2017,11.0141)(2018,10.9601)}
  {(2014,11.7685)(2015,12.6421)(2016,14.0537)(2017,15.5321)(2018,16.0279)}
  {(2014,13.2348)(2015,14.1657)(2016,14.5055)(2017,14.4909)(2018,15.3438)}
  {(2014,12.2822)(2015,12.8878)(2016,13.1634)(2017,13.6546)(2018,14.1207)}

Definīcija: 2a Vienkāršo iemānas un to izantošana, 2b - padziļinātās digitālās prasmes un to izmantošana.
2b1 - IKT specialistu skaits pret iedz. skaitu, 2b2 - STEM universitātes beidzēji


Cilvēkkapitāla aspektā Latvija atpaliek no ES vidējās vērtības, un pēdējā gadā progress nav
panākts. Interneta lietotāju īpatsvars iedzīvotāju vidū gandrīz atbilst ES vidējam rādītājam,
tomēr 52 \% Latvijas iedzīvotāju joprojām trūkst digitālo pamatprasmju, kas tiem liedz efektīvi
lietot internetu, turklāt 19 \% digitālo prasmju vispār nav (par 2 punktiem vairāk nekā ES
vidējais rādītājs).
\par
Latvijā digitālo prasmju līmenis sieviešu vidū ir nedaudz augstāks nekā vīriešiem. Sieviešu
vidū vismaz 50 \% ir digitālās pamatprasmes, taču vīriešiem tie ir tikai 46 \%. Atšķirīgs ir arī
strādājošo un nestrādājošo iedzīvotāju digitālo prasmju līmenis. No strādājošajiem 57 \%
digitālās prasmes ir pamata vai augstākā līmenī, savukārt nestrādājošo vidū šis rādītājs ir
tikai 33 \%. Arī izglītības līmenis ir svarīgs faktors saistībā ar digitālo prasmju apguvi. No tiem,
kas ieguvuši augstāko izglītību, 76 \% ir vismaz digitālās pamatprasmes (ES līmenī tie ir
84 \%), taču pamatizglītību vai vidēja līmeņa izglītību ieguvušo vidū šis īpatsvars ir tikai 35 \%.
Mazizglītotiem cilvēkiem šis rādītājs ir par 5 \% augstāks nekā ES vidējais, savukārt vidēji
izglītotiem cilvēkiem šī starpība salīdzinājumā ar ES vidējo rādītāju ir 20 punkti. IKT
speciālistu skaits ir stabils, taču ievērojami zem ES vidējā līmeņa. Turklāt pēdējos gados ir
samazinājies absolventu skaits STEM jomā (2013. gadā 14,1, bet 2016. gadā tikai 12,7 uz
1000 iedzīvotājiem).
\par
Izglītības attīstības pamatnostādnes 2014.–2020. gadam ietver rīcības virzienus, kas skar
IKT izmantošanu mācību procesā un digitālo prasmju pilnveidošanu. Dokumentā
“Informācijas sabiedrības attīstības pamatnostādnes 2014.–2020. gadam” zem pīlāra “IKT
izglītība un e-prasmes” ir paredzētas šādas darbības izglītības jomā: sabiedrības informētība
un gatavība izmantot e-iespējas, iedzīvotāju un uzņēmēju e-prasmju pilnveidošana, IKT
prasmju palielināšana valsts pārvaldē, IKT speciālistu un darbinieku sagatavošana atbilstīgi
darba tirgus vajadzībām, kā arī algoritmiskās domāšanas un informācijpratības palielināšana
izglītības programmās. Šīm darbībām tiek piešķirts valsts finansējums, kā arī ES finanšu
atbalsts.
\par
Turklāt Latvijā ir izveidota sava digitālo prasmju un darbvietu koalīcija, kurā iesaistītas
vairākas ministrijas, IKT nozares apvienības un uzņēmumi, kā arī Latvijas Tirdzniecības un
rūpniecības kamera. Koalīcijas darbu koordinē Latvijas Informācijas un komunikācijas
tehnoloģijas asociācija (LIKTA). Koalīcijas darbā ir noteikti prioritārie virzieni, kas definēti
iepriekš minētajos dokumentos un kas orientēti uz šādiem mērķiem: nodrošināt IKT
apmācību atbilstīgi darba tirgus vajadzībām, iesaistīt jauniešus IKT jomā, izveidot
mūsdienīgus un interaktīvus mācību procesus, palielināt informētību par digitālās pratības un
IKT prasmju nozīmīgumu.
\par
Pagājušajā gadā ir veikti vairāki pasākumi, lai īstenotu šīs stratēģijas, piemēram, projekti
“MVU apmācības inovāciju un digitālo tehnoloģiju attīstībai Latvijā” un “IKT profesionāļu
apmācības inovāciju veicināšanai un nozares attīstībai”. Šo projektu mērķis: sniedzot
iespēju apgūt nākotnes digitālajās darbvietās vajadzīgās IKT prasmes, atbalstīt jauniešu
nodarbināmību un personisko izaugsmi. Mērķis ir laikposmā no 2017. līdz 2020. gadam
noorganizēt augstas kvalitātes digitālo prasmju kursus 7000 MVU darbiniekiem un 1500 IKT
speciālistiem. 2017. gada oktobra beigās pirmajā projektā bija iesaistījušies jau vairāk nekā
400 uzņēmumi un no 7000 iecerētajām apmācībām bija noorganizētas vairāk nekā 900. Ap
šo pašu laiku 55 IKT uzņēmumi bija iesaistījušies otrajā projektā un 196 augsta līmeņa
specializētos IKT apmācību kursos savas prasmes un kvalifikācijas bija atjauninājuši 780 IKT
speciālistu.
\par
Šajā jomā ir veikti daudzsološi pasākumi, un var paiet noteikts laiks, kamēr būs jūtama to
ietekme, taču Latvijai vēl ir jāstrādā pie tā, lai uzlabotu iedzīvotāju un darbaspēka digitālās
prasmes, tādējādi labāk sagatavojoties savas tautsaimniecības un iedzīvotāju digitālajai
pārejai