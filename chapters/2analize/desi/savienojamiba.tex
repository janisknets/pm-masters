\phantomsection
\desigraph{desi_savienojamiba}{1. Savienojamība}
    {(2014,11.3305)(2015,11.3951)(2016,15.0115)(2017,15.4357)(2018,16.4)}
    {(2014,09.6959)(2015,11.6532)(2016,14.3374)(2017,15.3726)(2018,16.2237)}
    {(2014,11.7685)(2015,12.6421)(2016,14.0537)(2017,15.5321)(2018,16.0279)}
    {(2014,11.0332)(2015,12.4053)(2016,13.4780)(2017,14.6234)(2018,15.6445)}

Savienojamības rādītājs veidojās no 5 rādītājiem, 1a - piekļuve platjoslas internetam (20\%)(>30Mbps),
1b - piekļuve mobīlajam platjoslas internetam (30\%)(>30Mbps), 1c - piekļuve ātrdarbīgajam platojoslas internetam (20\%)(<100Mbps),
1d - īpāsi ātrdarbīgā interneta piekļuve (20\%)(>100Mbps), 1e - platjoslas interneta cena (10\%)
\par
Latvija kopumā ir labāki rādītāji nekā Eiropā. Uz doto brīdi 1a rādītāja Latvija atpaliek
no Eiropas rādītājiem. Tas ir izskaidrojams ar to ka jau vairāk nekā 93\% iedzīvotājiem
ir piekļuve pie platojoslas interneta, pārējie 7\% ir ārpus pilsētām, privātmājas, lauku mājas utml,
tādas vietas, uz kurām nav izdevīgi veikt tiešo pieslēgumu, jo Latvijā ir ļoti labs
4G mobilā tīkla pārklājums, kā arī Latvijas Mobīlais telefons piedalās aktīvā 5G tīkla
attīstībā. LĪdz ar to 1b rādītāji Latvijai ir salīdzinoši labi, taču ja vērtēšana būtu
striktāka, tad Latvija noteikti izceltos uz Eiropas līmeņa. Latvijai ir labākais rādītājs
atrdarbīgā un īpaši ātrdarbīgā platjoslas interneta rādītāja (1c un 1d). Apsteidzot gan kaimiņus,
gan Eiropu kopumā. Daļēji tam palīdz iedzīvotāju koncetrēšanos ap vienu lielo centru - 
Rīgas pilsētu, taču kopējie rādītāji visā valstī ir salīdzinoši labi.

\par
“Vidējās jūdzes projekts”, kas tika sākts 2012. gadā un kam piešķirts līdzfinansējums
no ES struktūrfondiem, lai lauku teritorijas savienotu ar valsts pamatinfrastruktūru,
tagad nonācis otrajā posmā. Plānots, ka otrā posma būvdarbi sāksies 2018. gada
pavasarī. Galvenokārt tie tiks veikti atlikušajās “baltajās” teritorijās (2014.–2015. gadā
apzināta 221 teritorija). Paredzēts, ka līdz 2020. gadam optiskie kabeļi tiks ierīkoti
2800 km garumā un izveidoti aptuveni 220 optiskā tīkla piekļuves punkti. Pēc tam
telesakaru operatoriem, izmantojot jauno tīklu, kas ļaus galalietotājiem piedāvāt
mazumtirdzniecības pakalpojumus, būs iespēja izveidot vietējās sakaru līnijas ar datu
pārraides ātrumu vismaz 30 Mbit/s (“pēdējā jūdze”). Tomēr šķiet, ka ne visur tiek
veiktas privātās investīcijas “pēdējās jūdzes” infrastruktūras izbūvē. Ir vajadzīgi
turpmāki centieni, piemēram, papildu valsts atbalsta shēmas un regulatīvi pasākumi,
lai novērtētu situāciju un piedāvātu risinājumus, kas attiecīgajos gadījumos ļautu
novērst ar “pēdējo jūdzi” saistīto plaisu. Iespēja mājās, pieslēdzoties no mobilajām
ierīcēm, izmantot mobilo operatoru nodrošinātos fiksētos sakaru pakalpojumus,
palīdz pārvarēt šo plaisu atsevišķos lauku apvidos, kur netiek veiktas investīcijas
“pēdējās jūdzes” savienojumos\cite{platjosla}
\par
Lai novērstu izveidojošos plaisu piekļuvē starp pilsētām un laukiem tika 
pieņemta direktīva par Platjoslas izmaksu samazināšanu. Kā arī liels uzvasrs tiek
likts uz 5G tīklu un viņa iespējām tieši laukos, kur nav tik daudz citu objektu, kuri 
bloķētu signāla pārraidi. Attiecīgi antenu skaitam būtu jābūt mazākam, nekā tas ir
nepieciešams pilsētās.

