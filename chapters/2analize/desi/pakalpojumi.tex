Pēdējā gada laikā digitālo publisko pakalpojumu jomā Latvijai ir izdevies būtiski uzlabot
rezultātu (+13 procentpunkti) un pakāpties no 14. uz 9. vietu. Šī pozitīvā tendence
skaidrojama ar e-pārvaldes pakalpojumu plašāku izmantošanu (+8 procentpunkti), vairāk
izmantotām automātiski daļēji aizpildītām veidlapām (+13 procentpunkti) un jo īpaši atvērto
datu pieejamību (+53 procentpunkti). Atvērto datu izmantošanu sekmējusi Latvijas Atvērto
datu portāla atklāšana, jo tādējādi nodrošināta tieša piekļuve valsts pārvaldes datu kopām
un metadatiem un iespēja savienot tās ar citām datu kopām, kas publicētas citos valsts
pārvaldes portālos. Salīdzinājumā ar iepriekšējo gadu šādā veidā ievērojami uzlabojies
valsts sniegums atvērto datu jomā, un tagad Latvija ierindojas 18. vietā ES.
\paragraph{}
E-pārvaldes politika galvenokārt ir izklāstīta dokumentā “Informācijas sabiedrības attīstības
pamatnostādnes 2014.–2020. gadam”, kur īpaša uzmanība ir veltīta atvērto datu principu
īstenošanai valsts pārvaldē un publisko pakalpojumu sniegšanas vienkāršošanai, kas
iespējama, pateicoties efektīviem un lietderīgiem e-pakalpojumiem un sadarbspējīgām
Digitālās ekonomikas un sabiedrības indekss 2018, ziņojums par Latviju. lpp. 11 no 11
informācijas sistēmām. Pīlārā “Sabiedrībai pieejami e-pakalpojumi un digitālais saturs” ir
ietverti šādi elementi: valsts pārvaldes datu un darījumu atvēršana citiem lietotājiem;
kopīgotas platformas un pakalpojumu izstrāde publisko pakalpojumu sniegšanai; tādu
oficiālo e-pastu adrešu izveide, kuras saziņai var izmantot iedzīvotāji un uzņēmēji; publisko
pakalpojumu digitalizācija; elektronisko rēķinu automatizēta izdošana un pieņemšana;
kultūras mantojuma digitalizācija un pieejamība; latviešu valodas lietojuma veicināšana
digitālajā vidē; e-veselības risinājumi efektīvai, drošai un uz pacientiem orientētai veselības
aprūpei. Pie pīlāra “Mūsdienīga un efektīva valsts pārvalde” ietvaros veiktajiem pasākumiem
jāmin valsts pārvaldes pamatdarbības modernizācija; publiskā e-līdzdalība un e-demokrātija;
vienota valsts pārvaldes datu telpa un IKT infrastruktūru optimizācija.
\paragraph{}
2018. gada februārī Ministru kabinets pieņēma informatīvo ziņojumu “Mākoņdatošanas
pakalpojumu izmantošana valsts pārvaldē”, kurā uzmanība vērsta uz mākoņdatošanas
pakalpojumu potenciālu valsts pārvaldes efektivitātes nodrošināšanā. Paziņojumā ierosināts
rīcības plāns nolūkā sagatavoties mākoņdatošanas pakalpojumu efektīvai izmantošanai
valsts pārvaldē, turklāt tajā iekļauti priekšlikumi par mākoņdatošanas pakalpojumu atsevišķu
vadības funkciju centralizāciju.
\paragraph{}
Paredzams, ka, samazinot administratīvo slogu, Latvijā tiks izveidota labvēlīgāka
uzņēmējdarbības vide un palielināsies to uzņēmumu (jo īpaši MVU) skaits, kuriem līdz šim
bijis grūtāk sākt savu uzņēmējdarbību vai oficiāli reģistrēties sarežģīto un apgrūtinošo
birokrātisko procedūru dēļ.
\par
Kā iezīmē iepriekšējā nodaļa Latvija ir salīdzinoši zemā vietā kopējā Eiropas līmenī.
Kamēr dažos rādītājos sasniegumi ir pat ļoti spoži, citi, svarīgāki, velk Latvijas kopējo
indeksu uz leju. Dotajā apakšnodaļā autors cenšas noteikt kādas iniciatīvas Latvijā jau
tika uzsāktas ar mērķi uzlabot šos rādītājus.
\par
Turpinājumā tiek apskatīti dažādi pieejamie kursi un izglītības materiāli un to cenas,
kā arī tiek apskatīta Skola 2030 programma.