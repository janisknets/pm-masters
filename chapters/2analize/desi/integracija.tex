Pagājušajā gadā saistībā ar digitālo tehnoloģiju integrāciju uzņēmumos Latvija ir guvusi
labus panākumus, no 25. vietas 2017. gadā pakāpjoties uz 23. vietu. Tomēr šajā jomā tā
joprojām atpaliek no lielākās daļas ES valstu. Situācijas uzlabošanos ietekmējuši uzņēmumi,
kas iegādājušies mākoņdatošanas pakalpojumus (pagājušajā gadā šis apjoms gandrīz
dubultojies, tagad sasniedzot 9,4 \%), un uzņēmumi, kas pieņēmuši elektronisku informācijas
koplietošanu. Par 2,5 procentpunktiem, sasniedzot 10,6 \%, palielinājies arī to MVU īpatsvars,
kas izmanto elektroniskos pārdošanas kanālus, tādējādi samazinot atšķirību no ES vidējā
rādītāja (17 \%). Nedaudz palielinājies arī to MVU apgrozījums, kuri nodarbojas ar ekomerciju 
(+0,5 procentpunkti, sasniedzot 8,6 \%). Tomēr varētu panākt vēl dažus
uzlabojumus, jo patlaban salīdzinoši maz uzņēmumu nodarbojas ar pārdošanu tiešsaistē pāri
robežām (4,7 \%). Augstās piegādes izmaksas ir galvenais šķērslis, ar ko nākas saskarties
uzņēmumiem, kuri vēlas tiešsaistē pārdot preces klientiem citās ES valstīs.
\par
Latvija nav izstrādājusi visaptverošu stratēģiju uzņēmumu digitalizācijai. Tomēr ir
sagatavotas vairākas iniciatīvas, kas sekmē “Rūpniecība 4.0” (Industry 4.0) izveidi; kā
piemēri minami izmēģinājuma projekts inženiertehniskajā nozarē, kas veicina izpratni par
koncepciju “Rūpniecība 4.0”, līdzdalība Interreg projektā “DIGINNO”, kurā iecerēts paātrināt
rūpniecības digitalizāciju Baltijas jūras reģionā, kā arī Interreg projekts “SKILLS+”, kura
mērķis ir veicināt tādu valsts politiku, kas sekmē IKT prasmju apgūšanu MVU vidū lauku
apvidos.
\par
Digitālās ekonomikas un sabiedrības indekss 2018, ziņojums par Latviju. lpp. 9 no 11
Tehnoloģiju pārneses programmas ietvaros paredzēts arī atbalsts inovācijas kuponu
izmantošanai. Inovācijas kuponu mērķis ir atbalstīt inovācijas darbības MVU vidē, sniedzot
tiem atbalstu pētniecības un izstrādes ārpakalpojumu izmantošanai, kas tiem ļautu ieviest
jaunus vai būtiski uzlabotus produktus vai tehnoloģijas.
Visaptverošas stratēģijas pieņemšana varētu palīdzēt uzlabot digitālo pāreju
tautsaimniecībā, piemēram, MVU un iedzīvotājiem nodrošinot plašāku piekļuvi daudz
lielākam tirgum.