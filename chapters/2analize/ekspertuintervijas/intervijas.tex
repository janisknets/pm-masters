\section{Padziļināto datorprasmju izglītības pieejamības analīze no ekspertu viedokļa}
\phantomsection
\subsection{Pētījuma metodoloģija}
%.1.4.1 TODO - padomāt par stilu
Autors nolēma izmantot Delfi aptaujas metodi, tā ļauj uzzināt dažādu pušu viedokli un iesaistītās puses
neietekmē viena otru intervijas laikā. Ja ir nepieciešams, tad aptaujas var atkārtot kārtās piedāvājot iepriekšējo
dalībnieku atbildes. Vienkāršā aptauja netika izmantota, jo iespējas veikt visaptverošu aptauju būtu sarežgīti
un rezultāti būtu vairāk piesaistīti konkrētai grupai cilvēku. Attiecīgi iegūtās atbildes nepareizi attēlotu reālo situāciju
\par
Intervijas sastāv no X jautājumiem, jautājumi ir atvērtā tipa, līdz ar to intervējamie varēja sniegt savu viedokli par
uzdoto jautājumu nevis vienkārši atbildēt uz iepriekš sagatavotiem jautājumiem ar Jā/Nē.
\begin{enumerate}
    \item \textit{Vai jūsuprāt Padziļinātās datorprasmes zināšanas būs ar vien vairāk nepieciešamas darba tirgū?}
    \item \textit{Kādas padziļinātās datorprasmes ir nepieciešamas jūsu darbā šodien?}
    \item \textit{Kādas padziļinātās datorprasmes jūs gribētu zināt vai jūtat ka būtu nepieciešams zināt?}
    \item \textit{Vai varat nosaukt, kur Latvijā var apgūt datorprasmes gan pamata, gan padziļinātās?}
    \item \textit{Kuras nozares Latvijai vajadzētu izvirzīt par prioritāti un sekmēt to attīstību?}
\end{enumerate}
\par
Intervijās piedalījās ITK jomas darbinieks, atlases personāla speciālists, izglītības sektora darbinieks,
divi citu nozaru specialisti.
\subsection{Pētījuma rezultātu analīze un interpretācijas}
1. jautājuma 
Noteikti jā, jau šodien varam redzēt prasības pēc pamata datorprasmēm darba sludinājumos, lai gan vēl pirms 5-10 gadiem
šādas prasības nebija.
Tie, kuriem būs šādas prasmes, spēs nodrošināt lielāku darba ražīgumu, kas ietekmēs viņu karjeras izaugsmi, līdz ar to 
tas ir lielisks ieguldījums nākotnē.
\subsection{Otrā jautājuma atbilžu analīze}
Drošā interneta lietošana, Informācijas meklēšana globālajā tīmeklī. Procesu automatizēšana, specializētās
programmatūras izmantošana
\subsection{Trešā jautājuma atbilžu analīze}
Vēl vairāk tehnoloģijas,
Automatizācija
\subsection{Ceturtā jautājuma atbilžu analīze}
Nē 
\subsection{Piektā jautājuma atbilžu analīze}
ITK protams, jo tā dod to ko mūsu politiķi saka jau vairākus gadus - veido darba vietas ar augsto pievienoto vērtību
pārsvarā fokusējoties uz eksportu, pie tam nav vajadzīgi nekādi dabas resursi.