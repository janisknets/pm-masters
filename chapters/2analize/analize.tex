\chapter{Digitālo prasmju ietekmes analīze uz darba tirgu}
Mūsdienu pasaule strauj attīstās un ar vien vairāk tiek runātas par dažādu arodu izzušanu.
Taču veidojās arī jaunie darba veidi, ar vien vairāk šie jaunie arodi prasa cilvēkam zināt
jaunās tehnoloģijas, datorprasmi gan pamatprasmju līmenī, gan padziļināti.
Autors veic izpēti par Latvijas rādītājiem Eiropas Savienības līmenī izmantojot \acrlong{desi} 
(turpmāk tekstā \acrshort{desi})
\section{Digitālās ekonomikas un savienojamības indekss}
Lai aktualizētu problēmu un izpētīt kopēju ietekmi uz darba tirgu no digitālām prasmēm 
autors nolēma izpētīt pieejamo statistiku \gls{es} līmenī, viens no labākiem
rādītājiem ir \gls{desi}. Šis indekss apkopo sevī 5 rādītāju kopas:
\begin{enumerate}
    \item Savienojamība
    \item Cilvēkkapitāls
    \item Interneta lietošana
    \item Digitālo tehnoloģiju integrācija
    \item Digitālie publiskie pakalpojumi
\end{enumerate}

\paragraph{}
Latvija dotajā indeksā ieņem 19. vietu no 28 \acrshort{es} valstīm. Salīdzinot
ar tuvākiem kaimiņiem Latvija ir pēdējā - Lietuva ir 12. vietā un Igaunija 9 vietā. Pēdējo
gadu laikā Latvijas rādītāji ir nemainīgi, kas norāda uz to ka Latvijas attīstība DESI
ir līdzvērtīga citām \acrshort{es} valstīm. Latvija ir sasniegusi lielus panākumus 
Digitālo publisko pakalpojumu rādītājā, Lielākās problēmas jau vairākus gadus sagādā
Cilvēkkapitāls un Digitālo tehnoloģiju intergrācijas rādītāji.
\paragraph{}
Kopumā Latvija pieder pie to valstu grupas, kuru rezultāti ir vidēji.
\paragraph{}
2013. gadā valdība pieņēma Informācijas sabiedrības attīstības pamatnostādnes 2014.–
2020. gadam – pašreizējo nacionālo digitalizācijas stratēģiju. Pamatnostādnes balstītas uz
septiņiem pīlāriem: IKT izglītība un prasmes, plaši pieejama piekļuve internetam, moderna
un efektīva valsts pārvalde, sabiedrībai pieejami e-pakalpojumi un digitālais saturs,
pārrobežu sadarbība digitālā satura vienotajā tirgū, pētniecība un inovācija IKT jomā,
uzticēšanās un drošība\cite{desi_Latvija, soc_dev}.

    \phantomsection
\desigraph{desi_savienojamiba}{1. Savienojamība}
    {(2014,11.3305)(2015,11.3951)(2016,15.0115)(2017,15.4357)(2018,16.4)}
    {(2014,09.6959)(2015,11.6532)(2016,14.3374)(2017,15.3726)(2018,16.2237)}
    {(2014,11.7685)(2015,12.6421)(2016,14.0537)(2017,15.5321)(2018,16.0279)}
    {(2014,11.0332)(2015,12.4053)(2016,13.4780)(2017,14.6234)(2018,15.6445)}

Savienojamības rādītājs veidojās no 5 rādītājiem, 1a - piekļuve platjoslas internetam (20\%)(>30Mbps),
1b - piekļuve mobīlajam platjoslas internetam (30\%)(>30Mbps), 1c - piekļuve ātrdarbīgajam platojoslas internetam (20\%)(<100Mbps),
1d - īpāsi ātrdarbīgā interneta piekļuve (20\%)(>100Mbps), 1e - platjoslas interneta cena (10\%)
\par
Latvija kopumā ir labāki rādītāji nekā Eiropā. Uz doto brīdi 1a rādītāja Latvija atpaliek
no Eiropas rādītājiem. Tas ir izskaidrojams ar to ka jau vairāk nekā 93\% iedzīvotājiem
ir piekļuve pie platojoslas interneta, pārējie 7\% ir ārpus pilsētām, privātmājas, lauku mājas utml,
tādas vietas, uz kurām nav izdevīgi veikt tiešo pieslēgumu, jo Latvijā ir ļoti labs
4G mobilā tīkla pārklājums, kā arī Latvijas Mobīlais telefons piedalās aktīvā 5G tīkla
attīstībā. LĪdz ar to 1b rādītāji Latvijai ir salīdzinoši labi, taču ja vērtēšana būtu
striktāka, tad Latvija noteikti izceltos uz Eiropas līmeņa. Latvijai ir labākais rādītājs
atrdarbīgā un īpaši ātrdarbīgā platjoslas interneta rādītāja (1c un 1d). Apsteidzot gan kaimiņus,
gan Eiropu kopumā. Daļēji tam palīdz iedzīvotāju koncetrēšanos ap vienu lielo centru - 
Rīgas pilsētu, taču kopējie rādītāji visā valstī ir salīdzinoši labi.

\par
“Vidējās jūdzes projekts”, kas tika sākts 2012. gadā un kam piešķirts līdzfinansējums
no ES struktūrfondiem, lai lauku teritorijas savienotu ar valsts pamatinfrastruktūru,
tagad nonācis otrajā posmā. Plānots, ka otrā posma būvdarbi sāksies 2018. gada
pavasarī. Galvenokārt tie tiks veikti atlikušajās “baltajās” teritorijās (2014.–2015. gadā
apzināta 221 teritorija). Paredzēts, ka līdz 2020. gadam optiskie kabeļi tiks ierīkoti
2800 km garumā un izveidoti aptuveni 220 optiskā tīkla piekļuves punkti. Pēc tam
telesakaru operatoriem, izmantojot jauno tīklu, kas ļaus galalietotājiem piedāvāt
mazumtirdzniecības pakalpojumus, būs iespēja izveidot vietējās sakaru līnijas ar datu
pārraides ātrumu vismaz 30 Mbit/s (“pēdējā jūdze”). Tomēr šķiet, ka ne visur tiek
veiktas privātās investīcijas “pēdējās jūdzes” infrastruktūras izbūvē. Ir vajadzīgi
turpmāki centieni, piemēram, papildu valsts atbalsta shēmas un regulatīvi pasākumi,
lai novērtētu situāciju un piedāvātu risinājumus, kas attiecīgajos gadījumos ļautu
novērst ar “pēdējo jūdzi” saistīto plaisu. Iespēja mājās, pieslēdzoties no mobilajām
ierīcēm, izmantot mobilo operatoru nodrošinātos fiksētos sakaru pakalpojumus,
palīdz pārvarēt šo plaisu atsevišķos lauku apvidos, kur netiek veiktas investīcijas
“pēdējās jūdzes” savienojumos\cite{platjosla}
\par
Lai novērstu izveidojošos plaisu piekļuvē starp pilsētām un laukiem tika 
pieņemta direktīva par Platjoslas izmaksu samazināšanu. Kā arī liels uzvasrs tiek
likts uz 5G tīklu un viņa iespējām tieši laukos, kur nav tik daudz citu objektu, kuri 
bloķētu signāla pārraidi. Attiecīgi antenu skaitam būtu jābūt mazākam, nekā tas ir
nepieciešams pilsētās.


    \desigraph{desi_cilvekkapitals}{2. Cilvēkkapitāls}
  {(2014,9.8611)(2015,10.2967)(2016,10.7013)(2017,11.0141)(2018,10.9601)}
  {(2014,11.7685)(2015,12.6421)(2016,14.0537)(2017,15.5321)(2018,16.0279)}
  {(2014,13.2348)(2015,14.1657)(2016,14.5055)(2017,14.4909)(2018,15.3438)}
  {(2014,12.2822)(2015,12.8878)(2016,13.1634)(2017,13.6546)(2018,14.1207)}

Definīcija: 2a Vienkāršo iemānas un to izantošana, 2b - padziļinātās digitālās prasmes un to izmantošana.
2b1 - IKT specialistu skaits pret iedz. skaitu, 2b2 - STEM universitātes beidzēji


Cilvēkkapitāla aspektā Latvija atpaliek no ES vidējās vērtības, un pēdējā gadā progress nav
panākts. Interneta lietotāju īpatsvars iedzīvotāju vidū gandrīz atbilst ES vidējam rādītājam,
tomēr 52 \% Latvijas iedzīvotāju joprojām trūkst digitālo pamatprasmju, kas tiem liedz efektīvi
lietot internetu, turklāt 19 \% digitālo prasmju vispār nav (par 2 punktiem vairāk nekā ES
vidējais rādītājs).
\par
Latvijā digitālo prasmju līmenis sieviešu vidū ir nedaudz augstāks nekā vīriešiem. Sieviešu
vidū vismaz 50 \% ir digitālās pamatprasmes, taču vīriešiem tie ir tikai 46 \%. Atšķirīgs ir arī
strādājošo un nestrādājošo iedzīvotāju digitālo prasmju līmenis. No strādājošajiem 57 \%
digitālās prasmes ir pamata vai augstākā līmenī, savukārt nestrādājošo vidū šis rādītājs ir
tikai 33 \%. Arī izglītības līmenis ir svarīgs faktors saistībā ar digitālo prasmju apguvi. No tiem,
kas ieguvuši augstāko izglītību, 76 \% ir vismaz digitālās pamatprasmes (ES līmenī tie ir
84 \%), taču pamatizglītību vai vidēja līmeņa izglītību ieguvušo vidū šis īpatsvars ir tikai 35 \%.
Mazizglītotiem cilvēkiem šis rādītājs ir par 5 \% augstāks nekā ES vidējais, savukārt vidēji
izglītotiem cilvēkiem šī starpība salīdzinājumā ar ES vidējo rādītāju ir 20 punkti. IKT
speciālistu skaits ir stabils, taču ievērojami zem ES vidējā līmeņa. Turklāt pēdējos gados ir
samazinājies absolventu skaits STEM jomā (2013. gadā 14,1, bet 2016. gadā tikai 12,7 uz
1000 iedzīvotājiem).
\par
Izglītības attīstības pamatnostādnes 2014.–2020. gadam ietver rīcības virzienus, kas skar
IKT izmantošanu mācību procesā un digitālo prasmju pilnveidošanu. Dokumentā
“Informācijas sabiedrības attīstības pamatnostādnes 2014.–2020. gadam” zem pīlāra “IKT
izglītība un e-prasmes” ir paredzētas šādas darbības izglītības jomā: sabiedrības informētība
un gatavība izmantot e-iespējas, iedzīvotāju un uzņēmēju e-prasmju pilnveidošana, IKT
prasmju palielināšana valsts pārvaldē, IKT speciālistu un darbinieku sagatavošana atbilstīgi
darba tirgus vajadzībām, kā arī algoritmiskās domāšanas un informācijpratības palielināšana
izglītības programmās. Šīm darbībām tiek piešķirts valsts finansējums, kā arī ES finanšu
atbalsts.
\par
Turklāt Latvijā ir izveidota sava digitālo prasmju un darbvietu koalīcija, kurā iesaistītas
vairākas ministrijas, IKT nozares apvienības un uzņēmumi, kā arī Latvijas Tirdzniecības un
rūpniecības kamera. Koalīcijas darbu koordinē Latvijas Informācijas un komunikācijas
tehnoloģijas asociācija (LIKTA). Koalīcijas darbā ir noteikti prioritārie virzieni, kas definēti
iepriekš minētajos dokumentos un kas orientēti uz šādiem mērķiem: nodrošināt IKT
apmācību atbilstīgi darba tirgus vajadzībām, iesaistīt jauniešus IKT jomā, izveidot
mūsdienīgus un interaktīvus mācību procesus, palielināt informētību par digitālās pratības un
IKT prasmju nozīmīgumu.
\par
Pagājušajā gadā ir veikti vairāki pasākumi, lai īstenotu šīs stratēģijas, piemēram, projekti
“MVU apmācības inovāciju un digitālo tehnoloģiju attīstībai Latvijā” un “IKT profesionāļu
apmācības inovāciju veicināšanai un nozares attīstībai”. Šo projektu mērķis: sniedzot
iespēju apgūt nākotnes digitālajās darbvietās vajadzīgās IKT prasmes, atbalstīt jauniešu
nodarbināmību un personisko izaugsmi. Mērķis ir laikposmā no 2017. līdz 2020. gadam
noorganizēt augstas kvalitātes digitālo prasmju kursus 7000 MVU darbiniekiem un 1500 IKT
speciālistiem. 2017. gada oktobra beigās pirmajā projektā bija iesaistījušies jau vairāk nekā
400 uzņēmumi un no 7000 iecerētajām apmācībām bija noorganizētas vairāk nekā 900. Ap
šo pašu laiku 55 IKT uzņēmumi bija iesaistījušies otrajā projektā un 196 augsta līmeņa
specializētos IKT apmācību kursos savas prasmes un kvalifikācijas bija atjauninājuši 780 IKT
speciālistu.
\par
Šajā jomā ir veikti daudzsološi pasākumi, un var paiet noteikts laiks, kamēr būs jūtama to
ietekme, taču Latvijai vēl ir jāstrādā pie tā, lai uzlabotu iedzīvotāju un darbaspēka digitālās
prasmes, tādējādi labāk sagatavojoties savas tautsaimniecības un iedzīvotāju digitālajai
pārejai
    Pagājušajā gadā saistībā ar digitālo tehnoloģiju integrāciju uzņēmumos Latvija ir guvusi
labus panākumus, no 25. vietas 2017. gadā pakāpjoties uz 23. vietu. Tomēr šajā jomā tā
joprojām atpaliek no lielākās daļas ES valstu. Situācijas uzlabošanos ietekmējuši uzņēmumi,
kas iegādājušies mākoņdatošanas pakalpojumus (pagājušajā gadā šis apjoms gandrīz
dubultojies, tagad sasniedzot 9,4 \%), un uzņēmumi, kas pieņēmuši elektronisku informācijas
koplietošanu. Par 2,5 procentpunktiem, sasniedzot 10,6 \%, palielinājies arī to MVU īpatsvars,
kas izmanto elektroniskos pārdošanas kanālus, tādējādi samazinot atšķirību no ES vidējā
rādītāja (17 \%). Nedaudz palielinājies arī to MVU apgrozījums, kuri nodarbojas ar ekomerciju 
(+0,5 procentpunkti, sasniedzot 8,6 \%). Tomēr varētu panākt vēl dažus
uzlabojumus, jo patlaban salīdzinoši maz uzņēmumu nodarbojas ar pārdošanu tiešsaistē pāri
robežām (4,7 \%). Augstās piegādes izmaksas ir galvenais šķērslis, ar ko nākas saskarties
uzņēmumiem, kuri vēlas tiešsaistē pārdot preces klientiem citās ES valstīs.
\par
Latvija nav izstrādājusi visaptverošu stratēģiju uzņēmumu digitalizācijai. Tomēr ir
sagatavotas vairākas iniciatīvas, kas sekmē “Rūpniecība 4.0” (Industry 4.0) izveidi; kā
piemēri minami izmēģinājuma projekts inženiertehniskajā nozarē, kas veicina izpratni par
koncepciju “Rūpniecība 4.0”, līdzdalība Interreg projektā “DIGINNO”, kurā iecerēts paātrināt
rūpniecības digitalizāciju Baltijas jūras reģionā, kā arī Interreg projekts “SKILLS+”, kura
mērķis ir veicināt tādu valsts politiku, kas sekmē IKT prasmju apgūšanu MVU vidū lauku
apvidos.
\par
Digitālās ekonomikas un sabiedrības indekss 2018, ziņojums par Latviju. lpp. 9 no 11
Tehnoloģiju pārneses programmas ietvaros paredzēts arī atbalsts inovācijas kuponu
izmantošanai. Inovācijas kuponu mērķis ir atbalstīt inovācijas darbības MVU vidē, sniedzot
tiem atbalstu pētniecības un izstrādes ārpakalpojumu izmantošanai, kas tiem ļautu ieviest
jaunus vai būtiski uzlabotus produktus vai tehnoloģijas.
Visaptverošas stratēģijas pieņemšana varētu palīdzēt uzlabot digitālo pāreju
tautsaimniecībā, piemēram, MVU un iedzīvotājiem nodrošinot plašāku piekļuvi daudz
lielākam tirgum.
    \input{chapters/2analize/desi/publiskie}
    Pēdējā gada laikā digitālo publisko pakalpojumu jomā Latvijai ir izdevies būtiski uzlabot
rezultātu (+13 procentpunkti) un pakāpties no 14. uz 9. vietu. Šī pozitīvā tendence
skaidrojama ar e-pārvaldes pakalpojumu plašāku izmantošanu (+8 procentpunkti), vairāk
izmantotām automātiski daļēji aizpildītām veidlapām (+13 procentpunkti) un jo īpaši atvērto
datu pieejamību (+53 procentpunkti). Atvērto datu izmantošanu sekmējusi Latvijas Atvērto
datu portāla atklāšana, jo tādējādi nodrošināta tieša piekļuve valsts pārvaldes datu kopām
un metadatiem un iespēja savienot tās ar citām datu kopām, kas publicētas citos valsts
pārvaldes portālos. Salīdzinājumā ar iepriekšējo gadu šādā veidā ievērojami uzlabojies
valsts sniegums atvērto datu jomā, un tagad Latvija ierindojas 18. vietā ES.
\paragraph{}
E-pārvaldes politika galvenokārt ir izklāstīta dokumentā “Informācijas sabiedrības attīstības
pamatnostādnes 2014.–2020. gadam”, kur īpaša uzmanība ir veltīta atvērto datu principu
īstenošanai valsts pārvaldē un publisko pakalpojumu sniegšanas vienkāršošanai, kas
iespējama, pateicoties efektīviem un lietderīgiem e-pakalpojumiem un sadarbspējīgām
Digitālās ekonomikas un sabiedrības indekss 2018, ziņojums par Latviju. lpp. 11 no 11
informācijas sistēmām. Pīlārā “Sabiedrībai pieejami e-pakalpojumi un digitālais saturs” ir
ietverti šādi elementi: valsts pārvaldes datu un darījumu atvēršana citiem lietotājiem;
kopīgotas platformas un pakalpojumu izstrāde publisko pakalpojumu sniegšanai; tādu
oficiālo e-pastu adrešu izveide, kuras saziņai var izmantot iedzīvotāji un uzņēmēji; publisko
pakalpojumu digitalizācija; elektronisko rēķinu automatizēta izdošana un pieņemšana;
kultūras mantojuma digitalizācija un pieejamība; latviešu valodas lietojuma veicināšana
digitālajā vidē; e-veselības risinājumi efektīvai, drošai un uz pacientiem orientētai veselības
aprūpei. Pie pīlāra “Mūsdienīga un efektīva valsts pārvalde” ietvaros veiktajiem pasākumiem
jāmin valsts pārvaldes pamatdarbības modernizācija; publiskā e-līdzdalība un e-demokrātija;
vienota valsts pārvaldes datu telpa un IKT infrastruktūru optimizācija.
\paragraph{}
2018. gada februārī Ministru kabinets pieņēma informatīvo ziņojumu “Mākoņdatošanas
pakalpojumu izmantošana valsts pārvaldē”, kurā uzmanība vērsta uz mākoņdatošanas
pakalpojumu potenciālu valsts pārvaldes efektivitātes nodrošināšanā. Paziņojumā ierosināts
rīcības plāns nolūkā sagatavoties mākoņdatošanas pakalpojumu efektīvai izmantošanai
valsts pārvaldē, turklāt tajā iekļauti priekšlikumi par mākoņdatošanas pakalpojumu atsevišķu
vadības funkciju centralizāciju.
\paragraph{}
Paredzams, ka, samazinot administratīvo slogu, Latvijā tiks izveidota labvēlīgāka
uzņēmējdarbības vide un palielināsies to uzņēmumu (jo īpaši MVU) skaits, kuriem līdz šim
bijis grūtāk sākt savu uzņēmējdarbību vai oficiāli reģistrēties sarežģīto un apgrūtinošo
birokrātisko procedūru dēļ.
\par
Kā iezīmē iepriekšējā nodaļa Latvija ir salīdzinoši zemā vietā kopējā Eiropas līmenī.
Kamēr dažos rādītājos sasniegumi ir pat ļoti spoži, citi, svarīgāki, velk Latvijas kopējo
indeksu uz leju. Dotajā apakšnodaļā autors cenšas noteikt kādas iniciatīvas Latvijā jau
tika uzsāktas ar mērķi uzlabot šos rādītājus.
\par
Turpinājumā tiek apskatīti dažādi pieejamie kursi un izglītības materiāli un to cenas,
kā arī tiek apskatīta Skola 2030 programma.
    \input{chapters/2analize/desi/izpete}
\section{Digitālās prasmju uzlabošanas iniciatīvas Latvijā}
Kā iezīmē iepriekšējā nodaļa Latvija ir salīdzinoši zemā vietā kopējā Eiropas līmenī.
Kamēr dažos rādītājos sasniegumi ir pat ļoti spoži, citi, svarīgāki, velk Latvijas kopējo
indeksu uz leju. Dotajā apakšnodaļā autors cenšas noteikt kādas iniciatīvas Latvijā jau
tika uzsāktas ar mērķi uzlabot šos rādītājus.
\par
Turpinājumā tiek apskatīti dažādi pieejamie kursi un izglītības materiāli un to cenas,
kā arī tiek apskatīta Skola 2030 programma.

    \subsection{Darba Tirgus analīze ITK nozarē}
    \subsection{Izglītības analīze ITK nozarē}
\section{Accenture Latvija izglītības projekti}
Accenture Latvija jau vairākus gadus izjūt kvalificēta darbaspēka trūkumu valstī. Šis fakts jau bija pamanīts
2014 gadā. Kā uzņēmums kurš ir ieinteresēts augt un attīstīties jau tajā laikā tika izveidots fonds Latvijas
skolniekiem - Start(it). Otrs projekts, kurš ļoti veiksmīgi darbojās jau vairāk nekā 10 gadus ir "Bootcamp" programma.
\par
Tālāk tiek veikta Start(it) un Bootcamp padziļināta analīze. Autors apkopo kopējo ietekmi uz digitālo
prasmju apgūšanu valstī šo projektu iespaidā. Tiek apkopoti statistiskie rādītāji. Tie nav tieši salīdzināmi,
taču dod priekšstatu par šo projektu efektivitāti.

    \subsection{Bootcamp rekrutēšanas programmas analīze}
    \subsection{Start(it) fonda analīze}
