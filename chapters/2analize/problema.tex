\section{Problēmas noteikšana un tās analīze}
Veicot datu analīzi no Eiropas Savienibas un Latvijas datiem, apskatot eksistējošo klāstu ar kursiem,
apskatot Accenture Latvia eksistējošās iniciatīvas, kā arī apkopojot ekspertu interviju rezultātus. 
autors kā \textbf{pamatproblēmu} izvirza - \textbf{Padziļināto digitālo prasmju apmācību trūkumu Latvijā}.
\clearpage
\begin{figure}[h]
    \centering
    \caption{Pamatproblēmas analīze ar prāta kartes metodi}
    \author{Autora veidots}
    \includegraphics[width=\textwidth]{images/zirneklis.PNG}
    \label{fig:zirneklis}
\end{figure}
Turpinājumā izmantojot prāta kartes metodi tiek noteikti dažādi cēloņi, kuru rezultātā veidojās situācija,
ka Latvijas padziļināto digitālo prasmju apmācībās veidojās iztrūkums. Šos cēloņus var redzēt \ref{fig:zirneklis}
attēlā. Turpinājumā šie cēloņi tiek analizēti un no tiem viens tiek izvirzīts par projekta konkrētu problēmu.
\begin{enumerate}
    \item \textbf{Trūkst izglītības materiālu latviešu valodā}.
Pasaules globālais tīmeklis ir pilns ar dažādiem materiāliem kuri palīdz apgūt dažādas programmēšanas iemaņas,
datu apstrādi, drošību internetā u.c., taču šie materiāli pārsvarā ir angļu valodā. To izprašana, it sevišķi 
vecuma grupās 40+, ir ļoti sarežģīta. Cilvēkam ne tikai ir jāapgūst jauna viela, kura ir pietiekoši sarežģīta,
bet arī jāmācās vēl viena valoda, vai ir pamatīgi jāuzlabo tās zināšanas. Latvijā ir pieejami tikai daži 
atvērtie resursi, kuri palīdz apgūt šo tematu. Savukārt tie kuri ir pieejami par maksu, ir diezgan ārpus cilvēku
iespēju robežām. Biež vien tie notiek tikai Rīgā, kas samazina potenciālo auditoriju. Otrs ietekmējošais faktors
ir ar cik lielu ātrumu attīstās informāciju tehnoloģijas. Bieži vien pēc gada, vai diviem pieejami informācija
ir novecojusi tik tālu ka vairs nav izmantojama. Līdz ar to pat eksistējošie mācību materiāli Latviešu valodā
diezgan ātri noveco.
    \item \textbf{Mācību materiālu šaurais spektrs un to pieejamība}.
Lielākā daļa mācību kursu, kuri ir pieejami Latvijā, ir par samaksu, vairums no tiem arī prasa samērā lielas
summas, pretī piedāvājot ļoti šauru un specializētu kursu apmācību. Ir ļoti maz piedāvājumu apgūt datorzinātnes,
vai datorikas pamatus, kuri palīdzētu vēlāk izmantot jebkuru informācijas tehnoloģiju rīku, programmu utt.
Vienīgais pieejamais variants kā iegūt nepieciešamās pamata zināšanas būtu studēt kādā no Latvijas universitātēm
tādā veidā apgūstot ne tikai vienu konkrētu programmēšanas valodu, bet gan teoriju un zinātnes pamatus.
    \item \textbf{Valsts līmenī nav nepieciešamais atblasts izglītības programmai}.
Kaut arī tika izstradāti vairāki dokumenti, lai sekmētu ITK attīstību, uzlabotu DESI rādītājus un kopumā uzlabotu
Latvijas vidējo dzīves līmeni, bieži vien šie projekti netiek pienācīgi atbalstīti. Viena no pozitīvām lietām
ir atvērto datu iniciatīva, kura jau ietekmējusi Latvijas pozīcijas DESI līmenī un ļauj vietējiem uzņēmējiem
viedot jaunus pakalpojumus, kuri pirms tam nebija pieejami, tādā veidā veicinot ekonomisko attīstību. Taču ir
nepieciešams daudz nopietnāka iesaistīšanas izglītības jomā, lai šo cēloni varētu novērst. Apskatoties vairāku
ministriju dokumentus var atrast dažādas iniciatīvas kā veicināt digitālo prasmju izglītību, taču šīs iniciatīvas
neiegūst vajadzīgo uzmanību un publiski par to netiek diskutēts.
    \item \textbf{Trūkst kvalificētu skolotāju, kuri spētu apmācīt cilvēkus}.
Start(it) fonda ietvaros tika noskaidrots, ka programmai Skola2030 paši skolotāji arī nav gatavi. Skolotāji nebūs
spējīgi pasniegt datoriku skolās, jo viņiem pašiem trūkst zināšanu par doto priekšmetu. Valstī nekad nav bijusi 
datorikas apmācība skolās. Skolotāji to varēja apgūt tikai paši savā laikā un intereses dēļ. Lai novērstu
doto cēloni vajadzētu piedāvāt skolotāju apmācības un palīdzēt viņiem sagatavoties pasniegt programmēšanu skolās.
Skatoties uz privāto sektoru, kā jau minēts, daudzi kursi koncentrējas uz konkrētās programmēšanas apmācībām, nevis
uz datorikas un datorizinātnes izskaidrošanu. Tā rezultātā cilvēks iemācās kā rakstīt pareizi, nevis zinot
pareizrakstības noteikumus, bet gan iegaumējot visus iespējamos variantus.
\end{enumerate}
\par
Izvērtējot pamatproblēmu un tās cēloņus, var secināt ka Latvijas valstij ir nopietni jāuzlabo datorprasmju 
pieejamība. Tā kā vienmēr ir labāk labot problēmas sākotnējo cēloni nevis tā sekas, autors
kā \textbf{konkrēto problēmu} izvirza - \textbf{Mācību materiālu šaurais spektrs un to pieejamība}.
Šī cēloņa novēršana ļautu uzlabot situāciju arī ar citiem cēloņiem, kaut arī tas varētu aizņemt vairākus gadus.
\par
Maģistra darba 2. nodālā tiek izveidots projekta priekšlikums, kurš sastāv no mērķu apraksta,
vēlāk tiek izvirzītas alternatīvas šīs problēmas risināšanai un pēcāk šo alterantīvu izvērtēšana.
Otrajā posmā tiek veikta izpēte un tiek atrasts labākais risinājums dotajā problēmā