\section{Problēmas noteikšana un tās analīze}
Veicot datu analīzi no Eiropas Savienibas un Latvijas datiem, kā arī apkopojot ekspertu interviju rezultātus
autors par \textbf{pamatproblēmu} Latvijas ITK industrijas attīstību izvirza - 
\textbf{Padziļināto digitālo prasmju apmācību trūkums}. 
\paragraph{}
Turpinājumā izmantojot prāta kartes metodi tiek noteikti dažādi cēloņi, kuri rezultāts izvēršas kā Latvijas
iedzīvotājiem trūkst digitālo prasmju; Šie cēloņi ir:
\begin{enumerate}
    \item \textbf{Trūkst izglītības materiālu latviešu valodā}.
Pasaules globālais tīmeklis ir pilns ar dažādiem materiāliem kuri palīdz apgūt dažādas programmēšanas iemaņas,
datu apstrādi, drošību internetā u.c., taču šie materiāli pārsvarā ir angļu valodā. To izprašana, it sevišķi 
vecuma grupās 40+, ir ļoti sarežģīta. Cilvēkam ne tikai ir jāapgūst jauna viela, kura ir pietiekoši sarežģīta,
bet arī jāmācās vēl viena valoda, vai ir pamatīgi jāuzlabo tās zināšanas. Latvijā ir pieejami tikai daži 
atvērtie resursi, kuri palīdz apgūt šo tematu. Savukārt tie kuri ir pieejami pa maksu, ir diezgan ārpus cilvēku
iespēju robežām. Biež vien tie notiek tikai Rīgā, kas samazina potenciālo auditoriju.
    \item \textbf{Cilvēki netic savām spējām apgūt vajadzīgās zināšanas}.
Cilvēki bieži vien domā, ka programmēšana prasa ļoti augstas matemātikas zināšanas, ka viņi nespēs to apgūt un
viņiem pat nav vērts mēģināt. Otrs faktors ir skolās netiek pietiekoši daudz stāstīts par iespēju darboties
dotajā sfērā meitenēm. Tradicionālais uzskats ir tāds, ka sievietes mācās sociālās zinības un viņas nespēs
apgūt vajadzīgās zināšanas. Taču pirmā programmētāja bija sieviete, kā arī ir vairākas pasaules mēroga IT jomas
dalībnieces kuras pierāda pretējo. Ieviešot programmēšanu skolās un parādīt jauniešiem ko viņi spēj izdarīt
pāris dienās ar datora palīdzību noteikti tos iedvesmotu.
    \item \textbf{Valsts līmenī nav nepieciešamais atblasts izglītības programmai}.
Kaut arī tika izstradāti vairāki dokumenti, lai sekmētu ITK attīstību, uzlabotu DESI rādītājus un kopumā uzlabotu
Latvijas vidējo dzīves līmeni, bieži vien šie projekti netiek pienācīgi atbalstīti. Viena no pozitīvām lietām
ir atvērto datu iniciatīva, kura jau ietekmējusi Latvijas pozīcijas DESI līmenī un ļauj vietējiem uzņēmējiem
viedot jaunus pakalpojumus, kuri pirms tam nebija pieejami, tādā veidā veicinot ekonomisko attīstību. Taču ir
nepieciešams daudz nopietnāka iesaistīšanas izglītības jomā, lai šo cēloni varētu novērst.
    \item \textbf{Trūkst kvalificētu skolotāju, kuri spētu apmācīt cilvēkus}.
Start(it) fonda ietvaros tika noskaidrots, ka programmai Skola2030 paši skolotāji nav gatavi. Viņi nebūs spējīgi
pasniegt programmēšanu skolās, jo viņiem pašiem trūkst zināšanu par doto praksi. Valstī nekad nav bijusi 
programmēšanas apmācība skolās. Skolotāji to varēja apgūt tikai paši savā laikā un intereses dēļ. Lai novērstu
doto cēloni vajadzētu piedāvāt skolotāju apmācības un palīdzēt viņiem sagatavoties pasniegt programmēšanu skolās.
\end{enumerate}
\paragraph{}
Izvērtējot pamatproblēmu un tās cēloņus, var secināt ka Latvijas valstij ir nopietni jāuzlabo datorprasmju 
pieejamība, it sevišķi fokusējoties uz skolām. Līdz ar to autors par \textbf{konkrēto problēmu} izvirza -
\textbf{Izglītības satura trūkums Latvijas skolām}.
\paragraph{}
Pastāstīt par konkrēto risinājumu - Start(it) un Skola2030
\paragraph{}
Nedaudz vairāk par Start(it) dotajā brīdī
\paragraph{}
Nedaudz vairāk arī par Accenture Latvia
\paragraph{}
Maģistra darba 2. nodālā tiek izveidots projekta priekšlikums, kurš sastāv no mērķu apraksta,
vēlāk tiek izvirzītas alternatīvas šīs problēmas risināšanai un pēcāk šo alterantīvu izvērtēšana.
Otrajā posmā tiek veikta izpēte un tiek atrasts labākais risinājums dotajā problēmā