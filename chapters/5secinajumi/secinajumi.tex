\chapter*{Secinājumi un priekšlikumi}
\addcontentsline{toc}{chapter}{Secinājumi un priekšlikumi} %adds unnumbered section to table of contents
\textbf{Secinājumi}
\begin{enumerate}
  \item Kaut arī Latvijā cenšas prioretizēt nozaru izaugsmi ar augsto pievienoto vērtību,
  tas nenotiek tik strauj cik gribētos
  \item Latvijā kopumā trūkst digitālo prasmju apmācību spektra un pieejamības
  \item Latvijas skolās tiek pasniegta pārsvarā tikai informātika, datorika ir liels retums
  \item Skolotājiem trūkst zināšanu, lai apmācītu datoriku skolās
  \item Pat neliels ieguldījums apmācību kursos spēj nest lielu pienesumu Latvijas ekonomikai kopumā
  \item Start(IT) fonda tīmekļa vietnes pielāgošana Skola 2030 programmai aizņemtu 190 dienas un
  prasītu 476 569.01 € lielu finansējumu.
\end{enumerate}
\textbf{Priekšlikumi}
\begin{enumerate}
  \item Īstenot darba autora piedāvāto projekta priekšlikumu, lai izveidotu jaunus mācību
  mācību materiālus Latvijas skolām
  \item Izvēloties projekta vadītāju būtu nepieciešams algot cilvēku, kuram ir pieredze gan
  ar IT jomu, gan ar pasniegšanu, gan ar projekta vadību, nepieciešamā izglītība projektu vadībā
  \item Projekta organizēšanai izmantot stingrās matricas organizatorisko formu.
  \item Autora izstrādāto rokasgrāmatu izmantot projekta īstenošanai.
\end{enumerate}
