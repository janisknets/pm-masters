\section{Projekta alternatīvas izvirzīšana un to sākotnējais izvērtējums}
Izmantojot izvirzītos mērķus autors izveidoja sarakstu ar alternatīvām, kuras varētu sasniegt vajadzīgos
mērķus.
\paragraph{}
Alternatīvu izvēli ietekmēja datu analīze no pirmās nodaļas.
\paragraph{}
Izvēlētās alternatīvas ir:
\renewcommand{\labelenumi}{\Alph{enumi}}
\begin{enumerate}
    \item alternatīva - Fonda restrukturizācija ar fokusu uz jaunu partenru meklēšanu
    \item alternatīva - Izglītības materiālu uzlabošana sadarbībā ar izglītības sektoru
    \item alternatīva - Izveidot materiālus mūžizglītības programmai
\end{enumerate}
\renewcommand{\labelenumi}{\arabic{enumi}}
\paragraph{}
A alternatīva uzver fokusu uz citu partneru meklēšanu, kuriem jau būtu vajadzīgās iemaņas un vēlme
sadarboties ar Start(it) projektu. Tādā veidā tiktu iegūti jauni dalībnieki projektā un viņi palīdzētu
izveidot jaunos materiālus. Iespējams šiem dalībniekiem jau būtu materiāli, kuru iederētos gan Skola2030
programmā, gan tālākizglītības programmās. Projekts fokusētos uz tieši piesaisti un popularizēšanu,
tādā veidā palielinot interesi par sevi. Lielāks uzvars tiktu likts tieši uz attiecību veidošanu, jaunu
partneru meklēšanu. Jaunu materiālu izveide nebūtu iesaistīta dotajā projekta. Meklējot jaunus partnerus
vajadzētu atrast tādus uzņēmumus, kuriem šādi materiāli jau būtu izstrādi, vai uzņēmumi kuri būtu gatavi
izstrādāt šādus materiālus tuvākā gada laikā.
\paragraph{}
B alterntīva fokusējās vairāk uz materiālu pārstrādi ar jau eksistējošiem parteriem. Galvenais fokuss būtu
uz sadarbību ar jau zināmiem un pārbaudītiem cilvēkiem. Eksistējošie materiāli tiktu atjaunoti, ļoti liels
skaits jaunu materiāu tiktu pievienots. Izveidojot šos materiālus kopā ar skolotājiem no tām skolām, kuras
iesaistījās Start(it) projektā un veiksmīgi sagatavo šos kursus. Kādas no šim skolām tiktu arī izvēlētas kā 
pilotskolas Skola2030 programmai. Šo skolu skolotāji tiktu apmācīti un sagatavoti darbam ar jaunajiem materiāliem.
\paragraph{}
C alternatīva paredz materiālu pielāgošanu tālākizglītībai, tādā veidā iesaistot jaunu mērķauditoriju. Šie
cilvēki titku pielāgoti jaunajam darba tirgum, kas sniegtu popularitāti projektam kopumā. Materiāliem tā pat
būtu jābūt pielāgotiem, līdz ar to varētu arī tos sagatbot tā, lai tie varētu būt izmantoti vidusskolu datorikas
stundās.
\paragraph{}
Lai salīdzinātu izvēlētās alternatīvas, maģistra darba autors veic alternatīvu salīdzinājumu,
izvērtējot katras alternatīvas priekšrocības, trūkumus, izmaksas un riskus (skat. Pielikumu).
\paragraph{} 
ALternatīvu salīdzinājums parādīja ka alternatīvas B un C rada labākus rezultātus, līdz ar to alternatīva A tiek
izslēgta no turpmākās izpētes.
Pielikt klāt arī paskaidrojumu kāpēc A tika izslēgta
\paragraph{}
B un C alternatīvas pēc būtības veido līdzvērtīgas darbības, vienīgi gala rezultāts būs mērķēts uz dažādām
auditorijām. Lai noteiktu kura no šīm alternatīvām ir labāka būs nepieciešama nedaudz padziļinātāka analīze
\paragraph{}
B alternatīvas galvenā priekšrocība ir tāda, ka tā fokusējās uz fonda galveno mērķi kopš tā izveidošanas, proti,
attīstīt un popularizēt datorikas apgūšanu skolās. Tas arī ļauj izmantot jau eksistējošos sadarbības partnerus,
līdz ar to daudz ātrāk var uzsākt tiešo darbu un sasniegt vairākus mērķus.
\paragraph{}
C alternatīva ļauj piesaistīt jaunu mērķauditoriju, popularizētu fondu kopumā valstī.
\paragraph{}
B alternatīvas lielākie trūkums ir paildus fonda dalībnieku meklēšanas neesamība.
\paragraph{}
C alternatīvas trūkums ir visu vajadzīgo mērķu sasniegšanā varētu būt diezgan sarežģīta un tā neatbilst fonda
galvenajiem uzstādījumiem.
\paragraph{}
Runājor par riskiem, tad šeit daudz vienkāršāk ir izveidot materiālus pēc skolas Latvijas iedzīvotājiem, jo tiem
nav jāatbilst VISC standartiem. Attiecīgi B alternatīva prasīs lielākus pūliņus un potenciālus noraidījumus no
valsts puses. Kā arī ir riski no birokrātijas puses. Ir daudz vienkāršāk iesaistīties tālākizlgītības programmās, jo
tām ir vienkārši mazākas prasības nekā pamat un vidusskolām.
\paragraph{}
Tā kā alternatīvas prasīs pietiekoši līdzīgus darba resursus, bet gala rezultāts tomēr būs nedaudz dažāds, diezgan
svarīgi ir veikt pilnu analīzi par ieguldījumiem un saņemtajiem rezultātiem. Ko arī varēs redzēt nākošās sadaļās.
