\section{Projekta alternatīvas izvirzīšana un to sākotnējais izvērtējums}
Izmantojot izvirzītos mērķus autors izveidoja sarakstu ar alternatīvām, kuras varētu sasniegt vajadzīgos
mērķus.
\par
Alternatīvu izvēli ietekmēja datu analīze no pirmās nodaļas.
\par
Izvēlētās alternatīvas ir:
\renewcommand{\labelenumi}{\Alph{enumi}}
\begin{enumerate}
    \item alternatīva - Start(it) jaunu sadarbības partneru piesaistīšana
    \item alternatīva - Start(it) tīmekļa vietnes pielāgošana Skola 2030 projektam
    \item alternatīva - Start(it) tīmekļa vietnes pielāgošana tālākizglītības vajadzībām
\end{enumerate}
\renewcommand{\labelenumi}{\arabic{enumi}}
\par
A alternatīva par prioritāti izvirza jaunu fonda dalībnieku meklēšanu. Šie dalībniekiem vajadzētu būt
iemaņām, materiāliem, kuri ļautu sasniegt izvirzītos mērķus. Tādā veidā tiktu iegūti jauni dalībnieki,
kuri palīdzētu izveidot jaunos mācību materiālus. Iespējams šiem dalībniekiem jau būtu materiāli, kuri
iederētos gan Skola2030 programmā, gan tālākizglītības programmās. Lielāks uzvars tiktu likts tieši uz 
attiecību veidošanu gan IT nozarē, gan ārpus tās. Jaunu materiālu izveide nebūtu iesaistīta dotajā projekta.
Meklējot jaunus partnerus vajadzētu atrast tādus uzņēmumus, kuriem šādi materiāli jau būtu izstrādi, vai
uzņēmumi kuri būtu gatavi izstrādāt šādus materiālus tuvākā gada laikā.
\par
B alterntīva prioritāte ir turpināt sadarboties ar šī brīzā fonda partneriem un pašu spēkiem izveidot šos
izlgītības materiālus. Eksistējošie materiāli tiktu atjaunoti, tiku pievienoti arī jaunie materiāli tajās 
datorprasmju jomās kur tie iztrūkst. Izveidojot šos materiālus kopā ar skolotājiem no tām skolām, kuras
iesaistījās Start(it) projektā un veiksmīgi sagatavo šos kursus. Kādas no šim skolām tiktu arī izvēlētas kā 
pilotskolas Skola2030 programmai. Šo skolu skolotāji tiktu apmācīti un sagatavoti darbam ar jaunajiem materiāliem.
\par
C alternatīva paredz materiālu pielāgošanu tālākizglītībai, tādā veidā iesaistot jaunu mērķauditoriju. Šie
cilvēki titku pielāgoti jaunajam darba tirgum, kas sniegtu popularitāti projektam kopumā. Materiāliem tā pat
būtu jābūt pielāgotiem, līdz ar to varētu arī tos sagatbot tā, lai tie varētu būt izmantoti vidusskolu datorikas
stundās.
\par
Lai salīdzinātu izvēlētās alternatīvas, maģistra darba autors veic alternatīvu salīdzinājumu,
izvērtējot katras alternatīvas priekšrocības, trūkumus, izmaksas un riskus. Šo tabulu var apskatīt \ref{app:ABCsalidzinajums} pielikumā
\par 
Alternatīvu salīdzinājums parādīja ka alternatīvas B un C rada labākus rezultātus, līdz ar to alternatīva A tiek
izslēgta no turpmākās izpētes.
Pielikt klāt arī paskaidrojumu kāpēc A tika izslēgta
\par
B un C alternatīvas pēc būtības veido līdzvērtīgas darbības, vienīgi gala rezultāts būs mērķēts uz dažādām
auditorijām. Lai noteiktu kura no šīm alternatīvām ir labāka būs nepieciešama nedaudz padziļinātāka analīze
\par
B alternatīvas galvenā priekšrocība ir tās darbības joma, jo tā sakrīt fonda galveno mērķi kopš tā izveidošanas, proti,
attīstīt un popularizēt datorikas apgūšanu skolās. Tas arī ļauj izmantot jau eksistējošos sadarbības partnerus,
līdz ar to projekts varētu būt sekmīgāks, jo dotajā jomā ir jau pieredze un labs sadarbības partneru tīkls.
\par
C alternatīvas īstenošana ļautu piesaistīt jaunu mērķauditoriju, palielināt fonda atpazīstamību un ieinteresēt
apmacību kursu dalībniekus uzsākt darbu IT nozarē, iespējams pie kāda no fonda dalībniekiem.
\par
B alternatīvas trūkums ir tāds, ka fondam pašam ir jāveido mācību saturs, kas ir gan finansiāli izaicinošs 
uzdevums, gan aizņemt diezgan daudz laika. Pie tam jaunajiem materiāliem ir jāatbilst visām \acrshort{visc}
prasībām.
\par
C alternatīva sastopas ar tādām pašām problēmām kā B alternatīva - jāveido jauns saturs, taču C alternatīvas gadījumā
tas ir vēl sarežģītāk, jo eksistējošie materiāli Start(it) tīmekļa vietnē tika veidoti skolniekiem, līdz ar to
potenciālais izglītību saturs ir vēl lielāks.
\par
Runājor par riskiem, tad šeit daudz vienkāršāk ir izveidot materiālus mūžizlgītības iedzīvotājiem, jo tiem
nav jāatbilst \acrshort{visc} standartiem. Attiecīgi B alternatīva prasīs lielākus pūliņus un potenciālus noraidījumus no
valsts puses. Kā arī ir riski no birokrātijas puses, jo valsts iestādēm ir samērā lielāks birokrātiskais slogs. Ir daudz
vienkāršāk iesaistīties tālākizlgītības programmās, jo tām ir vienkārši mazākas prasības nekā skolām.
\par
Tā kā alternatīvas prasīs pietiekoši līdzīgus darba resursus un gala rezultāts atšķirsies tikai ar pašu mācību materiāla
satura veidu, diezgan svarīgi ir veikt pilnu analīzi par ieguldījumiem un saņemtajiem rezultātiem. Ko arī varēs redzēt
nākošās sadaļā, kura savstarpēji salidzina B un C alternatīvas konceptuāli, pēc finansiāliem radītājiem, riskiem un stratēģijas.
