\subsection{B un C alternatīvas izdevumu prognozēšana}
%TODO vajag apalabot attiecīgi pret izmainam ieguvumos un izdevumos
Izdeuvmu aprēķinu B un C alternatīvām var atrast \ref{app:B_izdevumi} un \ref{app:C_izdevumi} pielikumos attiecīgi.
\par
B alternatīvas lielākās izmaksas ir saistītas ar kursu viedošanu un to vadīšanu, jo ir nepieciešams īrēt telpas,
sagatvot tehniku, kā arī jānodrošina dažas ekstras, piemēram, kafija. Šīs apmācības ir ieplānotas uz vasaras mēnešiem,
šie mēneši ir izvēlēti jo tad skolotājiem būtu laiks, lai apmeklētu šādus kursus; Kā arī tajā laikā vienam no fonda 
dalībniekiem - RTU - varētu būt pieejamas telpas par izdevīgāku cenu. Izmaksās ir ierēķināts programmētāja un satura
konsultanta darba algas, lai nodrošinātu mājaslapas uzturēšanu un pastāvīgu konsultantu skolotājiem, ja tāds būtu
nepieciešams. Kopējās izmaksas gadā veido 37147.98 EUR
\par
C alternatīvas gadījumā nav nepieciešams uztraukties par apmācībām, vienīgais kas ir jānodrošina ir mājaslapas darbība,
līdz ar to tiek algoti divi darbinieki - programmētājs un satura konsultants. Kā arī tiek ierēķināti izdevumi par
mājaslapas uzturēšanu mākonī un grāmatvedības pakalpojumi. Kopējās izmaksas gadā ir 25336.20 EUR.
\par
Tā kā B alternatīvai ir nepieciešams arī apmācīt skolotājus un pasniegt tiem kursus klātienē, tad šie izdevumi veido
starpību starp abām alternatīvām. Pārējās izmaksas ir vienādas, jo satura uzturēšanai tiek nodrošināti divi cilvēki un
pakalpojuma uzturēšana būtiski neatšķiras. C alternatīva ir lētāka.
\par
Lai izveidoto mākoņpakalpojumu izmaksas tika izmantoti pieejami risinājumi no Amazon Web Services, Microsoft Azure un
Heroku. Algu apmēri tika rēķināti no vidējām algām attiecīgā amatā. Tā kā uzturēšanai nav nepieciešami augstas klases
specialisti, tad šiem aprēķiniem vajadzētu būt pareiziem.
\par
Nākošā apkašnodaļā tiek veikts abu alternatīvu vērtējums, kas palīdzēs izvēlēties labāku alternatīvu.
