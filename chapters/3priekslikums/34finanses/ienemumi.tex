\phantomsection
\subsection{B un C alternatīvas ieguvumu prognozēšana}
%TODO fix this text, tagad ir ieguvumi nevis ienemumi.
Autors izpētīja potenciālos ieņēmumus no B un C alternatīvām, tabulas var redzēt \ref{app:B_ienemumi} un \ref{app:C_ienemumi} pielikumā.
Tā kā B alternatīva ir labdarības projekts, attiecīgi pakalpojumi tiks sniegti bez jebkādas samaksas. Kā arī zināšanas kopumā
ir ļoti grūti izvērtēt attiecīgā vērtībā, tad autors pieņēma lēmumu izmantot simbolisku samaksu par pakalpojumu - 1 EUR vērtībā.
Savukārt C alternatīvas gadījumā tiek izmantota 50 EUR maksa par pakalpojumu, Latvijā līdzvērtīgi kursi nav pieejami, tuvākais kas
eksistē ir klātienē apmācības ar sākotnējo cenu no 300 EUR par konkrētiem kursiem.
\par
B alternatīva ir sadalīta 4 tabulās, pirmās trīs tabulas ir par pirmajiem trim gadiem, pēc tam, no trešā līdz desmitajam gadam,
tiek uzskatīts ka skolu un skolnieku skaits nemainīsies dažādu seociālo un ekonomikso procesu rezultātā. Pirmajā gadā skolnieku skaits
tiek aprēķināts izejot no 30 apmācītiem skolotājiem, reizinot to ar 20 - skolnieku skaits vienā klasē un reizinot ar klašu skaitu - 9.
Šie aprēķini ir aptuveni un vairāk ir pietuvināti zemākam slieksnim. Vēlāk katru gadu ieņēmumi pieaug, jo tiek apmācīti ar vien vairāk
skolotāju, kuri spēs pasniegt attiecīgo priekšmetu, līdz ar to arī skolnieku skaits aug proporcionāli. Sākot ar trešo gadu tiek paredzēts,
ka ieņēmumi stabilizēsies. Tas notiks dēļ skolotāju rotācijas, demogrāfiskās situācijas iespaidā, kā arī ne visas skolas gribēs izmantot
attiecīgo iespēju. Attiecīgi maksimālais skolu skaits, kuras izmantos attiecīgo programmu apstāsies pie 300, kas ir aptuveni 3/7 no visām
Latvijas skolām.
\par
C alternatīvas ieņēmumu aprēķins ir nedaudz vienkāršāks, jo tiek vienkārši izveidots pakalpojums un tas tiek piedāvāts kopējā tirgū.
Izvērtējot potenciālo konkurentu cenas, kā arī nepieciešamās izmaksas uzturēšanai tika atrasta pakalpojuma cena, kura būtu pietiekoši
pieejama patērētājam, nestu nepieciešamo peļņu projekta atmaksai, kā arī paliktu zemāka par konkurentu alternatīvām.
\par
Tiešā salīdzinājuma gadījumā var redzēt ka B alternatīva nes lielākus ieguvumu. Tas ir pateicioties tam, ka jau eksistē patērētāju tirgus,
kurš ir nepiesātināts un tiek piedāvāta vienīgā alternatīva, daļēji bez izvēles iespējām. C alternatīvas gadījumā ir jākonkurē ar citiem
pakalpojumu sniedzējiem, kā arī mērķauditorija ir daudz neprognozējamāka.
\par
Lai veiktu attiecīgo salīdzinājumu tika izmantoti dati no Centrālās statistikas pārvaldes datubāzes. No tās tika iegūti dati par skolām,
skolnieku skaitu, demogrāfijas ietekmi uz skolu un skolnieku skaitu. C alternatīvas ieņēmumu prognozei tika izmantoti dati no konkurentu
mājaslapām. Tika izpētīts viņu piedāvājums un satura apjoms.  
\par
Nākošā sadaļā tiks izpētīti B un C alternatīvas izdevumi, kuri atstāj ietekmi uz kopējo peļņu.
