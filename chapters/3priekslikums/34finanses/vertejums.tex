\subsection{B un C alternatīvas finansiālais izvērtējums}
\par
Alternatīvu vērtējumu ir atrodami \ref{app:B_finansialais_vertejums} un \ref{app:C_finansialais_vertejums} pielikumos.
Tajos apkopojti ieņēmumi, izdevemumi, naudas plūsma u.c. Kā arī tiek aprēķināts projekta izdevīgums izmantojot
tādus rādītājus kā \gls{irr} un \gls{roi}.
\par
Aprēķini tabula ir pamatoti ar aprēķiniem, kuri tika aprakstīti iepriekšējās apakšnodaļās. Sākotnēji
tiek paņemti prognozētie ieņēmumi desmit gadu griezumā.
\par 
Nākošā rinda sastāv no izdevumiem desmit gadu garumā. Šos abus skaitļus izmanto, lai iegūtu ikgadējo 
peļņu, no kuras vēlāk tiek atņemtas amortizācijas izmaksas, taču dotajam projektam tādu nav, līdz 
ar to šī aile paliek ar nulles vērtību.
\par 
No šiem datiem var aprēķināt ar nodokli apliekamos ienākumus un pašu nodokli, savukārt tas iedod
prognozējamo tīro peļņu pēc nodokļu nomaksas. Izmantojot šos skaitļus tiek iegūta \gls{cf}. \Gls{ic} tika aprēķinātas 
\ref{app:B_izmaksas} un \ref{app:C_izmaksas} pielikumos. 
\par
Izmantojot \acrshort{cf} un \acrshort{ic} desmit gadu garumā tiek aprēķināta \gls{icc}. Iegūstot \acrshort{icc} 
tiek aprēķināta \gls{pv} un \gls{npv}, šīs vētības tiek aprēķinātas attiecīgi formulām \ref{app:Projekta_formulas}
pielikumā. Visveidzot tiek aprēķināts \acrshort{irr} un \acrshort{roi}, šie divi rādītāji ir galvenie nosakot 
projekta rentabilitāti un dzīvotspējīgumu.
\par
Abām aternatīvām tika aprēķināti \acrshort{pv} un \acrshort{npv}, šiem aprēķiniem tika izmantotas diskonta
likmes ar 5\% un 10\% vērtību. Salīdzinot abas alternatīvas tika secināts, ka B alternatīva atmaksājās daudz
ātrāk un nes lielāku ieguvumu, nekā C alternatīva. 
\par
Salīdzinot B un C alternatīvu pēc \gls{irr} arī var redzēt ka B alternatīva ir ar lielāku potenciālu.
B alternatīvai \gls{irr} ir 19.39\%, savukārt C alternatīvai tas ir 14.49\%. Attiecīgi abas alternatīvas ir
pelnošas.
\par
Salīdzinot B un C alternatīvu pēc \gls{roi} ir līdzīga pozīcija - 29.66\% pret 10.91\% attiecīgi. Vienkāršojot
šos datus, B alternatīva ir gandrīz trīs reizes ienesīgāka nekā C alternatīva, kā arī par katru investēto eiro
var projektā katru gadu var atpelnīt 30 centus, kas 10 gadu griezumā pārveidojās par 3 eiro, jeb trīskāršēju peļnu
\par
B alternatīva ir daudz izdevīgāka, bet tās ienākumi tiek rēķināti no potencīalā guvumua iespējām. Praktiskais
ieguvums Latvijai noteikti būs daudz lielāks, ja datorika tiks pasniegta visā valstī ikkatram skolniekam.
C alternatīva ir fokusēta uz reāla pakalpojuma sniegšanu, ar samaksu, kur nav jau eksistējošas lietotāju bāzes,
tā kā Start(it).lv ir izglītības fonds, nevis komerciāls uzņēmums, tad reklāma un materiālu pieejamība netiktu
vajadzīgi reklamēta, kā viens no variantiem lietotāju iegūšanai ir sadarbība ar valsts nodarbinātības aģentūru.
Šie faktori spēcīgi ietekmē ieņēmumus, līdz ar to arī visus sekojošos aprēķinus.
\par
Secinot iegūtos rezultātus no abu alternatīvu finanšu analīzes, abas alternatīvas ir finansiāli izdevīgas un
atmaksājas samērā ātri, jau piektajā gadā abos gadījumos tiek iegūta jau tīra peļņa un ir atmaksāti visi 
ieguldījumi. Taču B alternatīva kopumā sniedz daudz lielāku ieguvumu un ir daudz izdevīgāka kopumā.
\par
Nākošā nodaļā tiek viekta risku analīze, šis solis palīdzes noteikt potenciālos riskus, to novēršanas iespējas,
ietekmi uz projekta veiksmīgu izpildi. Risku analīze ir būtiska līdzīgu alternatīvu salīdzināšanai.
