\section{Projekta mērķu apraksts}
Izmantojot rezultātus no iepriekšējās nodaļas dotajā sadaļa tiek izvirzīti mērķi projektam.
\paragraph{}
Lai mērķi būtu viedi izvēlēti un neizrādītos par nesasniedzamiem, tiks izmantoti SMART 
kritēriji katram mērķim
\paragraph{}
\begin{itemize}
    \item Stratēgiskie mērķi
    \item Specifiskie mērķi
    \item Operatīvie mērķi
\end{itemize}
\paragraph{}
Operatīvos iedala vēl smalkāk
\begin{itemize}
    \item Funkcionālie mērķi
    \item Finansiālie mērķi
    \item Ekoloģiskie mērķi
    \item Sociālie mērķi
\end{itemize}
Šim projektam autors izvirza sekojošus mērķus:
\paragraph{}
\textbf{Projekta vispārējais mērķis} 
\textbf{Projekta konkrētais mērķis}
\paragraph{}
Projekta veiksmīgai realizācijai tika izvirzīti arī vairāki operatīvie mērķi
\begin{table}[h!]
    \centering
    \begin{tabular}{|l|p{6cm}|p{5cm}|p{2cm}|}
        \hline
        \textbf{Mērķa klase} & \textbf{Mērķis} & \textbf{Pārbaudes rādītājs} & \textbf{Avoti} \\
        \hline
        Funkcionālā & Izviedot materiālus Skola2030 programmai & Datorikas ieviešana skolās & VISC \\
        \hline
        Funkcionālā & Izveidot skolotāju apmācības materiālus & 1-9 klases mācību materiāli & Iekšējā uzskaite \\
        \hline
        Funkcionālā & Izveidot tālākizglītības kursus & Pārkvalifikācijas programmas materiāli & Iekšēhā uzskaite \\
        \hline
        Finansiālā & Partneru iesastīšana programma & Piesaistīt 2 parternus & Iekšējā uzskaite \\
        \hline
        Sociālā & Apmācīt skolotājus pasniegt datoriku & 30 Apmācīti skolotāji & Iekšejā uzskaite \\
        \hline
        Sociālā & Ieviest programmu skolās & 30 Skolās sāk pasniegt datoriku & IZM statistika \\
        \hline 
    \end{tabular}
\end{table}
\paragraph{}
TIek izvirzīti vairāki mērķi, konkrēti darbojoties ar Start(it) projektu. Tālāk ir jāizveidot dažādi viedi, kā
šo projektu padarīt par veiksmīgāku. Galvenais mērķis ir iesaistīties skola2030 programma, lai to panāktu vajadzētu
modernizēt skolas materiālus, kā arī apmacīt skolotājus. Šie materiāli varētu būt veidoti tā, lai tos arī varētu
izmatot tālākizlgītības nolūkos. Šos kursus varētu piedāvāt valsts nodarbinātības aģentūra.
\paragraph{}
Mērķu sasniegšanai tiek izvirzīti vairāki potenciālie risinājumi
