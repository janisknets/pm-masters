\phantomsection
\section{Projekta mērķu apraksts}
Izmantojot rezultātus no iepriekšējās nodaļas, dotajā sadaļā tiek izvirzīti projekta mērķi.
Lai mērķi būtu viedi izvēlēti un neizrādītos par nesasniedzamiem, tiks izmantoti SMART 
kritēriji katram mērķim
\par
Tālajā Pārvaldības apskata ("Management review") 1981 gada novembra izdevuma tika ievietots Džordža Dorana raksts
ar nosaukumu "There's a S.M.A.R.T. way to write management's goals and objectives", jeb "Ir gudrs veids kā aprakstīt
vadības mērķus". Tajā tiek aprakstīts cik svarīgi un sarežģīti ir izvirzīt tiešus un precīzus mērķus

Ideālā gadījumā mērķiem būtu jābūt izvirzītīem tā, lai tie būtu ar šiem pieciem kritērījiem:
\begin{itemize}
    \item \textbf{S} - Specifiskam (Specific) - konkrēti definiētiem, bez liekvārdības, izprotami cilvēkiem bez papildus paskaidrojumiem
    \item \textbf{M} - Mērāmi (Measurable) - tiem jābūt ar noteiktu lielumu, kvantitāti, kuru var novērtēt, dienas, cena, daudzums utml.
    \item \textbf{A} - Uzdodamam (Assignable) - skaidri definē kurš izpildīs doto uzdevumu.
    \item \textbf{R} - Realistiskam (Realistic) - mērķim jābūt sasniedzamam ar eksistējošiem resursiem
    \item \textbf{T} - Laikā definētiem (Time-related) - jābūt definētam laikam, līdz kuram šim mērķim jābūt pabeigtam.
\end{itemize}
Velāk viens no šiem punktiem tika pielāgots no \textit{Uzdodama} par \textit{Sasniedzamu} (Attainable), tādā veidā šis
akronīms tiek arī izmantots mūsdienu mērķu un uzdevumu veidošānā.
\par
\begin{itemize}
    \item Stratēgiskie mērķi - izpilda projekta misiju, vispārēji definēti, norāda projekta kopējo virzību;
    \item Specifiskie mērķi - nosaka projekta tiešo mērķi un ko tas sasniegs, tiek veidoti tā, lai sasniegtu projekta stratēģiskos mērķus;
    \item Operatīvie mērķi - definē projekta gala rezultāta kvalitāti; tiek izmantoti, lai novērtētu projekta gala stāvokli. Operatīvos iedala vēl smalkāk:
    \begin{itemize}
        \item Funkcionālie mērķi;
        \item Finansiālie mērķi;
        \item Ekoloģiskie mērķi;
        \item Sociālie mērķi.
    \end{itemize}
\end{itemize}
\par
Šim projektam autors izvirza sekojošus mērķus:
\par
\textbf{Projekta vispārējais mērķis} - Viecināt padziļināto datorprasmju attīstību Latvijā
\textbf{Projekta konkrētais mērķis} - Paplašināt padziļināto datorprasmju kursu spektra piedāvājumu Latvijas tirgū
\par
Projekta veiksmīgai realizācijai tika izvirzīti arī vairāki operatīvie mērķi
\begin{table}[!ht]
    \centering
    \begin{tabular}{|p{0.2\textwidth}|p{0.4\textwidth}|p{0.3\textwidth}|p{0.1\textwidth}|}
        \hline
        \textbf{Mērķa klase} & \textbf{Mērķis} & \textbf{Pārbaudes rādītājs} & \textbf{Avoti} \\
        \hline
        Funkcionālā & Nodrošināt daudzveidību & pieejamo apmācību veidu skaits & Iekšējā uzskaite \\
        \hline
        Funkcionālā & Nodrošināt piejamību apgūt mācības dažādos veidos & Piejamības veidu skaits & Iekšējā uzskaite \\
        \hline
        Sociālie & Uzlabot Latvijas DESI rezultātus & Latvijas pozicija pēc DESI & DESI \\
        \hline
        Ekonomiskie & Viecināt nodokļu iemaksu - PVN & Ieņēmumi no PVN no saistītām nozarēm & VID statistika \\
        \hline
        Ekonomiskie & Viecināt nodokļu iemaksu - IIN & Iēņēmumi no IIN no saistītām nozarēm & VID uzskaite \\
        \hline
    \end{tabular}
    \caption{Izvirzītie mērķi}
    \label{table:merki}
\end{table}
\par
Mērķu sasniegšana tiks realizēta uz Start(it).lv fonda bāzes un šī fonda tīmekļa vietni. Fondam jau
eksistē mācību materiāli un nepieciešamā infrastruktūra, taču tā ir jāuzlabo un jāpilnveido.
\par
Mērķu sasniegšanai tiek izvirzītas vairākas alternatīvas. Visu alternatīvu gala produkts būs tīmekļa vietne
ar uzlabotiem mācību materiāliem, kuri būs brīvi pieejami ikvienam interesentam. Alternatīvas piedāvā dažādus
veidus kā sasniegt izvirzīto mērķi - piesaistot citus partnerus fondam, kuri spēs nodrošināt saturu; 
izveidojot jaunu mācību saturu skolām un iesaistīties Skola 2030 programma; Vai arī izveidot jaunu saturu
pieaugušiem, kur varēs apgūt padziļinātās datorprasmes. Šo alternatīvu apraksts un savastarpējā salīdzināšana 
ir aprakstīta 2. nodaļā.