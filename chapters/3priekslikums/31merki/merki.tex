\section{Projekta mērķu apraksts}
Izmantojot rezultātus no iepriekšējās nodaļas dotajā sadaļa tiek izvirzīti mērķi projektam.
Lai mērķi būtu viedi izvēlēti un neizrādītos par nesasniedzamiem, tiks izmantoti SMART 
kritēriji katram mērķim
\par
Tālajā Pārvaldības apskata ("Management review") 1981 gada novembra izdevuma tika ievietos Džordža Dorana raksts
ar nosaukumu "There's a S.M.A.R.T. way to write management's goals and objectives", jeb "Ir gudrs veids kā aprakstīt
vadības mērķus". Tajā tiek aprakstīts cik svarīgi un sarežģīti ir izvirzīt tiešus un precīzus mērķus

Ideālā gadījumā mērķiem būtu jābūt izvirzītīem tā, lai tie būtu ar šiem pieciem kritērījiem:
\begin{itemize}
    \item \textbf{S} - Specifiksam (Specific) - konkrēti definiētiem, bez liekvārdības, izprotami cilvēkiem bez papildus paskaidrojumiem
    \item \textbf{M} - Mērāmi (Measurable) - tiem jābūt ar noteiktu lielumu, kvantitāti, kuru var novērtēt, dienas, cena, daudzums utml.
    \item \textbf{A} - Uzdodamam (Assignable) - skaidri definē kurš izpildīs doto uzdevumu.
    \item \textbf{R} - Realistiskam (Realistic) - mērķim jābūt sasniedzamam ar eksistējošiem resursiem
    \item \textbf{T} - Laikā definētiem (Time-related) - jābūt definētam laikam, līdz kuram šim mērķim jābūt pabeigtam.
\end{itemize}
Velāk viens no šiem punktiem tika pielāgots no \textit{Uzdodama} par \textit{Sasniedzamu} (Attainable), tādā veidā šis
akronīms tiek arī izmantots mūsdienu mērķu un uzdevumu veidošānā.
\par
\begin{itemize}
    \item Stratēgiskie mērķi - izpilda projekta misiju, vispārēji definnēti, norāda projekta kopējo virzību;
    \item Specifiskie mērķi - nosaka projekta tiešo mērķi un ko tas sasniegs, tiek veidoti tā, lai sasniegtu projekta stratēģiskos mērķus;
    \item Operatīvie mērķi - definē projekta gala rezultāta kvalitāti; tiek izmantoti, lai novērtētu projekta gala stāvokli. Operatīvos iedala vēl smalkāk:
    \begin{itemize}
        \item Funkcionālie mērķi;
        \item Finansiālie mērķi;
        \item Ekoloģiskie mērķi;
        \item Sociālie mērķi.
    \end{itemize}
\end{itemize}
\par
Šim projektam autors izvirza sekojošus mērķus:
\par
\textbf{Projekta vispārējais mērķis} - Uzlabot Latvijas DESI rādītājus padziļināto datorprasmju vērtējumu
\textbf{Projekta konkrētais mērķis} - Uzlabot Start(it) fonda tīmekļa vietnes saturu ar moderniem mācību materiāliem 
\par
Projekta veiksmīgai realizācijai tika izvirzīti arī vairāki operatīvie mērķi
\begin{table}[!ht]
    \centering
    \begin{tabular}{|p{0.2\textwidth}|p{0.4\textwidth}|p{0.3\textwidth}|p{0.1\textwidth}|}
        \hline
        \textbf{Mērķa klase} & \textbf{Mērķis} & \textbf{Pārbaudes rādītājs} & \textbf{Avoti} \\
        \hline
        Funkcionālā & Izviedot materiālus Skola2030 programmai & Datorikas ieviešana skolās & VISC \\
        \hline
        Funkcionālā & Izveidot skolotāju apmācības materiālus & 1-9 klases mācību materiāli & Iekšējā uzskaite \\
        \hline
        Funkcionālā & Izveidot tālākizglītības kursus & Pārkvalifikācijas programmas materiāli & Iekšēhā uzskaite \\
        \hline
        Finansiālā & Partneru iesastīšana programma & Piesaistīt 2 parternus & Iekšējā uzskaite \\
        \hline
        Sociālā & Apmācīt skolotājus pasniegt datoriku & 30 Apmācīti skolotāji & Iekšejā uzskaite \\
        \hline
        Sociālā & Ieviest programmu skolās & 30 Skolās sāk pasniegt datoriku & IZM statistika \\
        \hline 
    \end{tabular}
\end{table}
\par
Tiek izvirzīti vairāki mērķi, konkrēti darbojoties ar Start(it) projektu. Tālāk ir jāizveidot dažādi viedi, kā
šo projektu padarīt par veiksmīgāku. Galvenais mērķis ir iesaistīties skola2030 programma, lai to panāktu vajadzētu
modernizēt skolas materiālus, kā arī apmacīt skolotājus. Šie materiāli varētu būt veidoti tā, lai tos arī varētu
izmatot tālākizlgītības nolūkos. Šos kursus varētu piedāvāt valsts nodarbinātības aģentūra.
\par
Mērķu sasniegšanai tiek izvirzīti vairāki potenciālie risinājumi. Gala produkts būs tīmekļa vietnes mācību
materiālu uzlabošana, to var panākt dažādos veidos - gan piesaistot partnerus, kuriem ir iespējas izveidot
šādus materiālus, vai tie jau ir izveidoti; gan eksistējošiem fonda partneriem ieguldīt savas finanses 
satura uzlabošanai.
