\section{Projekta alternatīvu sākotnējais risku izvērtējums}
Atsauce uz pielikumiem - \ref{app:B_sakotnejie_riski} un \ref{app:C_sakotnejie_riski}
\paragraph{}
Paskaidrojums par risku iedalījumu
Iespējamie riski tika klasificēti pēc to izcelsmes vides, kas ir gan projekta ārējie, gan
iekšējie riski. Tika noteikti šādi risku veidi abām alternatīvām:
\begin{itemize}
    \item Saimnieciskie jeb komerciālie riski – riski, kas saistītie ar finansēm, nekvalitatīvu darbu,
cenu izmaiņām, apdrošināšanas izmaksām, inflāciju;
    \item Tehniskie riski – riski, kas saistīti nekvalitatīviem materiāliem un iekārtām, nekvalitatīvu
darba veikšanu;
    \item Tiesiski – politiskie riski – riski, kas saistīti ar ārējās vides (valsts) makrofaktoriem –
nodokļu politika, nemieri valstī, administratīvie ierobežojumi;
    \item Dabas riski – riski, kas saistīti ar dažādām dabas kataklizmām (plūdi, zemestrīces, utml);
    \item Personāla riski – riski, kas saistīti ar personālu.
\end{itemize}
\paragraph{}
Risku vērtēšanas princips,
\paragraph{}
Risku apkopojums pēc to veidiem - Saimnieciskie
\paragraph{}
Risku apkopojums pēc to veidiem - Tehniskie
\paragraph{}
Risku apkopojums pēc to veidiem - Tiesiskie
\paragraph{}
Risku apkopojums pēc to veidiem - Personāla
\paragraph{}
Apkopojums par riskiem
\paragraph{}
ievads nākošā sadaļā


