\section{Projekta alternatīvu sākotnējais risku izvērtējums}
%2.5
Iespējamie riski tika klasificēti pēc to izcelsmes vides, kas ir gan projekta ārējie, gan
iekšējie riski. Tika noteikti šādi risku veidi abām alternatīvām:
%TODO pielikt atsauci uz 11.2 no PMBOK 5th edition
\begin{itemize}
    \item Saimnieciskie riski – riski, kas saistītie ar finansēm, nekvalitatīvu darbu,
cenu izmaiņām, apdrošināšanas izmaksām, inflāciju;
    \item Tehniskie riski – riski, kas saistīti nekvalitatīviem materiāliem un iekārtām, nekvalitatīvu
darba veikšanu;
    \item Tiesiski – politiskie riski – riski, kas saistīti ar ārējās vides (valsts) makrofaktoriem –
nodokļu politika, nemieri valstī, administratīvie ierobežojumi;
    \item Dabas riski – riski, kas saistīti ar dažādām dabas kataklizmām (plūdi, zemestrīces, utml);
    \item Personāla riski – riski, kas saistīti ar personālu.
\end{itemize}
Riski tika sarindoti pēc to iespējas iestāties un ietekmes uz projektu. Jo lielāka iespēja,
ka risks iestāsies vai jo lielāka būs ietekme ja risks iestāsies, jo augstāks būs vērtējums.
Vērtējumi ir sadalīti trīs kategorijās zems, vidējs un augsts. Tiek arī piedāvātas potenciālās
preventetīvās darbības.
\par
Abām alternatīvām tika noteikti riski \ref{app:B_sakotnejie_riski} un \ref{app:C_sakotnejie_riski} pielikumā.
Kopā B alternatīvai tika identificēti 12 riski, kamēr C alternatīvai 13 riski. Šeit arī ir novērojams 
abo projektu līdzība, līdz ar to vairāki riski ir attiecināmi uz abām alternatīvām.
\par 
Skatoties uz saimnieciskiem riskiem B alternatīvai ir viens zems un viens augsts risks. Pirmais risks attiecas
uz iespējamo fonda dalībnieku izstāšanos, taču novērtēs kā zems, jo lielākais finansējums nāk no
Accenture Latvia uzņēmuma, kurš tieši vēlas šo projektu realizēt. Lai mazinātu šo risku tiek ieteikts
viecināt fonda dalībnieku sapratni par šī projekta nozīmību valsts mērogā. Projekta budžeta pārsniegšanas
riskam tiek ieteikts izmantot rūpīgu tēriņu uzskaiti. It sevišķi lielu uzmanību jāpievērš video
filmēšanas izmaksām, jo to svārstība var būtiski ietekmēt projekta budžetu, līdz ar to šis risks
tika novērtēs ar augustu riska pakāpi. C alternatīvia ir līdzīgi pirmie divi riski, taču vēl ir
identificēti papildus viens zems un viens vidējs risks, kopā veidojot četrus dotajā risku veidā.
Cilvēku intereses trūkums ir viens no paredzētiem riskiem, tas ir novērtēs kā zems, lai novērstu
šo risku, tiek ieteikts sadarboties ar valstiskām organizācijām, konkrēti ar \acrshort{nva}, kas
ļautu izmantot šos kursus un reklamēt tos bez liekiem ieguldījumiem reklāmas kampaņās. Vidēji
novērtētais risks runā par izveidoto materiālu neatbilstību cilvēku vajadzībām, lai to novērstu
autors iesaka veikt \glspl{designThinking} sesijas ar potenciāliem gala lietotājiem.
\par
Tehniskie riski abām alternatīvām ir identiski un to novēršana un vērtējumi arī ir vienādi.
Abi vidējie riski var būt novērsti veico kvaliatīvu analīzi izpētes fāzē par doto stāvokli
eksistējošai tīmekļa vietnei. Ir jāizvērtē cik kvalitatīva bija iepriekš izstrādāta vietne.
Cik apjomīgus uzlabojumus būtu jāveic un cik eksistējošā vietne ir gata jaunu mācību materiālu
izvietošanai. Mākoņpakalpojumu sadārdzināšanās risks ir novērtēts kā zems, jo sadārdzinājums
spēcīgi neitekmētu kopējās izmaksas un dotā tirgus situācija vairāk ir tendēta uz tirgus cenu
kritumu dēļ konkurences un jauno tehnoloģiju ieviešanu, kas ļauj mākoņpakalpoju sniedzējiem piedāvāt
lētākus risinājumus. Mācību platformas pārslodzes gadījumā ir arī noteikts zems risks, kurš ir
novēršams izmantojot industrijas standartus un tipiskos risinājumus, kā arī pielietot mākoņpakalpojumu
elastīgo serveru uzturēšanu, kas ļauj novērtēt peiprasījumu skaitu un paaugstināt server kapacitāti, ja
tas ir nepieciešams. Kā pēdējais risks ir pieminēts ļaundabīgu uzbrukumu gadījums gan pašai tīmekļa
vietnei ar tā saucāmiem DDOS uzbrukumiem, gan nekvalitatīvi izstrādāta programmatūra, kā novēršana
šim riskam tiek arī ieteikts izmantot industrijas standartus, kā arī pārliecināties par drošību
izmantojot OWASP tīmekļa vietnē izvietos ieteikumus.
%TODO ielikt atsauci uz OWASP
\par
Tiesiskos riskos tiek novērota neliela starpība dēļ projekta alternatīvu realizācijas. Uz B
alternatīvu ir daudz lielāka ietekme no valsts sektora, jo tas plāno sadarboties ar skolām, kas
automatiski iesaista \acrshort{izm} un tās padoto \acrshort{visc}. \acrshort{visc} prasību
neatbilstība būtība pilnībā var nobloķēt projektu, taču risks ir novērtēs kā zems, jo ar
\acrshort{visc} ir izveidojusies jau ilgstoša sadarbība. Lai novērstu šo risku autors piedāvā
veicināt vēl vairāk sadarbību ar \acrshort{visc} un skaidrot tiem šī projekta svarīgumu.
B alternatīvai arī zems politiskais risks - valsts izglītības reformas apstādināšana un
Skola 2030 projekta atcelšana. Tā kā šis projekts ir Eiropas finansēts un jau tiek izstrādāts
vairākus gadus, tā termiņš ir paredzēts līdz 2030 gadam, ir ļoti maz ticama šīs reformas
atcelšana. Tomēr ja šis risks iestāsies, autors iesaka sākt sadarbību ar individuālām skolām
un piedāvāt sagatavotos materiālus kā bezmaksas produktu tām. Start(it) fonda dotā brīža
vietne tieši tādā veidā jau darbojās. Protams, mūsdienās jebkuram IT izstrādātam projektam
ir jāatcerās par \acrshort{gdpr}, jeb "\gls{gdpr}", tulkojumā - Ģenerālā datu aizsardzības
regula. Regulas prasības ir skaidri aprakstītas un paskaidrotas, līdz ar to risks ir novērtēts
kā zems. Lai novērstu šo risku ir nepieciešams izmantot industrijas standartus un nodrošināt
kvalitatīvu izstrādi, šis risks ir atteicināms uz abām alternatīvām. C alternatīvai savukārt
ir tikai vēl viens zems politiskais risks - \acrshort{nva} sadarbības atteikums. Lai mazinātu
šī riska ietekmi tiek piedāvāts veikt ciešāku skaidrojošo darbu ar \acrshort{nva}, kā arī
atrast citus sadarbības partnerus caur kuriem varētu popularizēt dotos materiālus.
\par
Personāla riski arī ir vienādi abām alternatīvām, jo personāls kopumā atbild par līdzīgu
platformu izstrādi. Pirmais risks, kurš ir novērtēts augsts, kurš ir arī vienīgais augsti
novērtētais risks, ir mācību materiālu izstrādātāju trūkums. Tā kā Latvijā nav liels iedzīvotāju
skaits, tad ir samērā grūti atrast specialistus, kuri būtu pietiekoši zinoši gan datorzinātnēs,
gan skolu pasniegšanas mākā, gan būtu gatavi sagatavot kvalitatīvus mācību materiālus. Šo risku
ir ļoti sarežģīti novērst, autors piedāvā veikt ilgāku specialistu atlasti, kā arī censties
apmācīt daļēji atbilstošus cilvēkus. Otrs ir zemu novērtēts risks par programmētāju kvalitātes
un zināšanu trūkumu. Ņemot vērā ka projekta viens no patroniem ir Accenture Latvia, kurš ir
lielākais IT uzņēmums Latvijā, tad ar šī uzņēmuma pieredzi varēs nodrošināt vajadzīgos darbiniekus.
Protams, ir svarīgi sekot visiem ieteikumiem no industrijas puses, un tipiskiem risinājumiem.
\par
Kopumā riski ir samērā līdzigi, atšķirības var redzēt tiesisko risku jomā, jo alternatīvām
ir atšķirīgas mērķauditorijas. Saimniecisko risku gan ir vairāk C alternatīvai, jo tai ir nepieciešama
tiešā dalība un interese no mērķauditorijas, taja pašā laikā B alternatīvai mērķauditorija automatiski
būs, jo tā fokusējās uz skolām un gala produkta ieviešanu skolās.
\par
Turpinājumā abas alternatīvas tiek salīdzinātas pēc stratēģisko mērķu sasniegšanas. Tiek izvirzīti
kritēriji pēc kuriem salīdzināt dotās alternatīvas un skaidrots šo alternatīvu vērtējums.



