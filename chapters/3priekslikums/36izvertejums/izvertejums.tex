\section{Projekta alternatīvu stratēģiskais izvērtējums}
Lai varētu izvirzīt vienu no abām alternatīvām ir vajadzīgs novērtēt cik lielā mērā atlernatīvas
atbilst izvirzītiem mērķiem un fonda vajadzībā. Autors izvirzīja 9 fakrorus pēc kuriem tika 
salīdzinātas abas alternatīvas. Protams, salīdzinot projektus ir arī svarīgi saprast vai tie
ir finansiāli izdevīgi vai no tiem būs pozitīvi ieguvumi.
\par
Vērtēšanas faktori tika izvirzīti izmantojot \ref{table:merki} tabulā norādītos mērķus; Start(it)
fonda misiju un vīziju, kā arī fonda dalībnieku vajadzības un alternatīvu risku un ieguvumu vērtējumus.
Tālāk tiek aprakstīts katrs no izvirzītajiem faktoriem.
\par
\textit{"Vai tiek nodrošināta digitālo prasmju pieejamības veicināšana"} ir pirmais faktors,
tas cenšas novērtēt to pirmo no funkcionāliem mērķiem - paplašināt datorprasmju pieejamību. Jo vairāk
un plašāk doto risinājumu varēs izmantot uz dažādām platformām, tiktu nodrošinātas apmācības klātienē
un distancionāli, tiek novērtēts cik daudz jaunās tehnoloģijas tiktu iesaistītas.
\par
\textit{"Vai tiek nodrošināta digitālo prasmju daudzveidības palielināšana?"} faktors vērtē cik
lielā apmērā tiek palielināta daudzveidība digitālo prasmju apmācībās Latvijā. Uz doto brīdi vairāk
var atrast programmēšanas apmacības, taču viens no mērķiem ir panākt plašāku izglītību tieši 
datorzinātnē un datorprasmēs kopumā, kas sevī itever arī audio, video, bilžu apstrādi, tehnisko
risinājumu pieejamība, problēmu risināšanas paņēmieni u.c. Jo plašāku klāstu spēj piedāvāt konkrētā
alternatīva - jo vairāk punktu tā iegūst
\par
Nākošais faktors ir \textit{"Vai projekts nodrošina padziļinātām datorprasmēm mūzizglītības pieejamību?"}. 
Tajā tiek vērtēta iespēja materiālus apgūt jebkuram Latvijas iedzīvotājam, tā lai tie būtu pieejami ikvienam
un cilvēki spētu gūt no tiem labumu un pielietot ikdienā sev nepieciešamos nolūkos.
\par
Svarīgs faktors ir arī atpazīstamības veicināšana pašam fondam, lai tam varētu piesaistīt vairāk
fonda dalībniekus. Līdz ar to tiek izvirzīts \textit{"Vai projekts nodrošina fonda popularizēšanu valstī"}.
Tajā tiek novērtēts cik daudz tiks veicināta atpazīstamība, jo vairāk un plašāk tiek iesaistītas dažādas
platformas, jo vairāk punktus iegūst alternatīva.
\par
Tā kā Valsts kopējais tēls IT jomā var piesaistīt papildus investīcijas un veicina kopēju atpazīstamību 
Eiropā ka IT vadošo valsti, Latvijai vajadzētu uzalbot savus rādītājus \acrshort{desi}, tāpēc tiek izvirzīts 
\textit{"Vai projekts uzlabo DESI radītājus valstī?"} faktors. Tas novērtē cik lielā mērā dotais projekts varētu
ietekmēt dažādus \acrshort{desi} rādītājus, jo vairāk rādītāju tiek ietekmēti, jo vairāk punktus iegūst
alternatīva.
\par
Fonda dalībniekiem ir ļoti būtiski audzēt darbinieku skaitu ITK sektorā kopumā, jo uz doto brīdi valstī
tiek asi izjusts darbinieku trūkums, līdz ar to tiek ieveists faktors 
\textit{"Vai projekts palielinās ITK sektorā strādājošo skaitu?"}
\par
Viens no projekta mērķiem tika izvirzīts nodokļu ieguvumu palielināšana valstī. Līdz ar to kā viens no faktoriem
tiek izvirzīts \textit{"Vai dotais projekts palielinās nodokļu iemaksas valstī?"}. Šis ir viens no rādītājiem,
kurš ietekmē projekta kopēju veiksmīgumu. Jo bez nodokļu ieņēmumu uzlabošanās netiks sasniegts prognozētais
labuma guvums un projekta investīcijas neattaisnosies.
\par
Otrs ietekmējošais faktors uz projekta veiksmīgu izpildi, kaut arī mazākā apjomā, ir \textit{"Vai projekts ir
finansiāi attaisnojams"}. Dotajā faktorā tiek apskatīti finansiālie rādītāji projektam un tiek izvērtēts
tā potenciālais guvums pret sniegto finansējumu un ieguldījumiem.	
\par
Vērtēšanai tika izveidota tabula ar visiem faktoriem un tiem tika iedoti attiecīgi vērtējumi no vienas līdz
piecām ballēm. Abas alternatīvas savstarpēji tika salīdzinātas \ref{app:B_C_strategiskais_vertejums} pielikumā
ievietotā tabulā. Katrai alternatīvai tad tika iedots vērtējums pēc katra faktora. Vēlāk tika saskaitīts kopējais
baļļu skaits.
\par
Pēc punktu skaita B alternatīva saņēma 34 punktus, bet C alternatīva 33 punktus no 45 iespējamiem.
Punktu tuvība vēlreiz norāda uz projekta alternatīvu salīdzinošo tuvību. Tabula gan norāda ka katrai no alternatīvām
ir savas stiprās un vājās puses. B alternatīva kopumā vairāk atbilst fonda vajadzībām, abas alternatīvas 
samērā vienlīdzīgi spēj sasniegt projekta izvirzītos mērķus. C alternatīva spēj gūt lielāku labumu un ietekmi 
uz nodokļu ienākumiem. 
\par
Nākošā nodāļā tiek veikts alternatīvu salīdzinājums un labākās alternatīvas izvirzīšana pilnvērtīgam
projktam. Alternatīvas tiek salīdzinātas pēc iepriekš aprakstītā apakšnodaļās. Tiek izvērtēti finanšu
rādītāji, ieguvmi, budžets, uzturēšnas izmaksas, riski un citi faktori.
