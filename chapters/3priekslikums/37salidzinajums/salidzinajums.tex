\section{Projekta alternatīvu salīdzinājums un labākās alternatīvas izvēles pamatojums}
Izmantojot otrajā nodaļā veikto alternatīvu salīdzināšanas darbus autors nonāca pie vairākiem rezultātiem,
kuri ir apkopoti un paskaidroti zemāk esošās tabulā un tās parakstā.
\begin{table}[!ht]
    \centering
    \begin{tabular}{|p{0.2\textwidth}|p{0.35\textwidth}|p{0.35\textwidth}|}
        \hline
        \textbf{} & \textbf{B Alternatīva} & \textbf{C Alternatīva} \\
        \hline
        Projekts & Start(it) tīmekļa vietnes pielāgošana Skola 2030 projektam & Start(it) tīmekļa vietnes pielāgošana tālākizglītības vajadzībām \\
        \hline
        Projekta ilgums & 14.7 & 15.2 \\
        \hline
        Projekta budžets & 733 092.98 € & 826 182.17 € \\
        \hline
        Projekts atamksāšanas gads & 2 & 3 \\
        \hline
        Uzturēšanas izdevumi & 39 285.24 € & 29 536.20 € \\
        \hline
        Prognozējamie ieguvumi 3-5. gadā & 670 000 € & 715 000 €\\
        \hline
        PV pie r15\% & 1 602 190.58 € & 1 463 535.89 € \\
        \hline
        IRR, \% & 62.35 & 48.81 \\
        \hline
        ROI, \% & 60.60 & 48.37 \\
        \hline
        Riska pakāpe & Zema & Zema \\
        \hline
        Stratēģiskās nozīmes vērtējums & 34/45 & 33/45 \\
        \hline
        Vieta & 1 & 2 \\
        \hline
    \end{tabular}
    \caption{Alternatīvu salīdzinājums}
    \label{table:salidzinajums}
\end{table}
\par
Kā jau minēts, B alternatīvas piedāvājums ir labot eksistējošos mācību materiālus Start(it) fonda tīmekļa
vietnē Skola 2030 projektam. Kā arī izveidot jaunus mācību materiālus. Savukārt C alternatīvas mērķis ir
izveidot jaunu saturu mūžizglītības nolūkiem, kuru varētu brīvi izmantot ikviens Latvijas iedzīvotājs.
Abas alternatīvas kopumā uzlabo eksistējošo tīmekļa vietni un tai pievieno jaunus mācību un
video materiālus. Lielākā daļā rādītāju abas alternatīvas atradās ļoti tuvu viena otrai. 
\par
B alternatīv aizņem nedaudz mazāk laika uz tā rēķina, ka C alternatīvai ir nepieciešams filmēt 
vairāk video materiālu, līdz ar to aptuvenais garums ir arī lielāks. Starpība veidojas vien par
pusmēnesi, jāņem vērā ka abi projekti tiek prognozēti uz vairāk nekā gadu, līdz ar to 2 nedēļu starpība
ir samēra tuvi rādītāji.
\par
Budžetu starpība arī nav ļoti liela, lielākā starpība arī šeit viedojās no video filmēšanas izmaksām.
B alternatīvas kopējais nepiciešamais finansējums sastāda 733 092.98 €, kamēr C alternatīvai ir nepieciešami
826 182.17 €. Dotajā gadījumā gan attālums ir jau jūtami lielāks starp abiem rādītājiem.
\par
Tā kā B projekta ieguldījumi ir zemāki, tad tos arī ātrāk var atmaksāt. un no tā veidojās atmaksāšanās
gadu starpība. Taču ja salīdzinātu mēnešos, tad šie rezultāti būtu joprojām ļoti tuvi. B alternatīva 
finanšu ieguldījums nesīs pietiekošu labumu valstij kopumā nodokļu nomaksas veidā 2 gadu laikā, kamēr
C alternatīva to panāktu 3 gadu garumā.
\par
Uzturēšana B alternatīvai ir nedaudz dārgāķa, jo B atlernatīvā ir arī jānodrošina skolotāju apmācības
kursi ik ceturksni, kamēr C altenratīvai ir tikai uzturēšanas izmaksas un 2 algoti darbinieki 
projekta uzturēšanai. Prognozējamie vidējie ienākumi pēdējos 3 gados C atlernatīvai ir lielāki, dēļ 
tā, ka tā nodrošina lielāku \acrshort{iin} kāpumu. Taču dēļ tā ka ir nepieciešami zemāki ieguldījumi
B alternatīvā, tā joprojām paliek finansiāli izdevīgāka, to norāda arī IRR un ROI procenti.
\par
Abām alternatīvām tiek nozīmēta zema risku pakāpe, jo lielākā daļa risku ir viegli novēršami un tiem
nepiemīti lielas atšķirības kopumā. Līdz ar to šī aile uz kopēju vērtējumu neatstāj lielu ietekmi.
\par
Startēģiskā salīdzinājumā varēja novērot projektu līdzības turpinājumu, taču B alternatīva ieguva
par vienu punktu vairāk kopumā nekā C alternatīva - 34 pret 33 punktiem no 45 potenciālo punktu
skaita. 
\par
Salīdzinājuma rezultātā tiek izvēlēta B alternatīva, jo tā ātrāk atmaksājās, ieguva nedaudz lielāku
punktu stratēģiskā salīdzinājumā, kā arī tai bija mazāk risku. Pozitīvais ir tas, ka pat ja C 
alternatīva dotajā brīdī ir jānoraida, taču nākotnē to joprojām varētu ieviest, jo lielākā daļa
darbū sakrīt abām alternatīvām, vienīgais papildus darbs būtu attiecībā pret pašu mācību materiālu
sagatavošanu un video materiālu filmēšanu.
\par
Izvēloties šo alternatīvu tika izveidots projekta priekšlikums kuru var atrast \ref{app:Projekta_priekslikums}
pielikumā. Ralizācija ir paredzēta %TODO pievienot dienu skaitu
dienās. Projekts tiktu uzsākts 2019. gada 14. jūnijā. Tas tiktu pabeigs 2020 gada 2 februārī. Projekta kopējais
budžets sastādītu 733 092.98 €. Šis projekts risinātu konkrēto problēmu - Paplašināt padziļināto datorprasmju 
kursu spektra piedāvājumu Latvijas tirgū
trūkums Latvijā. Tas tiktu paveikts pielāgojot esošo Start(it) fonda tīmekļa vietni un sagatavojot jaunos 
mācību materiālus pēc Skola 2030 prasībām, kā arī vēlāk šie materiāli tiktu ieviesti lietošanai skolās.
Projetks arī paredz skolotāju izglītošanu kā darboties ar šiem materiāliem. Šis projekts arī risina 
izvirzīto vispārējo problēmu - Padziļināto digitālo prasmju pieejamības trūkumu Latvijā.
\par
Šīs nodaļas ietvaros tika veikta 3 alternatīvu izvirzīšana, no kurām viena tika uzreiz atskatīta no 
turpmākās salīdzināšanas. Vēlāk 2 atlerantīvas tika savstarpēji salīdzinātas pēc vairākiem faktoriem.
Nodaļas beigās B alternatīva tika izvirzīta kā labākā. Nākošā nodaļā tiek izstrādāta projekta rokasgrāmata
izvirzītajai alternatīvai. paredzot uzdevumus, kurus būs jāveic katrā no projekta fāzēm.