\phantomsection
\subsection{B un C alternatīvas realizācijā iegūstamo produktu apraksts}
%2.3.1
Lai labāk izpētītu B un C alternatīvas sākotnēji tika izstrādāts saraksts ar iegūstāmo produktu
aprakstu, tos var atrast \ref{app:B_detalizetais_aprkasts} un \ref{app:C_detalizetais_aprkasts} pielikumā.
Dotā apkašnodaļā tiek paskaidrota informācija par abu alternatīvu iegūstamiem produktiem.
\par
B alternatīvas galvenais fokuss ir uz skolu materiāliem. Dotā alternatīva kopumā ir četras sastāvdaļas:
Izglītības materiāli pamatskolām un vidusskolām; Palīgmateriāli skolotājiem un 30 skolotāji apmmācīšana.
Šie uzdevumi paredz paralēli uzlabot mācību materiālus portālā www.starit.lv un sagatavot skolotājus
apmācošus materiālus, jo sākotnējā izpēte parādīja, ka Latvijā nav pietiekoši daudz skolotāju, kuri varētu
pasniegt datorikas priekšmetu. Materiāliem ir jābūt izveidotiem saskaņā ar Skola2030 prasībām no \acrshort{visc}.
\par
C alternatīva tiek fokusēta uz pieaugušo pārkvalicēšanos un tālākizlgītību. Šeit ir galvenie mērķi ir pielāgot
eksitējošo saturu, kā arī izveidot jaunu pieaugušajiem. Galvenais mērķis ir iemācīt izmantot padziļinātās 
digitālās prasmes ikdienā savā darbā. Viens no izveidotajiem apmācības kursiem būtu veidots izmantojot eksistējošos
materiālus no Bootcamp apmācības, kuras piedāvā Accenture Latvia, ar mērķi apmācīt cilvēkus pamata zināšanām, 
kas ļautu cilvēkam uzsākt darbu IT jomā. 
\par
Abām alternatīvām ir vajadzīgs vienāds tehniskais nodrošinājums - mājaslapa un vieta kur uzturēt video materiālus.
Abu alternatīvu gadījumā būs arī nepieciešams filmēt materiālus, kurus vēlāk būs jārediģē attiecīgiem specialistiem.
Uzturēšanas ziņā arī abos gadījumos būs nepieciešami līdzīgi cilvēki.
\par
Mācību materiālu saturs gan diezgan atšķirās. B alternatīvas gadījuma saturs tiek veidots pēc striktiem \acrshort{visc}
nosacījumiem skolām un skolēniem, kā šīs alternatīvas pamatauditoriju. B alternatīvas iznākums būs pieejams arī
pārējai publikai. Tajā pašā laikā C alternatīva ir vairāk orientēta uz pieaugušajiem un palīdz cilvēkiem atrast
jaunu vai labāku darba vietu, vai arī palīdz uzsākt savā darba vietā izmantot jaunas prasmes, kas palielinātu 
produktivitāti.
