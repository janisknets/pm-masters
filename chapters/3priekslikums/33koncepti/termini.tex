\subsection{B un C alternatīvas darbu un termiņu noteikšana}
Abi produkti ir līdzīgi, taču B alternatīva paredz nedaudz vairāk darba, jo ir npieciešama pielāgošanās VISC, tajā pašā
laikā programmai jābūt gatavai.
\paragraph{}
Projekts var tikt iedalīts jebkāda skaita fāzēs, kur viena fāze ir
loģisku ar projektu saistītu aktivitāšu kopums, kurš noved pie konkrēta starprezultāta, kā arī tās ir
laikā nošķirtas %TODO citation needed
\paragraph{}
Fāzes:
\begin{itemize}
    \item Starta – tiek izpētīta un izstrādāta projekta ideja, iesaistītās puses, vai projektam ir
    nepieciešamais atbalsts, kādi ir nepieciešamie rezultāti, kā arī tas tiek saskaņots ar projekta
    uzdevuma devēju;
    \item Plānošanas – seko pēc projekta starta fāzes apstiprināšanas, tiek izstrādāti projekta
    gaitas, laika un termiņu plāna, notiek resursu, dažādu izmaksu un finanšu plānošana. Svarīgi
    ir šajā fāzē noteikt robežstabus un izvērtēt risku vadību;
    \item Izpētes – paredz stāvokļa izvērtēšana, izpildītāju, nepieciešamo sadarbības partneru
    klāstu un piedāvājumu, likumdošanas aktu izpēti;
    \item Pamatkoncepcijas – potenciālo objektu, sadarbības partneru, mērķauditorijas un
    projekta gala produkta konceptuālā risinājuma izstrāde;
    \item Detaļkoncepcija – tajā tiek sastādīti izmantojamo programmatūru, materiālu specifikāciju
    izstrāde, tiek veikts darbs ar konkursu izstrādi darba uzdevumu veicējiem, potenciālajam
    darbaspēkam;
    \item Realizācijas – paredz reālo darbu veikšanu, kas saistīta ar iepriekš izstrādāto projekta
    plānu, lai tiktu izpildītas projekta specifikācijas, šajā fāzē tiek plānota vislielākās izmaksas;
    \item Ieviešanas – tajā tiek veikta projekta gala rezultāta testēšana – programmatūra,
    produkts, iekārtas, tiek novērtētas un novērstas kļūdas;
    \item Nobeiguma – pieņemtās projekta komandas atbrīvošana, dalībnieku darba novērtēšana
    un pieredzes apkopošana, nerealizēto darbību izvērtēšana, faktisko izmaksu kalkulācija,
    finanšu un projektu atskaites sagatavošana;
\end{itemize}
\paragraph{}
Tika iztrādāts šāds modelis, skatīt pieliktumus \ref{app:pielikums4} un \ref{app:pielikums5}
\paragraph{}
Uzveram lielākās atšķirības
\paragraph{}
Apkopojums par nepieciešamo laiku abām alternatīvam
\paragraph{}
paskaidrojums par katru fāzi un atšķirībām tajās fāzēs
\paragraph{}
Starta un plānošanas fāzes būs līdzīgas
\paragraph{}
Izpētes fāzē vairāk laika vajadzēs VISC izpētei
\paragraph{}
Līdzvērtīgi tiks izmantoti resursi Detaļkoncepcija
\paragraph{}
Realizācija B alternatīva atkal būs garāka
\paragraph{}
Ieviešanai vajadzētu būt līdzīgai
\paragraph{}
Summary
