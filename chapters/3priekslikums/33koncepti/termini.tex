\subsection{B un C alternatīvas darbu un termiņu noteikšana}
B alternatīva fokusējās uz pamatskolu un vidusskolu satura pārveidošanu un uzlabošanu, savukārt C alternatīvas
galvenais uzdevums ir izveidot izglītojošos materiālus tālākizglītības nolūkos. Vēlāk dažus no tiem pielāgojot
tehnikumu un vidusskolu saturam.
\paragraph{}
Projekts var tikt iedalīts jebkāda skaita fāzēs, kur viena fāze ir
loģisku ar projektu saistītu aktivitāšu kopums, kurš noved pie konkrēta starprezultāta, kā arī tās ir
laikā nošķirtas %TODO citation needed
\paragraph{}
Fāzes:
\begin{itemize}
    \item Starta – tiek izpētīta un izstrādāta projekta ideja, iesaistītās puses, kā arī vai projektam ir
    nepieciešamais atbalsts, kādi ir nepieciešamie rezultāti, kā arī tas tiek saskaņots ar projekta
    uzdevuma devēju;
    \item Plānošanas – seko pēc projekta starta fāzes apstiprināšanas, tiek izstrādāti projekta
    gaitas, laika un termiņu plāna, notiek resursu, dažādu izmaksu un finanšu plānošana. Svarīgi
    ir šajā fāzē noteikt robežstabus un izvērtēt risku vadību;
    \item Izpētes – paredz stāvokļa izvērtēšana, izpildītāju, nepieciešamo sadarbības partneru
    klāstu un piedāvājumu, likumdošanas aktu izpēti;
    \item Pamatkoncepcijas – potenciālo objektu, sadarbības partneru, mērķauditorijas un
    projekta gala produkta konceptuālā risinājuma izstrāde;
    \item Detaļkoncepcija – tajā tiek sastādīti izmantojamo programmatūru, materiālu specifikāciju
    izstrāde, tiek veikts darbs ar konkursu izstrādi darba uzdevumu veicējiem, potenciālajam
    darbaspēkam;
    \item Realizācijas – paredz reālo darbu veikšanu, kuri tika iepriekš izstrādāti projekta
    plānā, lai tiktu izpildītas projekta specifikācijas, šajā fāzē tiek plānota vislielākās izmaksas;
    \item Ieviešanas – tajā tiek veikta projekta gala rezultāta testēšana – programmatūra,
    produkts, iekārtas, tiek novērtētas un novērstas kļūdas;
    \item Nobeiguma – pieņemtās projekta komandas atbrīvošana, dalībnieku darba novērtēšana
    un pieredzes apkopošana, nerealizēto darbību izvērtēšana, faktisko izmaksu kalkulācija,
    finanšu un projektu atskaites sagatavošana;
\end{itemize}
\paragraph{}
Tika iztrādāts šāds modelis, skatīt pieliktumus \ref{app:B_detalizetais_aprkasts} un \ref{app:C_detalizetais_aprkasts}
\paragraph{}
Lielākā starpība starp abiem produktiem ir to galvenā mērķauditorija B Alternatīvas gadījumā tie ir skolnieki, savukārt,
C alternatīvas gadījumā tas var būt jebkurš latviešu valodā runājošs cilvēks, kurš vēlas mainīt savu specialitāti.
Nepieciešamo darbu saraksts ir diezgan līdzīgs.
\paragraph{}
Apkopojums par nepieciešamo laiku abām alternatīvam
\paragraph{}
paskaidrojums par katru fāzi un atšķirībām tajās fāzēs
\paragraph{}
Tā kā projekti ir ar līdzvētīgiem mērķiem, tad starta un plānošānas fāzes būs diezgan līdzīgas. Vienīgais papildus darbs
B alternatīvas gadījumā ir papildus kursu ieplānošana, laikā.
\paragraph{}
Izpētes fāzē parādās lielākas atšķirības starp abiem projektiem. B alternatīvai ir nepieciešams izpētīt VISC prasības,
pārbaudīt eksistējošos Start(it).lv. C alternatīvai ir nepieciešams sagatavot tikai tālākizglītības kursus, attiecīgi
izpētes fāze ir vienkāršāka, jo nav specifisku izglītības iestāžu norādījumu.
\paragraph{}
Pamatkoncepcija un detaļkoncepcijas fāzēs atšķirības var redzēt vēl vairāk. B alternatīvai ir 
nepieciešams konkrēts darba spēks kurš varētu palīdzēt izveidot mācību materiālus; bet mācību materiālus C alternatīvai
varētu veidot uz Bootcamp programmas bāzes, līdz ar to nav nepieciešams veidot atvērtu konkursu satura veidotāju meklēšanai.
\paragraph{}
Realizācijas fāzē B Alternatīvai atkal papildus nāk skolotāju apmācība 30 dienu laikā; C alternatīva predz tikai mācību materiālu
filmēšanu un mājaslapas pārstrādi (kas arī notiks B alternatīvā). Filmēšana B alternatīvas gadījumā būs plašāka, jo ir nepieciešams
nofilmēt plašāku materiālu saturu, kā arī pamata video materiālu pārstrāde B alternatīvas gadījumā aizņems vairāk laika.
\paragraph{}
Ieviešanas fāzes būs samērā līdzīgas - pamata produkts, proti tīmekļa vietne www.startit.lv tiks izvietoti jaunie materiāli un
tiks piedāvāti ikvienam interesentam.
\paragraph{}
Apokopojot visas atšķirības starp B un C alternatīvām var secināt, ka B alternatīva prasīs lielākus darba ieguldījumus un veido
apjomīgāku produktu nekā C alternatīva. Taču B alternatīvas gala produkts ir arī ar lielāku ietekmi uz sabiedrību un dos lielāku
gala vērtību.
