\subsection{B un C alternatīvas izmaksas aprēķins}
Lai varētu sekmīgi realizēt projektu sākotnēji jāveic izmaksu plānošana, lai
varētu veiksmigi iegūt nepieciešamo finansējumu projektam. Zemāk tiek apskatīti
B un C alternatīvas izmaksu aprēķini.
\par
Tabulas ar šiem provizoriskiem izmaksu aprēķiniem var apskatīt \ref{app:B_izmaksas} un \ref{app:C_izmaksas}
pielikumos attiecīgi. Tabulās dati ir sagrupēti pēc vairākām izmaksu katerogrijām: vadīšanas,
personāla, biroja uzturēšanas, produkta izstrādes un ieviešansa izmaksas, skolotāju kursu izmaksas. 
\par
Kopējais finansējums, kurš ir nepieciešams, lai realizētu B alternatīvu, ir 733 092.98 €, lielāko
summu no šiem tēriņiem sastāda video materiālu filmēšanu - 522 720.00 €. Šī salīdzinoši lielā summa
veidojās no aprēķina, kur katra filmēšnas minūte, ar visu nepieciešamo personālu, tehniku un turpmāko
apstrādi kopā izmaksā 300 €. Tā kā ir plāns nofilmēt astoņus piecmadsmit minūšu mācību video katrai
klasei, tad kopā sanāk 1440 video minūtes ko filmēt. Nelielas izmaksas ir ieplānotas skolotāju pilot 
mācību kursu novadīšanai - 2144.12 €. 
\par
Otrās alternatīvas izmaksas ir diezgan līdzīgas B alternatīvai, jo kopējais projekts piedāvājums ir ļoti
līdzīgs. Lielākās atšķirības ir video materiālu cenā un apjomā - kopējās izmaksas 826 182.17 €, kas tiek
aprēķināts 1680 video minūtēm - piecmadsmit minūšu gari video septiņām dažādām tēmām, kopā sešpadsmit
video katrai tēmai. Tā kā ir jāfilmē vairāk video, tad arī ir izmaksas par šo pozīciju ir augstākas - 
609 840.00 €. Dotajai alternatīvai arī nav ieplānoti pilotkursi, kas samazina tādas izmaksas.
\par
Ta kā projekti ir samērā līdzīgi, tad arī pārējās pozīcijas ir līdzīgas abām alternatīvām.
Projekta vadības izmaksas kopā sastāda 99 770.13 € B alternatīvai, savukārt C alternatīvai, 
ar nedaudz garāku projekta garumu tās ir 103,028.95 €. Šīs izmaksas ir par komandu kura sastāv 
no 3 cilvēkiem - projekta vadītāja un diviem asistentīem, kā arī arējo personālu - grāmatvedi. Vēl 
tiek aprēķinātas izmaksas saistībā ar licencēm un tehnisko nodrošinājumu programmētājiem izstrādes laikā,
tās sastāda kopā 726.00 €. Abām alternatīvām tiek aprēķināts 10\% liela rezerve neparedzētiem gadījumiem,
B alternatīvai tie sastāda 66 644.82 €, kamēr C alternatīvai tie ir 75 107.47 €. 
\par
Kopumā lielākās izmaksas abām alternatīvām veido video filmēšana, taču C alternatīvai video garums
ir lielāks, līdz ar to arī kopējās paredzētās izmaksas ir lielākas. Kopumā var secināt, ka B alternatīvai
ir nepieciešamas zemākas izmaksas, gala rezultātā gan tiek ieguts nedaduz mazāk video materiāla, taču tiek
sagatavoti skolotāji mācību materiālu pasniegšanai.
\par
Dažādu pakalpojumu aprēķiniem tika izmantoti dažādas tīmekļa vietnes, kur tiek piedāvāti šādi pakalpojumi.
%TODO add refrences to these descriptions?
Materiālu minūšu apjoms tiek aprēķināts izmantojot skaitļus no sasniedzamajaiem rezultātiem potenciālajā
darbu sarakstā.
\par
Nākošā apakšnodālā tiek veikts B un C finansiālais salīdzinājums. Tiek apskatīts 
ieņēmumu un izdevumu prognozēšana, kā arī ekonomiskās efektivitātes noteikšana pēc 
projekta ieviešanas.

  